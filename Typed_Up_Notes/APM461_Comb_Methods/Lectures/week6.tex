\chapter{Week 6}

We continue our discussion of relevant probabilistic results

\section{The Chernoff/Bernstein/Hoeffding Bounds}

These results provide upper bounds on the deviation of a sum of i.i.d bounded random variables from its expected value. Let $X_1, \ldots, X_n$ be i.i.d with value either 0 or 1, with distribution so that 

\[\Pr[X_i = 1] = p \quad \Pr[X_i = 0] = 1-p\]

with $0 \leq p \leq 1$. Then for $t \geq 0$, and setting $X = \sum X_i$, we have that 

\[\Pr[|X - np| > t] \leq e^{-\frac{t^2}{3n}}\]

Taking $\varepsilon = \dfrac{t}{n}$, we get 

\[\Pr[|X - np| > \varepsilon n] \leq e^{-\frac{\varepsilon^2n}{3}}\]

\begin{lemma}
    Set $Z_i = X_i - p$, and $Z_n = \sum_{i=1}^n Z_i$. Then for any $c > 0$.

    \[\Pr[\sum_i Z_i > t] \leq \frac{(pe^{c(1-p)}+(1-p)e^{-cp})^n}{e^{ct}}\]
\end{lemma}

\begin{proof}
    We have 

    \begin{align*}
        \Pr[Z_n > t] & = \Pr[cZ_n > ct] \\
        & = \Pr[e^{cZ_n} > e^{ct}] \\
        & \leq \frac{\mathbb{E}[\prod_i e^{cZ_i}]}{e^{ct}} \tag{By Markov} \\
        & = \frac{\prod_i \mathbb{E}[e^{cZ_i}]}{e^{ct}} \\
        & = \frac{(pe^{c(1-p)} + (1-p)e^{-cp})^n}{e^{ct}}
    \end{align*}
\end{proof}

More importantly, we have the following inequalities:

\[\Pr[|X - np| > \omega(\sqrt{n})] = o(1)\]
\[\Pr[|X - np| > \Omega(n)] = e^{-\Omega(n)}\]

\section{The Erd\"{o}s-Renyi Graph}

The Erd\"{o}s-Renyi model is a way of creating a random graph that has applications in probabilistic methods, often used to prove the existence of graphs with various properties. 

\begin{definition}
    Let $n \in \N, p \in [0,1]$. Then the \textbf{Erd\"{o}s-Renyi graph} $G(n,p)$ has vertex set $[n]$, and for each $\{i,j\} \in {[n] \choose 2}$, the edge $(i,j)$ is in the edge set with probability $p$. 
\end{definition}

Now, we say that vertices $v_i,v_j,v_k$ for a \textbf{triangle} there are edges between each of them. We ask the following: what is the distribution of the number of triangles in $G(n,p)$? 

For each $\{i,j\}$, we define a randome variable that encdoes when an edge exists:

\[Z_{ij} = \begin{cases}
    1 & \text{ with probability $p$} \\ 0 & \text{ with probability $1-p$}
\end{cases}\]

Then for each triple $\{i,j,k\} \in {[n] \choose 3}$, we have another random variable encoding when a triangle between the vertices is present: 

\[X_{i,j,k} = \begin{cases}
    1 & Z_{ij} = Z_{jk} = Z_{ik} = 1 \\ 0 & \text{ otherwise}
\end{cases}\]

\[X = \sum_{(i,j,k) \in {[n] \choose 3}} X_{i,j,k}\]

Let's compute the expectation and variance of $X$. We have that 

\[\mathbb{E}[X] = \sum_{(i,j,k)} \mathbb{E}[X_{i,j,k}] = p^3{n \choose 3}\]

The variance computation is much more complicated: 

\begin{align*}
    Var(X) & = \mathbb{E}\left[\left(X - p^3 {n \choose 3}\right)^2\right] \\
            & = \mathbb{E}\left[\left(\sum_{i, j, k} (X_(i, j, k) - p^3)\right)^2\right] \\
            & = \mathbb{E}\left[\sum_{(i, j, k), (i', j', k') \in {[n]\choose 3}} (X_(i, j, k) - p^3) (X_(i', j', k') - p^3)\right] \\
            & = \sum_{(i, j, k), (i', j', k') \in {[n]\choose 3}} \mathbb{E}[(X_{i, j, k} - p^3) (X_{i', j', k'} - p^3)] \\
            & = \sum_{(i, j, k), (i', j', k') \in {[n]\choose 3} : |(i, j, k) \cap (i', j', k')| \leq 1} \mathbb{E}[(X_{i, j, k}- p^3) (X_{i', j', k'} - p^3)] \\
            & \quad + \sum_{(i, j, k), (i', j', k') \in {[n]\choose 3} : |(i, j, k) \cap (i', j', k')| = 2} \mathbb{E}[(X_{i, j, k} - p^3) (X_{i', j', k'} - p^3)] \\
            & \quad + \sum_{(i, j, k), (i', j', k') \in {[n]\choose 3} : |(i, j, k) \cap (i', j', k')| = 3} \mathbb{E}[(X_{i, j, k} - p^3) (X_{i', j', k'} - p^3)] \\
            & = \sum_{(a, b, c, d) \in {[n]\choose 4}} \mathbb{E}[(X_{a, b, c} - p^3) (X_{a, b, d} - p^3)] + \sum_{(i, j, k) \in{[n]\choose 3}} \mathbb{E}[(X_{i, j, k} - p^3)^2] \\
            & = \sum_{(a, b, c, d) \in {[n]\choose 4}} (\mathbb{E}[X_{a, b, c} X_{a, b, d}] - \mathbb{E}[p^3 X_{a, b, c}] - \mathbb{E}[p^3 X_{a, b, d}] + \mathbb{E}[p^6]) + \sum_{(i, j, k) \in {[n]\choose 3}} \mathbb{E}[(X_{i, j, k} - p^3)^2] \\
            & = \Theta(n^4) (p^5 - p^6) + \Theta(n^3)(p^3 - p^6)
\end{align*}

By Chebyshev, we have 

\[\Pr[|X - {n \choose 3}p^3| > t] \leq O\left(\frac{n^4p^5 + n^3p^3}{t^2}\right)\]

and applying $p = \dfrac{1}{2}$, 

\[\Pr[|X - \frac{1}{8}{n \choose 3}| > t] \leq O\left(\frac{n^4}{t^2}\right)\]

for $t = \omega(n^2), O(\frac{n^4}{t^2}) \in o(1)$. So 

\[X \in \left[\frac{1}{8}{n \choose 3} - \omega(n^2), \frac{1}{8}{n\choose 3} + \omega(n^2)\right]\]

with probability $1 - o(1)$. What values of $p = p(n)$ can we say that $X > 0$ with probability $1 - o(1)$? Set $t = {n \choose 3}p^3$, then 

\begin{align*}
    \Pr[X = 0] & \leq \Pr[|X - {n \choose 3}p^3| \geq {n \choose 3}p^3] \\
    & \leq O\left(\frac{n^4p^5 + n^3p^3}{n^6p^6}\right) \\
    & = O\left(\frac{1}{n^2p} + \frac{1}{n^3p^3}\right)
\end{align*}

If $p = \omega(\frac{1}{n^2})$, then $O(\frac{1}{n^2p + n^3p^3}) = o(1)$. If $p = \omega(\frac{1}{n})$, then $\Pr[X = 0] = o(1)$, while if $p = o(\frac{1}{n})$, then $\Pr[X = 0] = 1 - o(1)$. 
