\chapter{Manifolds and Smooth Maps}

\section{Definitions}

Differential Topology is a field of math that specializes in the study of what are called ``smooth manifolds.'' Before we get anywhere, we first need to define what these are.

The cool thing about calculus is that it is built upon the geometry of Euclidean space. What this means is that if something ``looks like'' Euclidean space, the same techniques will apply. Such spaces are called \textbf{manifolds}, spaces where if you zoom in far enough, you will get something that looks a lot like Euclidean space. If, upon zooming in, we find that the space looks like Euclidean space in $k$ dimensions, then we say that the space is a \textbf{$k$-dimensional manifold}, or just a \textbf{$k$-manifold}. Some notable examples of 2-manifolds include the sphere and torus. 

Let's make this idea more precise. First, we need to define the idea of smoothness.

\begin{definition}
    A mapping $f: U \to \R^m$, where $U \subset \R^n$ is open, is called \textbf{smooth} if it has continuous partial derivatives of all orders. 
\end{definition}

This is great when our domains are open subsets, but this isn't always true. To recitfy this, we need to use a local definition:

\begin{definition}
    A mapping $f: X \to \R^m$, $X \subset \R^n$ is an arbitary subset, is called \textbf{smooth} if, for each $x \in X$, there is an open set $U \subset \R^n$ and a smooth map $F: U \to \R^m$ such that $F\big|_{U \cap X} = f$. In other words, $f$ can be \textit{locally extended} to a smooth map on open sets.
\end{definition}

In other words, we define smoothness to be a local property: a map is smooth if it is smooth in a neighbourhood of every point in the domain. This is in contrast to it being a global property, where we consider $X$ as a single, unified object. 

With smooth now defined, we move to defining when two spaces looks ``the same.''

\begin{definition}
    A smooth map $f: X \to Y$ of subsets of Euclidean space is a \textbf{diffeomorphism} if it is a bijection with a smooth inverse $f^{-1}: Y \to X$. We call $X,Y$ \textbf{diffeomorphic} if there is a diffeomorphism between them. 
\end{definition}

For our purposes, two diffeomorphic spaces sets are basically the same set, just situated differented in the ambient space. Below are some examples of diffeomorphic and non-diffeomorphic spaces. 

Finally, we are able to define what a manifold actually is. 

\begin{definition}
    Let $X \subset \R^n$. We call $X$ a \textbf{$k$-dimensional manifold} if it is locally diffeomoprhic to $\R^k$. That is, for each $x \in X$, there is a neighbourhood $V$ of $x$ that is diffeomorphic to an open set $U \subseteq \R^k$. We call the corresponding diffeomoprhism $\phi: U \to V$ a \textbf{parameterization} of $V$, while the inverse $\phi^{-1}: V \to U$ is called a \textbf{coordinate system} on $V$.  
\end{definition}

The map $\phi^{-1}$ can be written as $k$ smooth functions in a tuple $(x_1, \ldots, x_k)$, called \textbf{coordinate functions}. Sometimes these are called \textbf{local coordinates} on $V$, and one writes a point of $V$ as this tuple. There is some subtly here, as these functions identify $V$ with $U$ implicitely. A point $v \in V$ is identified with 

\[(x_1(v), \ldots, x_k(v))\]

We will also write the dimension of $X$, the dimension of the space it is locally diffeomorphic to, as just $\dim X$. 

\begin{example}
    Consider the circle $S^1 = \{(x,y) \in \R^2 : x^2 + y^2 = 1\}$. We claim this is a 1-dimensional manifold. 

    We will parameterize it with 4 maps. First suppose $(x,y)$ is such that $y > 0$, then the map 

    \[\phi_1(x) = (x,\sqrt{1 - x^2})\]

    maps $I = [-1,1]$ bijectively to the upper semicircle. The inverse map, which just projects $(x,y)$ to the $x$-coordinate, is also smooth. Thus, $\phi_1$ is a parameterization. A similar parameterization, $\phi_2$, for the lower semicircle where $y < 0$, is defined as $\phi_2(x) = (x,-\sqrt{1-x^2})$. These maps give local parameterizations for all points of $S^1$ except $(1,0)$ and $(-1,0)$. To get those, we use the maps $\phi_3(y) = (\sqrt{1-y^2},y)$ and $\phi_4 =(-\sqrt{1-y^2},y)$ to send $I$ to the left and right semi-circles. 

    We have thus shown the circle is a 1-dimensional manifold by parameterizing it with 4 maps to between it and a 1-dimensional interval. Note that this can also be done with 2 maps using stereographic projection. In general, $S^n$ is an $n$-dimensional manifold, which can be proven in a similar way. 
\end{example}

\begin{example}
    Given manifolds $X,Y$ in $\R^N$ and $R^M$ respectively, one can form a new manifold by taking their cartesian product $X \times Y \subset \R^{M+N}$. 

    Suppose $X,Y$ have dimension $k,l$ repsectively. For each $x \in X$ we have an open set $W \subset \R^k$ and a local parameterization $\phi: W \to X$ at $x$. Similarly, for each $y \in Y$, we have an open set $U \subset \R^l$ and a local parameterization $\psi: U \to Y$ at $y$. Then we can define a map $\phi \times \psi: W \times U \to X \times Y$ by 

    \[(\phi \times \psi)(w,u) = (\phi(w),\psi(u))\]

    $W \times U$ is open in $\R^{k+l}$, and more importantly, $\phi \times \psi$ is a local parameterization of $X \times Y$ at $(x,y)$. Thus, $X \times Y$ is a manifold of dimension $k + l$. 
\end{example}

One final thing to define in this section is the idea that manifolds can sit inside larger manifolds. 

\begin{definition}
    If $X,Z$ are manifolds in $\R^N$ and $Z \subset X$, we call $Z$ a \textbf{submanifold} of $X$. 
\end{definition}

As a trivial example, any manifold $X$ is a submanifold of its ambient space $\R^N$. Similarly, every open set of $X$ is a submanifold of $X$. As a non-trivial example, $S^m$ is a submanifold of $S^n$ whenever $m < n$.  


\section{Derivatives and Tangent Spaces}

\section{The Inverse and Implicit Function Theorems}

\section{Immersions and Submersions}