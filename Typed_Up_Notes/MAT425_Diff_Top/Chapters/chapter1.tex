\chapter{Manifolds and Smooth Maps}

\section{Definitions}

Differential Topology is a field of math that specializes in the study of what are called ``smooth manifolds.'' Before we get anywhere, we first need to define what these are.

The cool thing about calculus is that it is built upon the geometry of Euclidean space. What this means is that if something ``looks like'' Euclidean space, the same techniques will apply. Such spaces are called \textbf{manifolds}, spaces where if you zoom in far enough, you will get something that looks a lot like Euclidean space. If, upon zooming in, we find that the space looks like Euclidean space in $k$ dimensions, then we say that the space is a \textbf{$k$-dimensional manifold}, or just a \textbf{$k$-manifold}. Some notable examples of 2-manifolds include the sphere and torus. 

\begin{center}
    \includegraphics[width=0.5\textwidth]{Chapters/Manifold Examples.png}
\end{center}

Let's make this idea more precise. First, we need to define the idea of smoothness.

\begin{definition}
    A mapping $f: U \to \R^m$, where $U \subset \R^n$ is open, is called \textbf{smooth} if it has continuous partial derivatives of all orders. 
\end{definition}

This is great when our domains are open subsets, but this isn't always true. To recitfy this, we need to use a local definition:

\begin{definition}
    A mapping $f: X \to \R^m$, $X \subset \R^n$ is an arbitary subset, is called \textbf{smooth} if, for each $x \in X$, there is an open set $U \subset \R^n$ and a smooth map $F: U \to \R^m$ such that $F\big|_{U \cap X} = f$. In other words, $f$ can be \textit{locally extended} to a smooth map on open sets.
\end{definition}

In other words, we define smoothness to be a local property: a map is smooth if it is smooth in a neighbourhood of every point in the domain. This is in contrast to it being a global property, where we consider $X$ as a single, unified object. 

\begin{center}
    \includegraphics[width=0.5\textwidth]{Chapters/Smooth Picture.png}
\end{center}

With smooth now defined, we move to defining when two spaces looks ``the same.''

\begin{definition}
    A smooth map $f: X \to Y$ of subsets of Euclidean space is a \textbf{diffeomorphism} if it is a bijection with a smooth inverse $f^{-1}: Y \to X$. We call $X,Y$ \textbf{diffeomorphic} if there is a diffeomorphism between them. 
\end{definition}

For our purposes, two diffeomorphic spaces sets are basically the same set, just situated differently in the ambient space. Below are some examples of diffeomorphic and non-diffeomorphic spaces. 

\begin{center}
    \includegraphics[width=0.5\textwidth]{Chapters/Diffeomorphisms.png}
\end{center}

Finally, we are able to define what a manifold actually is. 

\begin{definition}
    Let $X \subset \R^n$. We call $X$ a \textbf{$k$-dimensional manifold} if it is locally diffeomoprhic to $\R^k$. That is, for each $x \in X$, there is a neighbourhood $V$ of $x$ that is diffeomorphic to an open set $U \subseteq \R^k$. We call the corresponding diffeomorphism $\phi: U \to V$ a \textbf{parameterization} of $V$, while the inverse $\phi^{-1}: V \to U$ is called a \textbf{coordinate system} on $V$.  
\end{definition}

The map $\phi^{-1}$ can be written as $k$ smooth functions in a tuple $(x_1, \ldots, x_k)$, called \textbf{coordinate functions}. Sometimes these are called \textbf{local coordinates} on $V$, and one writes a point of $V$ as this tuple. There is some subtly here, as these functions identify $V$ with $U$ implicitely. A point $v \in V$ is identified with 

\[(x_1(v), \ldots, x_k(v))\]

We will also write the dimension of $X$, the dimension of the space it is locally diffeomorphic to, as just $\dim X$. 

\begin{example}
    Consider the circle $S^1 = \{(x,y) \in \R^2 : x^2 + y^2 = 1\}$. We claim this is a 1-dimensional manifold. 

    We will parameterize it with 4 maps. First suppose $(x,y)$ is such that $y > 0$, then the map 

    \[\phi_1(x) = (x,\sqrt{1 - x^2})\]

    maps $I = [-1,1]$ bijectively to the upper semicircle. The inverse map, which just projects $(x,y)$ to the $x$-coordinate, is also smooth. Thus, $\phi_1$ is a parameterization. A similar parameterization, $\phi_2$, for the lower semicircle where $y < 0$, is defined as $\phi_2(x) = (x,-\sqrt{1-x^2})$. These maps give local parameterizations for all points of $S^1$ except $(1,0)$ and $(-1,0)$. To get those, we use the maps $\phi_3(y) = (\sqrt{1-y^2},y)$ and $\phi_4 =(-\sqrt{1-y^2},y)$ to send $I$ to the left and right semi-circles. 

    We have thus shown the circle is a 1-dimensional manifold by parameterizing it with 4 maps to between it and a 1-dimensional interval. Note that this can also be done with 2 maps using stereographic projection. In general, $S^n$ is an $n$-dimensional manifold, which can be proven in a similar way. 
\end{example}

\begin{example}
    Given manifolds $X,Y$ in $\R^N$ and $R^M$ respectively, one can form a new manifold by taking their cartesian product $X \times Y \subset \R^{M+N}$. 

    Suppose $X,Y$ have dimension $k,l$ repsectively. For each $x \in X$ we have an open set $W \subset \R^k$ and a local parameterization $\phi: W \to X$ at $x$. Similarly, for each $y \in Y$, we have an open set $U \subset \R^l$ and a local parameterization $\psi: U \to Y$ at $y$. Then we can define a map $\phi \times \psi: W \times U \to X \times Y$ by 

    \[(\phi \times \psi)(w,u) = (\phi(w),\psi(u))\]

    $W \times U$ is open in $\R^{k+l}$, and more importantly, $\phi \times \psi$ is a local parameterization of $X \times Y$ at $(x,y)$. Thus, $X \times Y$ is a manifold of dimension $k + l$. 
\end{example}

One final thing to define in this section is the idea that manifolds can sit inside larger manifolds. 

\begin{definition}
    If $X,Z$ are manifolds in $\R^N$ and $Z \subset X$, we call $Z$ a \textbf{submanifold} of $X$. 
\end{definition}

As a trivial example, any manifold $X$ is a submanifold of its ambient space $\R^N$. Similarly, every open set of $X$ is a submanifold of $X$. As a non-trivial example, $S^m$ is a submanifold of $S^n$ whenever $m < n$.  


\section{Derivatives and Tangent Spaces}

We move to talking about derivatives. Recall from multivariate calculus that for a smooth map $f: \R^n \to \R^m$ and point $x$ in the domaain, the \textbf{directional derivative} in the direction of $h \in \R^n$ is 

\[df_x(h) = \lim_{t\to 0} \frac{f(x+th) - f(x)}{t}\]

If we fix $x$, we get a linear map $df_x: \R^n \to \R^m$ mapping vectors $h$ to their corresponding directional derivative. We call this map the \textbf{derivative} of $f$ at $x$. This map is oftentimes represented by a matrix of partial derivatives called the Jacobian matrix of $f$ at $x$:

\[
\begin{pmatrix}
    \pardif{f_1}{x_1}(x) & \cdots  & \pardif{f_1}{x_n}(x) \\
    \vdots              &        & \vdots              \\
    \pardif{f_n}{x_1}   & \cdots & \pardif{f_n}{x_n}
\end{pmatrix}
\]

For our purposes, we will think of the derivative as a linear map. 

This is a very natural definition, and we can see this in the chain rule. Let $U \subset \R^n, V \subset \R^m$ be open, with smooth maps $f: U \to V, g: V \to \R^l$. Then by the chain rule, for $x in U$,

\[d(g \circ f)_x = dg_{f(x)} \circ df_x\]

This gives us a commutative diagram of derivative maps:

\[\begin{tikzcd}
	{\R^n} && {\R^m} && {\R^l}
	\arrow["{df_x}", from=1-1, to=1-3]
	\arrow["{d(g \circ f)_x}"', curve={height=18pt}, from=1-1, to=1-5]
	\arrow["{dg_{f(x)}}", from=1-3, to=1-5]
\end{tikzcd}\]

Note that, as $df_x$ gives a linar approximation to $f$ at $x$, if $f: U \to \R^m$ is a linear map, $L$, then $df_x = L$ for all $x$. In particular, the inclusion map $\iota: U \to \R^n$ at any $x$ is just the identity map of $\R^n$. 

Since the derivative is the best linear approximation of a mapping, we can use them to find the linear space that best approximates a manifold $X$ at a point $x$. Let $X$ be a manifold in $\R^N$ and $\phi: U \to X$ a local parameterization at $x$, with $U$ open in $\R^k$. Assume that $\phi(0) = x$. Then the best linear approximation to $\phi$ at 0 is 

\[u \mapsto \phi(0) + d\phi_0(u) = x + d\phi_0(u)\]

The image of this map is a space centered at $x$. If we ignore the translation to $x$, we can define what this space is.

\begin{definition}
    The \textbf{tangent space} of $X$ at $x$ is the image of $d\phi_0: \R^k \to \R^N$, denoted $TX_x$. It is a vector subspace of $X$ whose translation to $x$, $x + TX_x$, is the closest flat approximation to $X$ at $x$. 

    A point $v \in TX_x$ is called a \textbf{tangent vector to $X$ at $x$}, though it is more natural to view $v$ as a vector from $x$ to $x + v$. 
\end{definition}

\begin{center}
    \includegraphics[width=0.5\textwidth]{Chapters/Tangent Space.png}
\end{center}

Because this definition is based on the local parameterization, one should ask if the tangent space depends on the choice of local parameterization. Suppose that $\psi: V \to X$ is another choice with $\psi(0) = x$. $U \cap V \neq \varnothing$, so without loss of generality we can shrink both sets so that $\psi(U) = \psi(V)$. The map

\[h = \psi^{-1} \circ \phi: U \to V\]

is a diffeomorphism (composition of diffeomorphisms), and $\phi = \psi \circ h$. Differentiating gives us 

\[d\phi_0 = d\psi_0 \circ dh_0\]

Thus, the image of $d\phi_0$ is in $d\psi_0$. Reversing the roles of $\phi,\psi$ gives the other direction, meaning the images are the same, and $TX_x$ is well-defined. 

The dimension of the tangent space is the same as the dimension of the manifold. Suppose $\dim X = k$, and let $W \subset \R^N$ be open. Let $\Phi': \R^N \to \R^k$ be smooth and extend $\phi^{-1}$ to all of $\R^N$. $\Phi' \circ \phi$ is the identity map on $U$. Smoothness of $\phi^{-1}$, and the chain rule, then imply that 

\[\operatorname{id} = d\Phi_x \circ d\phi_0\]

on $\R^k$. Hence, $d\phi_0: \R^k \to TX_x$ is an isomorphism, so the dimensions are the same. 

We are now ready to define the derivative of a map on arbitary manifolds $f: X \to Y$. Assuming $f(x) = y$, it should be a map of tangent space $df_x: TX_x \to TY_y$. It must satisfy two things. First, if the map is in Euclidean space, we should just get the usual derivative. Second, the chain rule must be satisfied. 

With all the usual terms and definitions, and assuming that $\phi(0) = x, \psi(0) = y$, we have a commutative diagram:

\[\begin{tikzcd}
	X && Y \\
	U & {} & V
	\arrow["f", from=1-1, to=1-3]
	\arrow["\phi", from=2-1, to=1-1]
	\arrow["{h=\psi^{-1} \circ f \circ \phi}"', from=2-1, to=2-3]
	\arrow["\psi"', from=2-3, to=1-3]
\end{tikzcd}\]

We already know what $d\phi_0,d\psi_0,dh_0$ must be, and the chain rule tells us that taking derivatives turns our above diagram into a diagram of linear maps:

\[\begin{tikzcd}
	{TX_x} && {TY_y} \\
	{\R^k} && {\R^l}
	\arrow["{df_x}", from=1-1, to=1-3]
	\arrow["{d\phi_0}", from=2-1, to=1-1]
	\arrow["{dh_0}"', from=2-1, to=2-3]
	\arrow["{d\psi_0}"', from=2-3, to=1-3]
\end{tikzcd}\]

Recalling that $d\phi_0$ is an isomorphism, we get exactly one acceptable definition of $df_x$:

\begin{definition}
    Let $f: X \to Y$ be a map of manifolds, with $f(x) = y$. Let $\phi,\psi$ be local parameterizations of $x,y$ repsectively, on open sets $U \subset \R^k, V \subset \R^l$ repsectively. Then the \textbf{derivative of $f$ at $x$} is the linear map

    \[df_x: TX_x \to TY_y \quad df_x(v) = (d\psi_0 \circ dh_0 \circ d\phi_0^{-1})(v)\]
\end{definition}

Again, we need to check that this definition does not depend on the choice of parameterization. Suppose that $\phi',\psi'$ parameterize $x,y$ as well, and without loss of generality assume the images of $\phi,\phi'$ and $\psi,\psi'$ are equal. We get new diffeomorphisms

\[\Phi = \phi'^{-1} \circ \phi \quad \Psi = \psi'^{-1} \circ \psi\]

with $\Phi(0) = \Psi(0) = 0$. Setting $h' = \psi'^{-1} \circ f \circ \phi'$, we get that 

\[ h = \Psi^{-1} \circ h' \circ \Phi \]

so at 0, we get that $dh_0 = d\Psi^{-1}_0 \circ dh'_0 \circ d\Phi_0$. Thus, 

\begin{align*}
    df_x & = d\psi_0 \circ dh_0 \circ d\phi_0^{-1} \\
         & = d\psi_0 \circ d\Psi^{-1}_0 \circ dh'_0 \circ d\Phi_0 \circ d\phi_0^{-1}\\
         & = d\psi'_0 \circ dh'_0 \circ d\phi'^{-1}_0
\end{align*}

as required. 

We also need to check that our definition satisfies the chain rule. 

\begin{theorem}[Chain Rule]
    If $f: X \to Y, g: Y \to Z$ are smooth maps of manifolds, then 

    \[d(g\circ f)_x = dg_{f(x)} \circ df_x\]
\end{theorem}

\begin{proof}
    Let $\eta: W \to Z$ at $z = g(y)$, with $W$ open in $\R^m$ and $\eta(0) = z$, be a parameterization of $Z$ at $z$. We get a commutative diagram 

    \[\begin{tikzcd}
        X && Y && Z \\
        \\
        U && V && W
        \arrow["f", from=1-1, to=1-3]
        \arrow["g", from=1-3, to=1-5]
        \arrow["\phi", from=3-1, to=1-1]
        \arrow["{h = \psi^{-1} \circ f \circ \phi}"', from=3-1, to=3-3]
        \arrow["\psi", from=3-3, to=1-3]
        \arrow["{j = \eta^{-1} \circ g \circ \psi}"', from=3-3, to=3-5]
        \arrow["\eta"', from=3-5, to=1-5]
    \end{tikzcd}\]

    and which reduces to 

    \[\begin{tikzcd}
        X && Z \\
        \\
        U && W
        \arrow["{g \circ f}", from=1-1, to=1-3]
        \arrow["\phi", from=3-1, to=1-1]
        \arrow["{j \circ h}"', from=3-1, to=3-3]
        \arrow["\eta", from=3-3, to=1-3]
    \end{tikzcd}\]

    Thus, by definition,

    \[d(g \circ f)_x = d\eta_0 \circ d(j \circ h)_0 \circ d\phi_0^{-1}\]

    The standard chain rule applies for $j \circ h$, so $d(j \circ h)_0 = (dj)_0 \circ (dh)_0$. Hence,

    \[d(g \circ f)_x = (d\eta_0 \circ dj_0 \circ d\psi^{-1}_0) \circ (d\psi_0 \circ dh_0 \circ d\phi_0^{-1}) = dg_y \circ df_x\]

    as required. 
\end{proof}

\section{The Inverse and Implicit Function Theorems}

Before we start talking about the topology of manifolds, we need to study the local behavior of smooth maps. 

\subsection{The Inverse Function Theorem}

Suppose $X,Y$ are manifolds of the same dimension, and $f: X \to Y$ is smooth. The simplest behavior this map can have at a point $x$, is that it sends neighbourhoods of $x$ to neighbourhoods of $f(x)$ via a diffeomorphism. We call such maps \textbf{local diffeomorphisms}. How can we check if a map is a local diffeomorphism? A necessary condition of such maps is that the derivative $df_x$ is an isomorphism of tangent spaces. Suprisingly, this is also a sufficient condition:

\begin{theorem}[The Inverse Function Theorem]
    Let $f: X \to Y$ be smooth such that $df_x$ at $x$ is an isomorphism. Then $f$ is a local diffeomorphism at $x$. 
\end{theorem}

This is a powerful theorem, the map $df_x$ can be thought of as a matrix of numbers, and this matrix is invertible (hence yielding an isomorphism), precisely when it has non-zero determinant. Thus, checking for a local property of smooth maps amounts to just showing that a matrix has a non-zero determinant! 

Proofs of the Inverse Function Theorem when $X,Y$ are open subsets of Euclidean space are abundant. Generalizing to manifolds amounts to using local parameterization. 

\begin{proof}[Proof of Inverse Function Theorem]
    Take parameterizations $\phi: U \to X, \psi: V \to Y$ around $x$ and $f(x)$ and set $h = \psi \circ f \circ \phi^{-1}$. By assumption, $df_x = d\psi_0 \circ dh_0 \circ d\phi^{-1}_0$ is an isomorphism, so $dh_0$ must be an isomorphism. The Inverse Function Theorem on Euclidean space applies, meaning there is an open neighbourhood $W \subset \phi(U)$ of $\phi^{-1}(x)$ such that $h(W)$ is an open neighbourhood of $\psi^{-1}(f(x))$, and $h$ restricts to a diffeomorphism $W \to h(W)$. It then follows that $f = \psi^{-1} \circ h \circ \phi$ is a diffeomorphism from $\phi^{-1}(W)$, an open neighbourhood of $x$, to $\psi^{-1}(h(W))$, an open neighbourhood of $f(x)$.
\end{proof}

Another way of viewing the inverse function theorem is to say that, when $df_x$ is an isomorphism, one can choose local coordinates $x_1, \ldots, x_k$ around $x$ and $y$ such that $f$ looks like the identity near $x$. In general, we say two maps $f: X \to Y$ and $f': X' \to Y'$ are \textbf{equivalent} if there are diffeomorphisms $\phi: X' \to X, \psi: Y' \to Y$ such that the resulting diagram commutes. From this angle, we can say that if $df_x$ is an isomorphism, when $f$ is locally equivalent, at $x$, to the identity. 

\subsection{The Implicit Function Theorem}

Unfortunately, our maps will sometimes go between manifolds with different dimensions. For example, suppose $f: X \to Y$ goes between manifolds of dimension $k,l$ respectively, such that $l < k$. Can we say anything about the local behavior of $f$? Yes! In fact, we get another farily important theorem.

\begin{theorem}[The Implicit Function Theorem]
    Let $f: X \to Y$ be a map between manifolds of dimension $n + m$ and $n$ repsectively, and let $x \in X$. Suppose that the matrix of partial derivatives corresponding to the first $n$ variables at $x$ is invertible. Then there exist open sets $U_1 \subset \R^n, U_2 \subset \R^m$ with $(x_1, \ldots, x_n) \in U_1$ and a map $\phi: \R^n \to \R^m$ such that for every $x' \in U_1$, there is a unique point $\phi(x') \in U_2$ such that 

    \[f(x',\phi(x')) = f(x)\]
\end{theorem}

In other words, near $x$, $f$ acts like projection onto the first $n$ coordinates. Like before, the proof essentially boils down to the proof in Euclidean space, which can easily be found. The proof for manifolds then amounts to using local parameterizations.

Something else that should be noted is that the implicit and inverse function theorems are equivalent, meaning you can use one to prove the other. 

\section{Immersions and Submersions}

\subsection{Immersions}

What is the best local behavior $f: X \to Y$ can have when $\dim X < \dim Y$? The best we can expect is for $df_x$ to be an injective map. 

\begin{definition}
    If $f: X \to Y$ is such that $\dim X < \dim Y$ and $df_x$ is injective, then we say $f$ is an \textbf{immersion} at $x$. 

    The \textbf{canonical immersion} is the inclusion map $(a_1, \ldots, a_k) \to (a_1, \ldots, a_k, 0, \ldots, 0)$ from $\R^k \to \R^{l}$ for $k \leq l$. 
\end{definition}

The canonical immersion is special. Up to diffeomorphism, all immersions are locally the canonical one.

\begin{theorem}[Immersion Theorem]
    Let $f: X \to Y$ be an immersion at $x$ and $y = f(x)$. Then there are local coordinates around $x$ and $y$ such that 

    \[f(x_1, \ldots, x_k) = (x_1, \ldots, x_k, 0, \ldots, 0)\]
\end{theorem}

\begin{proof}
    Choose local parameterizations $\phi: U \to X, \psi: V \to Y$ for $x,y$, with $\phi(0) = x, \psi(0) = y$. We need to augment the map $h = \psi^{-1} \circ f \circ \phi$ to apply the Inverse Function Theorem. 

    $dg_0$ is injective, so we may apply a change of basis to $\R^l$ to make it a matrix of the form

    \[\begin{pmatrix}
        I_k \\
        \hline 0
    \end{pmatrix}\]

    We can now define a map $G: U \times \R^{l-k} \to \R^l$ by 

    \[G(x,z) = h(x) + (0,z)\]

    so apply $h$ to the first $k$ coords, and leave the rest unchanged. $G$ maps an open set of $\R^l$ to $\R^l$, and $dG_0$ is just the identity matrix, so the Inverse Function Theorem applies. Thus, $G$ is a local diffeomorphism of $\R^l$ at 0. We also get that $\psi \circ G$ is a local diffeomorphism at 0, and so we may use it as a local parameterization of $Y$ at $y$. By construction, $h = G \circ (\text{canonical immersion})$, so if we make $U,V$ sufficiently small, we get the following commutative diagram:

    \[\begin{tikzcd}
        X && Y \\
        \\
        U && V
        \arrow["f", from=1-1, to=1-3]
        \arrow["\phi", from=3-1, to=1-1]
        \arrow["{\text{canonical immersion}}"', from=3-1, to=3-3]
        \arrow["{\psi \circ G}"', from=3-3, to=1-3]
    \end{tikzcd}\]

    and the claim is proven. 
\end{proof}

\begin{corollary}
    If $f$ is an immersion at $x$, it is an immersion in a neighbourhood of $x$. 
\end{corollary}

\subsection{Submersions}

Now suppose that $\dim X \geq \dim Y$. Now the best we can say about $df_x$ is that it is surjective. 

\begin{definition}
    If $f: X \to Y$ is such that $df_x$ is surjective, we say $f$ is a \textbf{submersion} at $x$. 

    The \textbf{canonical submersion} is just projection $(x_1, \ldots, x_k) \to (x_1, \ldots, x_l)$ where $k \geq l$. 
\end{definition}

Like immersions, every submersion is locally equivalent to the canonical submersion near $x$. 

\begin{theorem}[Submersion Theorem]
    Suppose $f: X \to Y$ is a submersion at $x$ and $y = f(x)$. Then there are local coordinates around $x$ and $y$ such that $f(x_1,\ldots, x_k) = (x_1,\ldots,x_l)$. 
\end{theorem}

\begin{proof}
    The proof is similar to that of the Immersion Theorem. We now define $G: U \to \R^k$ by 

    \[G(a) = (h(a), a_{l+1}, \ldots, a_k)\]

    Inverse Function Theorem then applies and we get a local diffeomorphism $G^{-1}$ of a neighbourhood $U'$ of 0 to $U$. $\phi \circ G^{-1}$ is also a local diffeomorphism, and $h = (\text{canonical isomorphism}) \circ G$, so $h \circ G^{-1}$ is the canonical submersion. We get a similar commutative diagram and the claim follows. 
\end{proof}

\begin{corollary}
    If $f$ is a submerison of $x$, it is a submersion in a neighbourhood of $x$. 
\end{corollary}

\subsection{The Preimage Theorem}

The most valuable insights of these classification theorems is the geometric nature of solutions. Let $f: X \to Y$ and $y \in Y$. Then we call the set $f^{-1}(y)$ the \textbf{preimage} of $y$. This isn't always a nice object, but suppose $f$ is a submersion at $x \in f^{-1}(y)$. By the Submersion Theorem we can choose local coordinates around $x$ and $y$ such that 

\[f(x_1, \ldots, x_k) = (x_1, \ldots, x_l)\]

and $y$ corresponds to $(0,\ldots, 0)$. Thus, near $x$, $f^{-1}(y)$ is just the set of points 

\[\{(0,\ldots, 0, x_{l+1},\ldots, x_k)\}\]

In more formal terms, let $V$ be the neighbourhood of $x$ on which these coordinates are defined. Then $f^{-1}(y) \cap V$ is the set of points for which the first $l$ coordinates of $x$ are 0. The functions $x_{k+1}, \ldots, x_l$ form a coordinate system on $f^{-1}(y) \cap V$, a relatively open subset of $f^{-1}(y)$. 

\begin{center}
    \includegraphics[width=0.5\textwidth]{Chapters/Preimage Theorem.png}
\end{center}

We are thus led to a new definition:

\begin{definition}
    For a map $f: X \to Y$, a point $y \in Y$ is a \textbf{regular value} for $f$ if $df_x$ is surjective at all $x \in f^{-1}(y)$. It is called a \textbf{critical value} if it is not regular. The point $x$ for which $df_x$ is subjective is called a \textbf{regular point}, and called a \textbf{critical point} if it is not regular. 
\end{definition}

Our above argument proves another important theorem:

\begin{theorem}[Preimage Theorem]
    If $y$ is a regular value of $f: X \to Y$, then $f^{-1}(y)$ is a submanifold of $X$ with dimension $\dim X - \dim Y$. 
\end{theorem}