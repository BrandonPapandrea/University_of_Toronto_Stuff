\chapter{Week 4}

\section{The Heat Equation, Continued}

\subsection{The Heat Equation on an Infinite Rod}

We will prove some formulas related to the heat equation in the special case that our rod has infinite length. We thus consider the case where $x \in \R$. We will also not consider 0 in the domain of $t$; the reason why will be obvious.

\begin{remark}
    The derivation we present here will seem somewhat random, as the tool used for understanding it, called the Fourier Transform, will not be disucssed until Weeks 11 \& 12. The textbook presents another similar derivation that may yield better insights for some as to how we derive the following.
\end{remark}

\subsubsection{The Heat Kernel}

We first define a function $S(x,t): \R \times (0,\infty) \to \R$ by

\[ S(x,t) = \frac{1}{\sqrt{4\pi kt}}e^{\frac{-x^2}{4kt}} \]

Note that $S(x,t) > 0$ for all $x,t$ in the domain. 

\begin{prop}
    $S$ solves the heat equation.
\end{prop}

\begin{proof}
    We have

    \begin{align*}
        \partial_tS & = \frac{1}{\sqrt{4\pi k}} \cdot \frac{-1}{2t^{3/2}}e^{\frac{-x}{4kt}} + \frac{1}{\sqrt{4\pi kt}} \frac{x^2}{4kt^2}e^{\frac{-x}{4kt}} \\
                    & = \frac{-1}{2}t^{-1}S + \frac{x^2}{4kt^2}S \\
        \partial_xS & = \frac{1}{\sqrt{4\pi kt}} \cdot \frac{-x}{2kt}e^{\frac{-x^2}{4kt}} \\
                    & = \frac{-x}{2kt}S \\
        \partial_{xx}S & = \frac{-1}{2kt}S - \frac{x}{2kt}\partial_xS \\
                       & = \frac{-1}{2kt}S + \frac{x^2}{4k^2t^2}S
    \end{align*}

    Thus,

    \begin{align*}
        \partial_tS(x,t) - k\partial_{xx}S(x,t) & = S(x,t)\left(\frac{-1}{2t} + \frac{x^2}{4kt^2} + \frac{1}{2t} - \frac{x^2}{4kt^2}\right) \\
        & = 0
    \end{align*}

    as desired.
\end{proof}

Before we proceed, recall that the integral of the Gaussian function $e^{-x^2}$ is given by

\[ \int_{-\infty}^\infty e^{-x^2} \ dx = \sqrt{\pi} \]

Now for each $t > 0$, we may consider 

\[ \int_{-\infty}^\infty S(x,t) \ dx \]

Informally, for a fixed $t$, $S$ will go to 0 exponentially fast as $x \to \pm\infty$. Thus, this integral is well-defined. We write it as

\[ \frac{1}{\sqrt{4\pi kt}}\int_{-\infty}^\infty e^{-\left(\frac{x}{2\sqrt{kt}}\right)^2} \ dx \]

We now make a change of variables with

\[ y = \frac{x}{2\sqrt{kt}} \ , \ dy = \frac{1}{\sqrt{kt}} \ dx \]

Under this change, we get

\[ \frac{1}{\sqrt{\pi}}\int_{-\infty}^\infty e^{-y} \ dy = \frac{1}{\sqrt{\pi}}\sqrt{\pi} = 1 \]

We sometimes call $S$ the \textbf{heat kernel}. Let's try and understand what happens to the heat kernel as $t \to 0$. We first make some observations:

\begin{enumerate}
    \item When $x = 0$, 

    \[ S(x,t) = \frac{1}{\sqrt{4\pi kt}} \xrightarrow[]{t \to 0} \infty \]

    \item Suppose $|x| > \delta > 0$, where $\delta$ is small. In this case,

    \[ |S(x,t)| \leq \frac{1}{\sqrt{4\pi kt}}e^{\frac{-\delta^2}{4kt}} \]

    as $t \to 0$, the exponential is moving to 0 faster than the $\sqrt{4\pi kt}$ can blow $S$ up to infinity, and so we are going to 0. 
\end{enumerate}

These observations help us understand what is going on: we start with a Gaussian-like curve with integral value 1. As $t$ gets smaller and smaller, this curve gets thinner and thinner; $S$ is positive in a very small area around $x = 0$, while at $x = 0$, $S$ gets larger to preserve the value of the integral. Finally, at $t = 0$, we have a function that is 0 everywhere except at $x = 0$, where it has a value of infinity. This is shown below, where we show the graphs of $S$ for $t = 1, 0.5, 0.1$, and $t = 0.001$.

\begin{center}
    \includegraphics[width=0.7\textwidth]{Week 4/Heat Kernel as t goes to 0.png}
\end{center}

Informally, when $t = 0$:

\[ S(x,0) = \begin{cases}
    \infty & x = 0 \\ 0 & x \neq 0
\end{cases} \]

Because the integral of $S$ over $\R$ is always 1, and because $S$ behaves in the above manner as $t \to 0$, we say $S$ is an \textbf{approximation to the identity}. 

\subsubsection{The Convolution}

$S$ is special because it is related to an important operation in PDE theory. For a continuous function $\phi(x)$ and $t > 0$, we can define

\[ u(x,t) = \int_{-\infty}^\infty S(x-y,t)\phi(y) \ dy \]

called the \textbf{convolution} of $S$ and $\phi$.

\begin{prop}
    If $t > 0$, then $u$ solves the heat equation.
\end{prop}

\begin{proof}
    The result mainly follows from $S$ solving the heat equation. We have by differentiation under the integral sign,

    \begin{align*}
        \partial_tu & = \int_{-\infty}^\infty (\partial_tS)(x-y,t)\phi(y) \ dy \\
        \partial_{xx}u & = \int_{-\infty}^\infty (\partial_{xx}S)(x-y,t)\phi(y) \ dy 
    \end{align*}

    Thus,

    \[ \partial_tu - k\partial_{xx} = \int_{-\infty}^\infty (\partial_tS - k\partial_{xx}S)(x-y,t)\phi(y) \ dy = 0 \]

    as desired.
\end{proof}

The relevance of $\phi$ becomes clear once we notice that as we approach 0, the convolution goes to $\phi(x)$. 

\begin{prop}
    $\lim_{t \to 0} u(x,t) = \phi(x)$.
\end{prop}

\begin{proof}
    We have that

    \begin{align*}
        u(x,t) & = \int_{-\infty}^\infty S(x-y,t)\phi(y) \ dy \intertext{Set $z = x-y, \ dz = -dy$. This flips the bounds of integration, producing a double negative. We get} \\
        & = \int_{-\infty}^\infty S(z,t)\phi(x-z) \ dz \intertext{for $0 < t < 1$, we get} \\
        & \approx \int_{z \text{ close to 0}}S(z,t)\phi(x) \ dz \\
        & = \phi(x) \int_{-\infty}^\infty S(z,t) \ dz \\
        & = \phi(x)
    \end{align*}
\end{proof}

\subsection{IVPs on the Infinite Rod}

We consider the IVP

\[ \begin{cases}
    u_t - ku_{xx} = 0 & (x,t) \in \R \times (0,\infty) \\
    u(x,0) = \phi(x)
\end{cases} \]

Our work in the previous subsection shows that a solution to this IVP is 

\[ u(x,t) = \int_{-\infty}^\infty S(x-y,t)\phi(y) \ dy \]

Note that this only deals with the \textit{existence} of a solution, not uniqueness or whether it is stable or not. We will addresss uniqueness shortly.

\begin{example}
    Solve 

    \[ \begin{cases}
        u_t - ku_{xx} = 0 \\ u(x,0) = e^{-x}
    \end{cases} \]

    We know a solution is

    \[ u(x,t) = \frac{1}{\sqrt{4\pi kt}}\int_{-\infty}^\infty e^{-\frac{(x-y)^2}{4kt}}e^{-y} \ dy \]

    We will simplify the term in the integral as follows:

    \begin{align*}
        \exp\left(\frac{-(x-y)^2}{4kt}-y\right) & = \exp\left(\frac{-x^2 + 2xy - y^2}{4kt} - y\right) \\
        & = \exp\left(\frac{-x^2 + 2xy - y^2 - 4kty}{4kt}\right) \\
        & = \exp\left(\frac{-y^2 - (4kt - 2x)y - x^2}{4kt}\right) \\
        & = \exp\left(\frac{-y^2 - 2(2kt - x)y - (2kt - x)^2 + (2kt - x)^2 - x^2}{4kt}\right) \\
        & = \exp\left(\frac{-(y + (2kt - x)^2}{4kt} + \frac{4k^2t^2 - 4ktx}{4kt}\right) \\
        & = \exp\left(\frac{-(y + (2kt - x)^2}{4kt}\right)\exp(kt - x) \\
        & = \exp\left(-\left(\frac{y + (2kt - x)}{2\sqrt{kt}}\right)^2\right)\exp(kt - x) \\
    \end{align*}

    Now setting $z = \dfrac{y + 2kt - x}{2\sqrt{kt}}, \ dz = \dfrac{dy}{\sqrt{4kt}}$, we get

    \begin{align*}
        u(x,t) & = \frac{\exp(kt - x)}{\sqrt{\pi}}\int_{-\infty}^\infty e^{-z^2} \ dz \\
        & = \exp(kt - x)
    \end{align*}
\end{example}

\subsection{Uniqueness on the Infinite Rod}

What does it mean when a solution to the IVP is \textit{unique}? Given $\phi(x)$, suppose that

\[ u_1(x,t) = u_2(x,t) \]

both solve the IVP

\[ \begin{cases}
    u_t - ku_{xx} = 0 & (x,t) \in \R \times (0,\infty) \\
    u(x,0) = \phi(x)
\end{cases} \]

It follows that

\[ (u_1)_t - k(u_1)_{xx} = (u_2)_t - k(u_2)_{xx} = 0 \]
\[ u_1(x,0) = u_2(x,0) = \phi(x) \]

so in general, is $u_1 = u_2$? Surprisingly, no!

Let's rephrase the above question. Define $w = u_1 - u_2$. Now if

\[ w_t - kw_{xx} = 0 \]
\[ w(x,0) = 0 \]

does $w = 0$? Now the answer is yes, but only with extra assumptions added. 

Recall our solution from before:

\[ u(x,t) = \frac{1}{\sqrt{4\pi kt}}\int_{-\infty}^\infty \exp\left(\frac{-(x-y)^2}{4kt}\right)\phi(y) \ dy \]

Let's assume that $\phi(x)$ is both continuous and \textit{compactly supported}, meaning it is non-zero only on some closed interval $[-R, R]$. For a fixed $t$, what happens when we take $x \to \pm\infty$? Well, the exponential will go to 0, so the integral will also go to 0. Thus $u \to 0$.

For any $T > 0$, define

\[ A(x) = \max_{t \in [0,T]}u(x,t) \]

By the above logic, we get that $\lim_{x \to \pm\infty}A(x) = 0$. This is our extra assumption, and it acts like a boundary condition at $\pm \infty$. 

\begin{theorem}[Uniqueness of Heat Equation on the Infinite Rod]
    Suppose that $w(x,t): \R \times (0,\infty) \to \R$ solves 

    \[ w_t - kw_{xx} = 0 \]
    \[ w(x,0) = 0 \]

    and for each $T > 0$,

    \[ \lim_{x \to \pm\infty} \max_{t \in [0,T]}w(x,t) = 0 \]

    Then $w(x,t) = 0$ everywhere.
\end{theorem}

\begin{proof}
    We apply the Maximum Principle. Let $T > 0$ and pick $R$ to be very large. By the Maximum Principle, 

    \[ \max_{[-R, R] \times [0,T]}w(x,t) = \max_{t = 0} w(x,t) \text{ or } \max_{x = -R}w(x,t) \text{ or } \max_{x = R} w(x,t) \]

    We know that $w(x,0) = 0$. Furthermore, by our extra assumption, the second and third max values will go to 0 as $R$ goes to infinity. Thus,

    \[ \max_{t \in (0,T)} w(x,t) \leq 0 \]

    We repeat this idea for $-w$, which tells us that 

    \[ \min_{t \in (0,T)}w(x,t) \geq 0 \]

    Combining, we get that

    \[ w(x,t) = 0 \ , \ t \in (0,T) \]

    Taking $T \to \infty$ completes the proof.
\end{proof}

\section{Comparing the Heat and Wave Equations}

There are several differences between the heat and wave equations that are of importance to us. We'll focus on qualatiative properties here, not quantitative ones.

\begin{enumerate}
    \item \textbf{The Speed of Propagation} 

    We know that waves have a finite speed of propagation which is related to the value $c$. As $c \to \infty$, the speed of propagation increases, and in the limit, the light path curve will encompass the entire $(x,t)$ plane. 

    Contrast this with the heat equation, which actually has an \textit{infinite} speed of propagation.

    \begin{theorem}
        Let $\phi(x)$ be continuous with $\phi(x) \geq 0$ and suppose there is an $x_0$ such that $\phi(x_0) > 0$. Set

        \[ u(x,t) = \int_{-\infty}^\infty S(x-y,t)\phi(y) \ dy \]
        \[ S(x,t) = \frac{1}{\sqrt{4\pi kt}}e^{\frac{-x^2}{4kt}} \]

        which solves the heat equation $u_t - ku_{xx} = 0$ with $u(x,0) = \phi(x)$. Then if $t > 0$, $u(x,t) > 0$ for all $x$.
    \end{theorem}

    \begin{proof}
        $S(x,t) > 0$ when $t > 0$, so $S(x-y,t) > 0$ when $t > 0$. Now

        \[ u(x,t) = \int_{-\infty}^\infty S(x-y,t)\phi(y) \ dy \]

        We know there is a point $y_0$ such that $\phi(y_0) > 0$, and $\phi$ is continuous. Thus there is an interval $I$ on which $\phi$ is strictly positive. Then

        \[ \int_{-\infty}^\infty S(x-y,t) \phi(y) \ dy \geq \int_I S(x-y,t)\phi(y) \ dy > 0 \]
    \end{proof}

    what this means is simple: At $t=0$, our rod has nonzero temperature in only a finite interval. But, the instant we let time begin to move in time, every point on the rod has some nonzero temperature. 

    \begin{remark}
        Hyperbolic PDEs have finite speed of propagation, whereas almost all other PDEs have an infinite speed of propagation.
    \end{remark}

    \item \textbf{Singularities}

    How many derivatives do the solutions to the wave and heat equations have?

    The wave equation has \textit{singularity propagation}. Because the solutions just look like two functions moving in opposite directions, any singularites they have will move alongside them, and be present in the final solution. Solutions to the wave equation are of the form

    \[ u(x,t) = f(x+ct) + g(x-ct) \]

    with $f,g$ arbitrary, so

    \[ u(x,0) = f(x) + g(x) \]

    If $g = 0$, then $u(x,0) = f(x)$, meaning $u(x,t) = f(x+ct)$. Thus, if $f$ is, say, $C^2$ but not $C^3$, so must $u$. In other words, the differentiability of $u$ corresponds to the differentiability of $f,g$. 

    On the other hand, the heat equation \textit{immediately} smoothens out the initial data once $t > 0$, removing any singularites. Solutions are of the form

    \[ u(x,t) = \frac{1}{\sqrt{4\pi kt}}\int_{-\infty}^\infty e^{\frac{-(x-y)^2}{4kt}}\phi(y) \ dy \]

    Suppose $\phi$ is continuous. Then for $t > 0$,

    \[ \frac{\partial u}{\partial x}= \frac{1}{\sqrt{4\pi kt}}\int_{-\infty}^\infty \frac{\partial}{\partial x}e^{\frac{-(x-y)^2}{4kt}}\phi(y) \ dy = \frac{1}{\sqrt{4\pi kt}}\int_{-\infty}^\infty \frac{-2(x-y)}{4kt}e^{\frac{-(x-y)^2}{4kt}}\phi(y) \ dy \]

    Notice that even if $y \to \infty$, the exponential will dominate the other term in the integral, so it still converges. This argument is valid for derivatives with respect to $t$, as well as all other higher order derivatives. This tells u that $u$ is a $C^\infty$ function, meaning it is smooth. 

    \item \textbf{Long-Time Behaviour}

    What happens to $u(x,t)$ as $t \to \infty$?

    For the wave equation, $f,g$ will just translate in opposite directions as $t \to \infty$, which is boring so. Instead, consider the problem

    \[ \begin{cases}
        u_t - c^2u_{xx} = 0 \\
        u(x,0) = \phi(x), u_x(x,0) = \psi(x)
    \end{cases} \]

    and assume that $\phi, \psi$ are compactly supported. Then 

    \[ u(x,t) = \frac{1}{2}[\phi(x+ct) + \psi(x-ct)] + \frac{1}{2c}\int_{x-ct}^{x+ct} \psi(s) \ ds \]

    Remember the light path curve figure from Week 3? Well, the integral of $\psi$ corresponds to the bottom of that triangle, and $u(x,0)$ is nonzero is some bounded interval on that side of the triangle, with the rest of the triangle being as before. Notice that if $t$ is large, we are going to be outside of this triangle, and so by causality, $x \pm ct$ will be so large that $\phi, \psi$ are going to vanish. Thus, if $t \gg 1$, we just get

    \[ u(x,t) = \frac{1}{2c}\int_{-\infty}^\infty \psi(s) \ ds \]

    Think of this as fixed displacement. We flick a string, and after a long enough time, the string looks flat, but raised or lowered by the above distance. 

    For the heat equation, we again assume that $\phi$ has compact support. Then we have that

    \[ \lim_{t \to \infty} e^{\frac{-(x-y)^2}{4kt}} = e^0 = 1 \]

    so $\lim_{t \to \infty} u(x,t) = 0$. 

    Another way to see this is to define 

    \[ w(x,t) := \sqrt{t}u(x,t) = \frac{1}{\sqrt{4\pi k}}\int_{-\infty}^\infty e^{-\frac{(x-y)^2}{4kt}}\phi(y) \ dy \]

    and 

    \[ \lim_{t \to \infty} w(x,t) = \frac{1}{\sqrt{4\pi k}}\int_{-\infty}^\infty \phi(y) \ dy \]

    and we see that

    \[ u(x,t) = \frac{1}{\sqrt{t}}\left(\frac{1}{\sqrt{4\pi k}}\int_{-\infty}^\infty \phi(y) \ dy \right) + o\left(\frac{1}{\sqrt{t}}\right) \]

    so as $t \to \infty$, $u \to 0$.
\end{enumerate}