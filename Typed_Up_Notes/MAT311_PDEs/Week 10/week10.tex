\chapter{Week 10}

\section{The Laplace Equation}

We now pivot towards studying the Laplace equation. In dimensions 1,2, and 3, the Laplace equation is written as

\[ u_{xx} = 0 \quad u_{xx} + u_{yy} = 0 \quad u_{xx} + u_{yy} + u_{zz} = 0 \]

We may also write this equation as 

\[ \Delta u = 0 \]

where $\Delta$ is the \textbf{Laplacian} operator. 

This equation deals with situations where a solution is in an \textit{equilibrium/stationary} state, that is, they are independent of time. Take for example the heat equation in 1 dimension,

\[ u_t - ku_{xx} = 0 \]

If our solution is in a stationary state, then it is independent of time, meaning the $u_t$ is gone, leaving us with the 1 dimensional Laplace equation. 

In a sense, the Laplace equation is the most relevant PDE in mathematical physics, playing an important role in electrostatics, steady fluid flow, and Brownian motion, among other areas. Studying this equation in detail - like we did for the heat and wave equations - is somewhat outside the scope of the course, but we can still look at it and understand some interesting properties.

\subsection{The Maximum/Minimum Principle}

Like the heat equation, the Laplace equation has a Max/Min Principle, though, it is a little more complex compared to before. 

Our domain is now not a square of values, but rather any open set $D \subset \R^2$ with a ``nice" boundary (``nice" just means that it doesn't cause us any problems). We suppose that, inside of $D$,

\[ \Delta u = 0 \]

and that $u$ extends continuously to $\partial D$. We know that $D \cup \partial D$ is a closed and bounded subset of $\R^2$, so the Extreme Value Theorem tells us that $u$ attains a maximum and minimum on $D$. Take a wild guess where they are...

\begin{theorem}[The Maximum/Minimum Principle for Laplace Equations]
    Let $D \subset \R^2$ and $u$ be as above, then

    \[ \max_D u = \max_{\partial D}u \quad \text { and } \quad \min_D u = \min_{\partial D} u \]
\end{theorem}

One may notice that this is somewhat related to the 2nd Derivative Test: If a critical point likes within the interior, the $\nabla u = 0$. If that point is a strict extremum, then $\nabla^2 u$ has all positive or negative eigenvalues (depending on if it's a max or min). Now $\Delta = \operatorname{tr}\nabla^2 u$. However, if we have a strict max/min, this value will be positive or negative, and not 0, meaning the point cannot be an extremum. 

\subsubsection*{Uniqueness of the Dirichlet Problem}

The Maximum/Minimum Principle can allow us to prove, like in studying the heat equation, that the Laplace equation has unique solutions in certain scenarios. Let's consider the Laplace equation with Dirichlet boundary conditions. We let $D$ be a bounded, open subset of $\R^2$ and let

\[ h: \partial D \to \C \]

be a smooth function, and $F: D \to \C$ be a smooth too. We seek to find a solution to

\[ \begin{cases}
    u: D \to \C \\ \Delta u = F \\ u|_{\partial D} = h
\end{cases} \]

For simplicity, we can take $F = 0$. 

\begin{prop}
    The above problem has at most 1 solution.
\end{prop}

\begin{proof}
    Suppose that both $u_1, u_2$ solve the problem. We define

    \[ w = u_1 - u_2 \]

    We see that

    \begin{align*}
        \Delta w & = \Delta(u_1 - u_2) \\
        & = \Delta u_1 - \Delta u_2 \\ 
        & = 0 \\
        w|_{\partial D} & = (u_1 - u_2)|_{\partial D} \\
        & = u_1|_{\partial D} - u_2|_{\partial D} \\
        & = 0
    \end{align*}

    so the problem for $w$ is

    \[ \begin{cases}
        \Delta w = 0 \\
        w|_{\partial D} = 0
    \end{cases} \]

    so the Maximum/Minimum Principle applies. We get that

    \[ \max_D w = \max_{\partial D} w = 0 \]
    \[ \min_D w = \min_{\partial D} w = 0 \]

    hence $w = 0$, meaning $u_1 = u_2$.
\end{proof}

\subsection{Invariance Under Rigid Motion}

Another important property of the Laplace equation is that is invariant under \textit{rigid motions}, that is, the symmetries of the plane. There are 2 such symmetries:

\begin{enumerate}[label=(\roman*)]
    \item Translation: $(x,y) \mapsto (x+a, y+b)$ for constants $a,b$. 
    \item Rotation: $(x,y) \mapsto (x\cos\alpha + y\sin\alpha, x\sin\alpha + y\cos\alpha)$ for some $\alpha \in[0,2\pi)$.
\end{enumerate}

What's special about this is that $\Delta$ is the only differential operator with this property. This makes is perfect for studying \textit{isotropic} situations in engineering, where there is no preferred direction. 

The fact that we have rotational invariance suggests that the Laplace equation would have a simpler form when converted to \textit{polar coordinates}. Let's compute it. Given polar coordinates $(r, \theta)$, we have

\[ x = r\cos\theta, \quad y = r\sin \theta \]
\[ r = \sqrt{x^2 + y^2}, \quad \theta = \arctan\left(\frac{y}{x}\right) \]

Thus, we get that 

\[ \pardif{}{x} = \pardif{}{r}\pardif{r}{x} + \pardif{}{\theta}\pardif{\theta}{x} \]
\[ \pardif{r}{x} = \left[\frac{1}{2}(x^2 + y^2)^{-1/2}(2x)\right] = \frac{x}{\sqrt{x^2 + y^2}} = \frac{r\cos\theta}{r} = \cos\theta \]
\[ \pardif{\theta}{x} = \frac{1}{1 + \left(\frac{y}{x}\right)^2}\left(-\frac{y}{x^2}\right) = \frac{-y}{x^2 + y^2} = \frac{-r\sin\theta}{r^2} = \frac{-\sin\theta}{r} \]

so 

\[ \pardif{}{x} = \pardif{}{r}\cos\theta - \pardif{}{\theta}\frac{\sin\theta}{r} \]

Doing the same thing for $\pardif{}{y}$ gives

\[ \pardif{}{y} = \pardif{}{r}\sin\theta + \pardif{}{\theta}\frac{\cos\theta}{r} \]

Thus, we can rewrite the Laplacian as 

\begin{align*}
    \partial_x^2 + \partial_y^2 & = \left(\cos\theta\partial_r - \frac{\sin\theta}{r}\partial_\theta\right)^2 + \left(\sin\theta\partial_r + \frac{\cos\theta}{r}\partial_\theta\right)^2 \\
    & =\cos^2\theta\partial_r^2 + \frac{\cos\theta\sin\theta}{r^2}\partial_\theta - \frac{\cos\theta\sin\theta}{r}\partial_{r\theta} + \frac{\sin^2\theta}{r}\partial_r - \frac{\sin\theta\cos\theta}{r}\partial_{r\theta} + \frac{\sin\theta\cos\theta}{r^2}\partial_\theta + \frac{\sin^2\theta}{r^2}\partial_\theta^2 \\
    & \quad + \sin^2\theta\partial_r^2 - \frac{\sin\theta\cos\theta}{r^2}\partial_\theta + \frac{\sin\theta\cos\theta}{r}\partial_{r\theta} + \frac{\cos^2\theta}{r}\partial_r + \frac{\cos\theta\sin\theta}{r}\partial_{r\theta} - \frac{s\in\theta\cos\theta}{r^2}\partial_\theta + \frac{\cos^2\theta}{r^2}\partial_\theta \\
    & = \partial_r^2 + \frac{1}{r}\partial_r + \frac{1}{r^2}\partial_\theta^2
\end{align*}

We can repeat this process with any dimension. For instance if we converted the 3 dimension Laplace equation $\partial_x^2 + \partial_y^2 + \partial_z^2 = 0$ into spherical coordinates $(r, \theta, \varphi)$, we'd get that

\[ \Delta = \partial_r^2 + \frac{2}{r}\partial_r + \frac{1}{r^2\sin\theta}\partial_\theta(\sin\theta\partial_\theta) + \frac{1}{r\sin\theta}\partial_\varphi^2 \]

Given that $\Delta$ is symmetric when a rotation is applied, one might wonder if there also exist solutions which are symmetric when rotation is applied. Such solutions, called \textbf{spherically symmetric} solutions, only depend on $r$:

\[ \Delta u = 0, \quad u = u(r) \]

Given the 2 dimension Laplace in polar coordinates, we get that

\[ \frac{d^2}{dr^2} + \frac{1}{r}\frac{du}{dr} = 0 \]     

This is an ODE which has solutions given boundary

\[ u(r) = A\log(r) + B \]

where $A,B$ are constants. Logarithms will play an important role later on. 

When we go to 3 dimensions, the spherically solutions satisfy

\[ \frac{d^2u}{dr^2} + \frac{2}{r}\frac{du}{dr} = 0 \]

which have solutions

\[ u = \frac{A}{r} + B \]

You might notice a pattern, and there is one! For dimension $n \geq 3$, we get that the spherically symmetric solutions satisfy

\[ \frac{d^2u}{dr^2} + \frac{(n-1)}{r}\frac{du}{dr} \]

and satisfy

\[ u = \frac{A}{r^{n-2}} + B \]

\subsection{Coefficients of the Laplace Equation}

By invariance under rotation, solutions to the Laplace equation, expressed in polar coordinates, are periodic in $\theta$ with period $2\pi$. Thus, we may write them as a full Fourier series with the complex exponential.

\[ u(r,\theta) = \sum_{n=-\infty}^\infty u_n(r)e^{in\theta} \]

This is kind of like separation of variables in the sense that we take a product of a function of $r$ with a function of $\theta$. 

\begin{remark}
    The textbook writes $u$ in terms of sine and cosine. We use the complex exponential for ease of calculations, but one can also formulate all of this using the trigonometric functions. 
\end{remark}

If $\Delta u = 0$, should we expect a nice equation for $u_n(r)$? For arrbitrary PDEs, probably not, but for the Laplace equation we do, particularly in polar coordinates. This is because the Laplacian can be broken up into two components: one with $\partial_r$ terms and another with $\partial_\theta$ terms. The $\partial_r$ terms will not affect the exponential, and for the other term, we get

\[ \frac{\partial^2}{\partial\theta^2} e^{in\theta} = (in)^2e^{in\theta} \]

so it preserves the exponential. This is just like our previous discussion on Fourier series with sine and cosine: we will get the original term back, times some constant.

We claim that, for each $n$, 

\[ u_n''(r) + \frac{1}{r}u_n'(r) - \frac{n^2}{r^2}u_n(r) = 0 \]

Why? As $\Delta u = 0$, we have that

\[ \frac{1}{2\pi}\int_{-\pi}^\pi \left(\partial_r^2 u + \frac{1}{r}\partial_r u + \frac{1}{r^2}\partial_\theta^2u\right)e^{-in\theta} d \theta = 0 \]

Breaking up the left side into three integrals, we get

\begin{align*}
    \frac{1}{2\pi}\int_{-\pi}^\pi \partial_r^2u e^{-in\theta}d \theta & = \frac{d^2}{dr^2}\left[\frac{1}{2\pi}\int_{-\pi}^\pi u e^{-in\theta} \ d\theta \right] \\
    & = u_n''(r) \\
    \frac{1}{2\pi}\int_{-\pi}^\pi \frac{1}{r}\partial_ru e^{-in\theta} \ d\theta & = \frac{1}{r}\dif{}{r}\left[\frac{1}{2\pi}\int_{-\pi}^\pi u e^{-in\theta}\right] \\
    & = \frac{1}{r}u_n'(r) \\
    \frac{1}{2\pi}\int_{-\pi}^\pi \frac{1}{r}\partial_u^2 e^{-in\theta} \ d\theta & = \frac{1}{2\pi}\partial_\theta u e^{-in\theta}\bigg|_{-\pi}^\pi - \frac{1}{2\pi}\int_{-\pi}^\pi \frac{1}{r^2}\partial_\theta u \partial_\theta e^{-in\theta} \ d\theta \\
    & = \frac{1}{2\pi}\int_{-\pi}^\pi \frac{1}{r^2}u(-in)^2e^{-in\theta} \ d\theta \\
    & = \frac{-n^2}{r^2}u_n(r) 
\end{align*}

where in the third integral we perform integration by parts twice. What is so nice about this ODE is that we can actually solve it. If we make a guess that $u_n(r) = r^\alpha$ for some $\alpha \in \R$, we get that

\begin{align*}
    & \alpha(\alpha - 1)r^{\alpha-2} + \alpha r^{n-2} - n^2r^{\alpha-2} = 0 \\
    \implies & \alpha(\alpha-1) + \alpha - n^2 = 0 \\
    \implies & \alpha^2 - n^2 = 0 \\
    \implies & \alpha = \pm n \in \Z
\end{align*}

It is known that for an ODE of this form, finding two linearly independent solutions gives us all possible solutions. Clearly these are linearly independent, so we get that

\[ u_n(r) = A_nr^{|n|} - B_nr^{-|n|} \]

for constants $A_n,B_n$. Note that we put absolute values on the exponents to ensure that the $A_n$ constant is always attached to the $r$ with positive exponent. Moreover, if $n = 0$,

\[ u_0''(r) + \frac{1}{r}u_0'(r) = 0 \]

meaning that $\alpha = 0$ and so our solution is

\[ u_0(r) = A_0 + B_0\log(r) \] 

Thus, we can express our solution as 

\[ u(r,\theta) = A_0 + B_0\log(r) + \sum_{\substack{n=-\infty \\ n \neq 0}}^\infty (A_nr^{|n|} + B_nr^{-|n|}) e^{in\theta} \]

Given we are working in polar coordinates, this formula is great for when for domains with a \textit{circular} boundary, like a ball or annulus. 

\subsection{The Dirichlet Problem on a Ball}

Solving the Dirichlet problem for the Laplace equation in an arbitrary domain $D$ is outside the scope of the course due to the massive number of possibilities for what $D$ looks like. Thus, we will restrict our study to particuarly nice domains. We begin by considering a ball of radius 1. We seek to solve

\[ \begin{cases}
    \Delta u = 0 & (x,y) \in B_1(0) = \{x^2 + y^2 < 1\} \\
    u|_{\partial B_1(0)} = h(x,y)
\end{cases} \]

which we can restate in polar coordinates as 

\[ \begin{cases}
    \Delta u = 0 & (r, \theta) \in \{r < 1, \theta \in [-\pi, \pi)\} \\
    u|_{r = 1} = h(\theta)
\end{cases} \]

To solve this, we use the formula from before. we first find our $A_n$ and $B_n$. 

For $n = 0$, observe that as $r \to 0$,

\[ A_0 + B_0\log(r) \to -\infty \]

because of $\log(r)$. We want $u_n$ to be well-defined, so we require that $B_0 = 0$. 

For $n \neq 0$, we again observe that as $r \to 0$,

\[ (A_nr^{|n|} + B_nr^{-|n|})e^{in\theta} \to -\infty \] 

because of $r^{-|n|}$. Thus, $B_n = 0$, too. So

\[ u(r,\theta) = \sum_{n=-\infty}^\infty A_nr^{|n|}e^{in\theta} \]

What about the $A_n$'s? Note that on the boundary, where $r = 1$,

\[ h(\theta) = u(1,\theta) = \sum_{n=-\infty}^\infty A_ne^{in\theta} \]

which is the full Fourier series for $h(\theta)$. Thus, the $A_n$'s correspond to the full Fourier coefficients of $h$:

\[ A_n = \frac{1}{2\pi}\int_{-\pi}^\pi h(\theta)e^{-in\theta} \ d\theta \]