\chapter{Week 5}

\section{Working on the Half-Line}

We are now going to change our problems slightly to only consider $x \in (0,\infty)$, with boundary conditions applied at $\{x = 0\}$. 

\subsection{The Heat Equation}

\subsubsection{Dirichlet Boundary Conditions}

We will consider the problem

\[ \begin{cases}
    v(x,t) : (0,\infty) \times (0,\infty) \to \R  \\
    v_t - kv_{xx} = 0 \\
    v(x,0) = \phi(x), \ v(0,t) = 0
\end{cases} \]

Recall that $v(0,t) = 0$ is a (homogeneous) Dirichlet boundary condition, meaning there is no heat at one end of the rod. Let's also assume that $\phi(0) = 0$. 

The trick to solve this problem is to use an \textbf{odd extension}. 

\begin{definition}
    A function $\psi: \R \to \R$ is \textbf{odd} if $\psi(-x) = -\psi(x)$. 

    Given a function $\phi(x) : (0, \infty) \to \R$, we may define the \textbf{odd extension} of $\phi$ by 

    \[ \phi_{odd}: \R \to \R, \ \phi_{odd}(x) = \begin{cases}
        \phi(x) & x > 0 \\
        -\phi(-x) & x < 0 \\
        0 & x = 0
    \end{cases} \]
\end{definition}

So what's the point of defining this? Suppose that $u(x,t): \R \to \R$ solves

\[ \begin{cases}
    u_t - ku_{xx} = 0 \\ u(x,0)  = \phi_{odd}(x)
\end{cases} \]

meaning that $u(x,t) = \int_{-\infty}^\infty S(x-y)\phi_{odd}(y) \ dy$. We claim that $u(x,t)$ is odd in $x$. To see this, first recall that $u$ is smooth for $t > 0$. We have that

\[ S(x,t) = \frac{1}{\sqrt{4\pi kt}}e^{\frac{-x^2}{4kt}} \]

Notice that $S(-x,t) = S(x,t)$, as we square $x$. Thus,

\begin{align*}
    u(-x,t) & = \int_{-\infty}^\infty S(-x-y,t)\phi_{odd}(y) \ dy \\
    & = \int_{-\infty}^\infty S(x+y,t)\phi_{odd}(y) \ dy \intertext{Set $z = -y, dz = dy$. This flips the integral, and so the minus signs cancel out, hence we get}\\
    & = \int_{-\infty}^\infty S(x-z,t)\phi_{odd}(-z) \ dz \\
    & = -\int_{-\infty}^\infty S(x-z,t)\phi_{odd}(z) \ dz \\
    & = -u(x,t)
\end{align*}

as desired. We also have that $u(0,t) = -u(0,t)$, so $u(0,t) = 0$. Graphically, this looks like some function which flips over the $x$-axis at $x = 0$: 

\begin{center}
    \includegraphics[width=0.7\textwidth]{Week 5/Odd Extension.png}
\end{center}

So, if we define $v(x,t) := u(x,t)$, then $v$ will solve the heat equation, and

\begin{align*}
    v(x,0) & = u(x,0) = \phi_{odd}(x) = \phi(x) \\
    v(0,t) & = u(0,t) = 0
\end{align*}

as desired. Thus,

\[ v(x,t) = \int_{-\infty}^\infty S(x-y,t)\phi_{odd}(y) \ dy \]

We can solve this to be in terms of $\phi$ rather than $\phi_{odd}$ as follows:

\begin{align*}
    v(x,t) & = \int_{0}^\infty S(x-y,t)\phi_{odd}(y) \ dy + \int_{-\infty}^0 S(x-y,t)\phi_{odd}(y) \ dy \\
    & = \int_{0}^\infty S(x-y,t)\phi(y) \ dy + \int_{-\infty}^0 S(x-y,t)\phi(-y) \ dy \intertext{Set $z = -y$ and $dz = -dy$. Thus,}\\
    & = \int_0^\infty S(x-y,t) \phi(y) \ dy - \int_0^\infty S(x+z,t)\phi(z) \ dz \\
    & = \int_0^\infty [S(x-y,t) - S(x+y,t)]\phi(y) \ dy
\end{align*}

\subsubsection{Neumann Boundary Conditions}

We can use a similar idea for the heat equation with Neumann boundary conditions:

\[ \begin{cases}
    u(x,t): (0,\infty) \times (0,\infty) \to \R \\
    u_t - ku_{xx} = 0 \\
    u(x,0) = \phi(x), \ u_x(x,0) = 0
\end{cases} \]

where we think of the condition as saying there is insulation at the end of the rod. 

The trick before was to use the fact that if $f$ is odd and continuous, then $f(0) = 0$. Now, if $f$ is \textit{even} and smooth, then $f'(0) = 0$. Why? By Taylor's Theorem,

\[ f(x) = f(0) + f'(0)x + O(x^2) \]

Thus,

\[ f(-x) = f(0) - f'(0)x + O(x^2)  \]

if $|x| \ll 1$, and $f(x) = f(-x)$, meaning $f$ is even, then the remainder is very small, so we can ignore it. Thus,

\[ f'(0) = -f'(0) \implies f'(0) = 0 \]

From here, our derivations are similar to before, only this time we use the \textbf{even extension} of $\phi$, given by

\[ \phi_{even}(x) = \begin{cases}
    \phi(x) & x \geq 0 \\
    \phi(-x) & x < 0
\end{cases} \]

at the end, we will see that

\[ u(x,t) = \int_0^\infty [S(x-y),t) + S(x+y,t)]\phi(y) \ dy \]

Note the slight difference from Dirichlet conditions: this time we add in the integral instead of subtract.

\subsection{The Wave Equation}

Let's do the same thing for the wave equation:

\subsubsection{Dirichlet Boundary Conditions}

We find $v(x,t): (0,\infty) \times (0,\infty) \to \R$ satisfying

\[ \begin{cases}
    v_{tt} - c^2v_{xx} = 0 \\
    v(x,0) = \phi(x), \ v_t(x,0) = \psi(x) \\
    v(0,t) = 0
\end{cases} \]

If we mimic the strategy for the heat equation by considering the odd extensions of $\phi, \psi$, then we will see that

\[ v(x,t) = \frac{1}{2}[\phi_{odd}(x+ct) + \phi_{odd}(x-ct)] + \frac{1}{2c}\int_{x-ct}^{x+ct}\psi(s) \ ds \]

Let's unpack this quickly to see what's going on. Consider the regions in the below figure, where the line is given by $x = ct$ in the $(x,t)$ plane:

\begin{center}
    \includegraphics[width=0.7\textwidth]{Week 5/Wave Equation Dirichlet Regions.png}
\end{center}

The region below the line, shaded in black, is where $x - ct < 0$, meaning $x > ct$. Here, we see that

\begin{align*}
    \phi_{odd}(x+ct) & = \phi(x+ct) \\
    \phi_{odd}(x-ct) & = \phi(x-ct) \\
    \int_{x-ct}^{x+ct}\psi_{odd}(s) \ ds & = \int_{x-ct}^{x+ct} \psi(s) \ ds
\end{align*}

So in this region, $v$ is just the usual solution to this form of the wave equation; one can see that this follows from causality of the wave equation. 

In the region above the line, shaded in blue, $x < ct$. Thus,

\begin{align*}
    \phi_{odd}(x+ct) & = \phi(x+ct) \\
    \phi_{odd}(x-ct) & = -\phi(ct-x) \\
    \frac{1}{2c}\int_{x-ct}^{x+ct} \psi_{odd}(s) \ ds & = \frac{1}{2c}\int_{0}^{x+ct} \psi_{odd}(s) \ ds + \frac{1}{2c}\int_{x-ct}^0 \psi_{odd}(s) \ ds \\
    & = \frac{1}{2c}\int_{0}^{x+ct} \psi(s) \ ds - \frac{1}{2c}\int_{x-ct}^0 \psi(-s) \ ds \intertext{taking $\tau = -s, d\tau = -ds$, we get} \\
    & = \frac{1}{2c}\int_{0}^{x+ct} \psi(s) \ ds - \frac{1}{2c}\int_0^{ct-x}\psi(\tau) \ d\tau \\
\end{align*}

So in this region,

\[ v(x,t) = \frac{1}{2}[\phi(x+ct) - \phi(ct-x)] + \frac{1}{2c}\int_{ct-x}^{ct+x} \psi(s) \ ds \]

What does this mean? Think of that path light cone from before, and suppose that one of its rays hits the $t$-axis in the $(x,t)$ plane. What will happen is that ray will bounce off the axis at an angle, before shooting back towards the $x$-axis. As $t$ increases, this corresponds to the path light cone perfectly bouncing off of the $t$-axis, and now translating to the right instead of the left. This situation will occur when, for example, we have an infinitely long string which we nail down at one end. 

\begin{center}
    \includegraphics[width=0.7\textwidth]{Week 5/bouncing off the t axis.png}
\end{center}

In a long term setting, recall that 

\[ \lim_{x \to \infty} u(x,t) = \frac{1}{2c}\int_{-\infty}^\infty \psi(s) \ ds \]

when $\phi, \psi$ compactly supported. What happens when we consider a Dirichlet boundary condition? Well, if $t \gg 1$, with $x$ fixed, then we will lie in the blue region from before, and $x + ct, ct - x$ will become both positive and very large, eventually reaching outside the support of $\phi$ and $\psi$. This means that

\[ \int_{ct-x}^{xt+x} \psi(s) \ ds = 0 \]

once $ct-x$ lies outside of $\psi$'s support. Thus,

\[ \lim_{t \to \infty}u(x,t) = 0 \]

Visually, this just means that if we flick our string, one wave will move off to the right, and the other to the left, where it will reflect at the point $x = 0$ (since it is fixed at 0), before following the other wave to the right. Thus, for very large values of $t$, the string will not have moved at all. 

\section{The Heat Equation, with a Source}

Before, we always set our PDE to be equal to 0. However, there are certain instances where we would rather have it be equal to some nonzero function of $x,t$. In this section, we will see how to solve the heat equation under these conditions. 

\subsection{No Boundary Conditions}

We will try and solve $u(x,t): \R \times (0,\infty) \to \R$ such that

\[ \begin{cases}
    u_t - ku_{xx} = f(x,t) \\ u(x,0) = \phi(x)
\end{cases} \]

where $f$ is some function. This represents heat \textit{entering} the system, like putting a hot coal on top of our rod. 

Deriving a solution will require us to use \textbf{Duhamel's Principle}, which tells us that, from the standard IVP for the heat equation (the one without $f$), we can get a formula for the IVP with $f$. 

The solution is given by

\[ u(x,t) = \int_{-\infty}^\infty S(x-y,t)\phi(y) \ dy + \int_0^t\int_{-\infty}^\infty S(x-y,t-s)f(y,s)\ dy \ ds \]

where $S(x,t)$ is as before. Let's first check that this works. Without loss of generality, we can set $\phi(x) = 0$. This is because, by linearity, we take $u_1, u_2$ so that

\[ \begin{cases}
    (u_1)_t - k(u_1)_{xx} = 0 \\ u_1(x,0) = \phi(x)
\end{cases} \quad \text{and} \quad \begin{cases}
    (u_2)_t - k(u_2)_{xx} = f(x,t) \\ u_2(x,0) = 0
\end{cases} \]

Then setting $u := u_1 + u_2$, $u$ solves the problem we are dealing with. 

To check our solution, we use the following formula for differentiation of a definite integral:

\[ \frac{d}{dt}\int_{a(t)}^{b(t)} h(t,x) \ dx = h(b(t),x)b'(t) - h(a(t),x)a'(t) + \int_{a(t)}^{b(t)}h_t(t,x) \ dx \]

Since the integral sends $t-s$ to 0, we get that 

\begin{align*}
    u_t & = \lim_{\tau \to 0}\int_{-\infty}^\infty S(x-y,\tau)f(y,t) \ dy + \int_0^t \int_{-\infty}^\infty \partial_tS(x-y,t-s) f(y,s) \ dy \ ds \\
    & = f(x,t) + \int_0^t \int_{-\infty}^\infty \partial_tS(x-y,t-s) f(y,s) \ dy \ ds \\
    u_x & = \int_0^t\int_{-\infty}^\infty \partial_xS(x-y,t-s)f(y,s) \ dy \ ds \\
    u_{xx} & = \int_0^t\int_{-\infty}^\infty \partial_x^2S(x-y,t-s)f(y,s) \ dy \ ds
\end{align*}

Thus, we get that

\[ u_t - ku_{xx} = f(x,t) + \int_0^t\int_{-\infty}^\infty (\partial_tS -k \partial_x^2S)(x-y,t-s)f(y,s) \ dy \ ds = f(x,t) \]

which is what we want. 

\subsubsection{The General Idea}

We can generalize this idea to any equation of the form $u(x,t), f(x,t)$ for which we get

\[ \begin{cases}
    \partial_t u - Lu = f \\ u(x,t) = \phi(x)
\end{cases} \]

where $L$ is a differential operator depending on $x$ (it doesn't have to, but for now it will). 

Assume that for each $\phi(x)$ we have a function $P_\phi(x,t)$ such that

\[ \partial_tP_\phi - LP_\phi = 0 \quad P_\phi(x,0) = \phi(x) \]

We get that $P_\phi$ solves the heat equation, and so 

\[ P_\phi(x,t) = \int_{-\infty}^\infty S(x-y,t)\phi(y) \ dy \]

Using this, we can now solve our problem, with the solution given by 

\[ u(x,t) = \int_0^t P_{f(x,s)}(x,t-s) \ ds \]

We have that

\[ \partial_t u = P_{f(x,s)}(x,0) + \int_0^t \partial_tP_f(x,t-s) \ ds = f(x,t) + \int_0^t \partial_tP_f(x,t-s) \ ds \]

\[ Lu = \int_0^t L(P_f)(x,t-s) \ ds \]

Thus,

\[ \partial_tu - Lu = f(x,t) + \int_0^t (\partial_tP_f - LP_f)(x,t-s) \ ds = f(x,t) \]

Now, solving for $P$ might be hard, but once we find it, solving the problem with a source is actually pretty easy. For example, we can use this strategy for the wave equation. 

\subsection{Dirichlet Boundary Conditions: The Half-Line  with a Source}

We consider the half-line problem from before, now with a source:

\[ \begin{cases}
    v(x,t): (0,\infty) \times (0,\infty) \to \R \\ 
    v_t - kv_{xx} = f(x,t) \\
    v(x,0) = \phi(x) \\
    v(0,t) = 0
\end{cases} \]

There are two approaches to this: We can use odd reflections of $\phi$ and $f$ in $x$, and use the formula we just derived, or, we can start with the formula when $f = 0$, which we solved last week, and use Duhamel's Principle. Regardless of method, we will get that

\[ v(x,t) = \int_0^\infty [S(x-y,t) - S(x+y,t)]\phi(y) \ dy + \int_0^t\int_0^\infty [S(x-y,t-s) - S(x+y,t-s)]f(y,s) \ dy \ ds \]

\subsection{An Inhomogeneous Boundary Condition}

We can use the methods for equations with sources to solve the heat equation with an \textit{inhomogenous} boundary condition. Consider the problem

\[ \begin{cases}
    v_t - kv_{xx} = 0 \\ v(x,0) = 0 \\ v(0,t) = h(t)
\end{cases} \]

for $x \in (0,\infty)$. We can transform it into a problem we've already done by setting

\[ w(x,t) := v(x,t) - h(t) \]

We have that

\begin{align*}
    w_t - kw_{xx} & = v_t - kv_{xx} - h'(t) \\
                  & = -h'(t) \\
    w(0,t) & = v(0,t) - h(t) \\
           & = h(t) - h(t) \\
           & = 0 \\
    w(x,0) & = v(x,0) - h(0) \\
           & = -h(0)
\end{align*}

Thus, we get the problem

\[ \begin{cases}
    w_t - kw_{xx} = h'(t) \\ w(x,0) = -h(0) \\ w(0,t) = 0
\end{cases} \]

which is a problem that we just did previously. Thus,

\[ w(x,t) = \int_0^\infty [S(x-y,t) - S(x+y,t)](-h(0)) \ dx + \int_0^t \int_0^\infty [S(x-y,t-s) - S(x+y,t-s)](-h'(s)) \ dy \ ds \]