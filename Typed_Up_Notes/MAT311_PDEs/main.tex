\documentclass[11pt, oneside]{book}
\usepackage{fix-cm}
\newcommand{\dif}[2]{\dfrac{d#1}{d#2}}
\newcommand{\pardif}[2]{\dfrac{\partial #1}{\partial #2}}
\newcommand{\twopardif}[2]{\dfrac{\partial ^2 #1}{\partial #2^2}}



\usepackage[margin=1.1in]{geometry}
\usepackage{amsfonts}
\usepackage{amsmath}
\usepackage{amssymb}
\usepackage{amsthm}
\usepackage{enumitem}
\usepackage{fancyhdr}
\usepackage{graphicx}
\usepackage{multirow}
\usepackage{multicol}
\usepackage{hyperref}
\usepackage{esdiff}
\usepackage{tocloft}
\usepackage{titlesec}
\usepackage{caption}
\usepackage{array}
\usepackage{dsfont}
\RequirePackage[breakable, skins, theorems]{tcolorbox}

% Tikz and PGFPlots setup
\usepackage{tikz}
\usepackage{pgfplots}
\pgfplotsset{compat=1.18}
\usetikzlibrary{intersections} % Add this line to load the intersections library
\usepgfplotslibrary{fillbetween}
\usetikzlibrary{patterns}
\usetikzlibrary{fadings}
\usetikzlibrary{matrix}
\usetikzlibrary{backgrounds}
\usetikzlibrary{calc}
\usetikzlibrary{graphs}
\usetikzlibrary{shapes.geometric}
\usetikzlibrary{decorations.markings}

\makeatletter 
\renewcommand{\sectionmark}[1]{\markboth{#1}{}}
\renewcommand\ps@plain{\let\@mkboth\@gobbletwo
     \let\@oddhead\@empty
     \def\@oddfoot{\reset@font\hfil}
     \let\@evenhead\@empty\let\@evenfoot\@oddfoot}
\makeatother


% Table of Contents, plus various title formattings
%%%%%%%%%%%%%%%%%%%%%%%%%%%%%%%%%%%%%%%%%%%%%%%%%%%%%%%%
\renewcommand{\cfttoctitlefont}{\sffamily\Huge\bfseries\filcenter}
\renewcommand{\cftaftertoctitle}{\par\noindent\hrulefill\par}
\renewcommand{\cftchapfont}{\sffamily\Large\bfseries\vspace*{0.5em}}
\renewcommand{\cftchappagefont}{\Large\bfseries}
\renewcommand{\cftpartfont}{\sffamily\Large\bfseries\hfill}
\renewcommand{\cftpartpagefont}{\color{white}}
\renewcommand{\cftbeforepartskip}{2.0em}
\setlength{\cftbeforechapskip}{2.0em} 
% \setlength{\cftbeforesecskip}{1pt} 
\setlength{\cftchapnumwidth}{25pt} 
\setlength{\cftsecindent}{2.5em} % Increase the indentation for sections
\titleformat{\chapter}[display]
  {\normalfont\sffamily\Huge\bfseries}
  {\normalfont \bfseries \flushright\fontsize{44}{44}\selectfont\thechapter} 
  {20pt} 
  {\Huge\bfseries} 
  [\vspace{5pt}\titlerule] 


\titlespacing*{\chapter}{0pt}{-20pt}{40pt} 
\titleformat{\section}
  {\sffamily\large\bfseries\MakeUppercase} % Uppercase section titles in text
  {\thesection}{1em}{}



\titleformat*{\subsection}{\sffamily \bfseries \MakeUppercase}
\titleformat*{\subsubsection}{\sffamily \bfseries}
\titleformat{\part}[display]
    {\sffamily\Huge\bfseries\filcenter}
    {\partname\space\thepart}
    {20pt}
    {}
%%%%%%%%%%%%%%%%%%%%%%%%%%%%%%%%%%%%%%%%%%%%%%%%%%%%%%%%

\linespread{1.05}

% Suppress warnings 
\setlength{\headheight}{15pt}
\addtolength{\topmargin}{-2.5pt}

% List spacing
%%%%%%%%%%%%%%%%%%%%%%%%%%%%%%%%%%%%%%%%%%%%%%%%%%%%%%%%
\setlist[enumerate]{%
	itemsep=0.0625em,
	topsep=0.125em%
}%

\setlist[itemize]{%
	itemsep=0.25em,
	topsep=0.25em%
}%
%%%%%%%%%%%%%%%%%%%%%%%%%%%%%%%%%%%%%%%%%%%%%%%%%%%%%%%%

% Header and links
%%%%%%%%%%%%%%%%%%%%%%%%%%%%%%%%%%%%%%%%%%%%%%%%%%%%%%%%
\pagestyle{fancy}
\rhead{\sffamily \bfseries \thepage}
\lhead{\sffamily Chap. \thechapter \quad \leftmark}
\cfoot{}  % Clear center footer
\renewcommand{\headrulewidth}{0pt}
% Hyperref
\usepackage{hyperref}
    \hypersetup{
        colorlinks=true,
		    linkcolor=black,
    }
%%%%%%%%%%%%%%%%%%%%%%%%%%%%%%%%%%%%%%%%%%%%%%%%%%%%%%%%

\usepackage{xcolor}
\definecolor{custombg}{HTML}{293133}
\definecolor{randombs}{HTML}{fff3ec}
% \pagecolor{custombg}
% \color{white}


% Boxes for Questions, Theorems, etc.
\newtcolorbox{pbox}[1]{
	colback = custombg!100,
	colframe = black!5,
	title = #1,
	parbox = false,
	fontupper = \color{white},
	title=\textcolor{black}{#1},
}

% Theorems
\newtheoremstyle{boldslanted}
  {\topsep}  
  {\topsep}   
  {\slshape}  
  {0pt}
  {\bfseries}
  {.}        
  {5pt plus 1pt minus 1pt} 
  {} 

  \NewDocumentCommand{\newtheoremstyled}{m m m}{%
  \newtcbtheorem[number within=chapter]{#1}{#2}{%
    colback=custombg,
    colframe=#3,
    fonttitle=\bfseries,
    fontupper=\color{#3}\slshape,
    enhanced,
    sharp corners,
    coltitle=#3,
    attach boxed title to top left={%
      xshift=2ex,
      yshift=-3.5mm,
      yshifttext=-2mm},
    boxed title style={%
      sharp corners,
      colframe=custombg,
      colback=custombg}
  }{th}
}

\definecolor{lightcoral}{HTML}{FFE0D6}
\definecolor{coral}{HTML}{FFDBCA}
\definecolor{mintblue}{HTML}{D6F2EE} 
\definecolor{purple}{HTML}{E4C4E6}

\newtheoremstyled{mytheo}{Theorem}{lightcoral}
\newtheoremstyled{mydef}{Definition}{mintblue}
\newtheoremstyled{myex}{Excursion}{purple}

% Theorems
%%%%%%%%%%%%%%%%%%%%%%%%%%%%%%%%%%%%%%%%%%%%%%%%%%%%%%%%
\theoremstyle{boldslanted}
\newtheorem*{aim}{Aim}
\newtheorem*{claim}{Claim}
\newtheorem{corollary}{Corollary}
\newtheorem*{conjecture}{Conjecture}
\newtheorem{example}{Example}
\newtheorem{theorem}{Theorem}[chapter]
\newtheorem{definition}[theorem]{Definition}
\newtheorem{lemma}[theorem]{Lemma}
\newtheorem*{notation}{Notation}
\newtheorem{prop}[theorem]{Proposition}
\newtheorem{question}{Question}
\newtheorem*{qc}{Quick Check}
\newtheorem*{jp}{Journal Prompt}

\newtheorem*{remark}{Remark}
\newtheorem*{solution}{Solution}

\theoremstyle{definition}
\newtheoremstyle{sfaxiom}
  {\topsep}   
  {\topsep}   
  {\normalfont} 
  {0pt}       
  {\sffamily\bfseries} % <-- Sans serif heading
  {.}         
  {5pt plus 1pt minus 1pt}
  {}   
\newtheorem{com}{Comment}
\newtheorem{excursion}{Excursion}[chapter]
\newtheorem{problem}{Problem}
\newtheorem*{note}{Note}
\newtheorem{exercise}{Exercise}[chapter]
\theoremstyle{sfaxiom}
\newtheorem{axiom}{Axiom}
%%%%%%%%%%%%%%%%%%%%%%%%%%%%%%%%%%%%%%%%%%%%%%%%%%%%%%%%

\renewcommand{\labelitemi}{--}
\renewcommand{\labelitemii}{$\circ$}
% \renewcommand{\labelenumi}{(\alph{*})}

% Special sets
%%%%%%%%%%%%%%%%%%%%%%%%%%%%%%%%%%%%%%%%%%%%%%%%%%%%%%%%
\newcommand{\N}{\mathbb{N}}
\newcommand{\Q}{\mathbb{Q}}
\newcommand{\R}{\mathbb{R}}
\newcommand{\Z}{\mathbb{Z}}
\newcommand{\C}{\mathbb{C}}
%%%%%%%%%%%%%%%%%%%%%%%%%%%%%%%%%%%%%%%%%%%%%%%%%%%%%%%%

% Brackets
%%%%%%%%%%%%%%%%%%%%%%%%%%%%%%%%%%%%%%%%%%%%%%%%%%%%%%%%
\newcommand{\abs}[1]{\left\lvert #1\right\rvert}
\newcommand{\brak}[1]{\left\{#1\right\}}
%%%%%%%%%%%%%%%%%%%%%%%%%%%%%%%%%%%%%%%%%%%%%%%%%%%%%%%%

% Adding text above eqal sign
\newcommand{\eqtext}[1]{\ensuremath{\stackrel{\text{#1}}{=}}}

%new commands
\renewcommand{\d}{\,\mathrm{d}}

% Proof environment
\newenvironment{myproof}[1][\proofname]{%
  \proof[\rm \bf #1]%
}{\endproof}

\definecolor{blurple}{HTML}{5539CC}
% Custom colored footnote
%%%%%%%%%%%%%%%%%%%%%%%%%%%%%%%%%%%%%%%%%%%%%%%%%%%%%%%%
\newcommand{\pfootnote}[2][black]{%
    \renewcommand{\thefootnote}{\textcolor{#1}{\arabic{footnote}}}%
    \footnote{\textcolor{#1}{#2}}%
}
%%%%%%%%%%%%%%%%%%%%%%%%%%%%%%%%%%%%%%%%%%%%%%%%%%%%%%%%


% Centered subsubsection command
\newcommand{\centeredsubsubsection}[1]{%
  \subsubsection*{\centering#1}
}

\definecolor{left} {HTML}{293133}

\renewcommand{\maketitle}{
    \begin{titlepage}
        \begin{tikzpicture}[remember picture,overlay]
            \node [
                shading = axis,
                rectangle,
                left color=left,
                right color=left!85!white,
                shading angle=90,
                anchor=north west,
                minimum width=\paperwidth,
                minimum height=\paperheight
            ] at (current page.north west) {};
        \end{tikzpicture}

        \begin{center}
        \hspace{0pt}
            \vfill
            \fontsize{35}{35}\selectfont \textbf{ \color{white} MAT311} \\
            \vspace{0.3cm}
            \fontsize{35}{35}\selectfont \textbf{ \color{white} Partial Differential Equations} \\
            \vspace{0.3cm}
            \fontsize{20}{20}\selectfont \textbf{ \color{white} by Brandon Papandrea}
            \vfill
        \end{center}
    \end{titlepage}
}

\begin{document}     
\pagenumbering{roman}
\maketitle

\newpage

The following is based on lecture notes taken during the Fall 2025 offering of MAT311: Partial Differential Equations, at the University of Toronto - Mississauga, and are based on the first six chapters of the textbook \textit{Partial Differential Equations: An Introduction} by W.A. Strauss. The notes are broken up in sections based on the week they were taught, and not necessarily broken up based on the textbook chapters. The intention is for these notes to be a polished version of my own lecture notes that allows me to revise and look over the material multiple times, and thus should not be considered a primary source for learning about PDEs or their applications. 

\newpage

\tableofcontents
\clearpage

\pagenumbering{arabic}
\setcounter{page}{1}

\chapter{Week 1}

\section{Matchings in Bipartite Graphs}

We let $G = (L,R,E)$ be a bipartite graph, where $E \subseteq L \times R$.

\begin{center}
    \includegraphics[width=0.3\textwidth]{Lectures/Example Bipartite.png}
\end{center}

A \textbf{matching} in $G$ is a subset $M \subseteq E$ such that no two edges in $M$ share a vertex. If $|L| = |R| = n$, then we can consider a \textbf{perfect matching}, which is a matching $M$ such that $|M| = n$. 

\subsection{Hall's Theorem}

Given an bipartite graph $G$ with $|L| = |R|$, when does $G$ contain a perfect matching? 

Obviously, one condition is that each vertex must have a non-zero degree. The graph on the left has vertices of the degree 0, so no perfect matching exists, where as the graph on the right has a perfect matching, shown in pink. 

\begin{center}
    \includegraphics[width=0.5\textwidth]{Lectures/No Perfect v Perfect.png}
\end{center}

Even if this condition is met, we might still have problems. Conside the graph below:

\begin{center}
    \includegraphics[width=0.3\textwidth]{Lectures/Graph Bottleneck.png}
\end{center}

Notice that there are bottlenecks, areas where vertices have degree 1, and their repsective edges meet at the same vertex. No matter which edge we choose, we will have to exclude one of these vertices from our matching, hence no perfect matching exists. 

Surprisingly, finding one of these bottlenecks is necessary and sufficient in showing that a graph has no perfect matching. This is known as \textbf{Hall's Theorem}, and while it is an if and only if condition, it is not efficient if we do not know where the bottleneck is; this is called an NP Problem.

To formulate the theorem, we first state a definition. For a subset of vertices $S \subseteq L$ in a bipartite graph $G$, we define the neighbourhood of $S$ in $G$ as 

\[N_G(S) = \{v \in R : \exists u \in S, (u,v) \in E\}\]

\begin{theorem}[Hall's Theorem]
    If $G$ has no perfect matching, then there exists a set $S \subseteq L$ such that $|N_G(S)| < |S|$. 
\end{theorem}

\begin{proof}
    We proceed by induction on $n$, the number of vertices in $L$ and $R$. We seek to prove the contrapositive, that is, if $G$ is such that for all $S \subseteq L$, $|N_G(S)| \geq |S|$, then $G$ has a perfect matching. The claim is obvious if $n = 1$, so suppose $n > 1$ and the theorem is true for all values less than $n$. We split our proof into two cases:

    First, suppose that for all subsets $S$ with $0 < |S| < n$, $|N_G(S)| \geq |S| + 1$. Then let $(u,v) \in E$ be an edge and consider the graph 

    \[G' = (L\setminus\{u\}, R\setminus\{v\}, E\setminus\{(u,v)\})\]

    For all $S \subseteq L\setminus\{u\}$, $|N_{G'}(S)| \geq |S|$, so by the induction hypothesis, we have a perfect matching $M'$ in $G'$. Then the matching 

    \[M' \cup \{(u,v)\}\]

    is perfect in $G$, as desired.

    Now, suppose that there is a subset $S \subseteq L$ with $0 < |S| < n$ such that $|N_G(S)| = |S|$. We define two subgraphs,

    \[G' = (S, N_G(S), E \cap (S \times N_G(S))) \quad G'' = (L\setminus S, R \setminus N_G(S), E \cap (L\setminus S \times R\setminus N(S)))\]

    By the induction hypothesis, $G'$ has a perfect matching, $M'$. Does it hold for $G''$. Let $T \subseteq L \setminus S$. We know by assumption that

    \[|N_G(S \cup T)| \geq |S \cup T| = |S| + |T|\]

    But we also know that 

    \[|N_G(S \cup T)| = |N_G(S)| + |N_{G''}(T)| = |S| + |N_{G''}(T)|\]

    Thus, $|N_{G''}(T)| \geq |T|$, and the induction hypothesis applies, giving us a perfect matching $M''$ in $G''$. Combining $M'$ and $M''$ gives the desired perfect matching in $G$. 
\end{proof}

\subsection{An Algorithm to Check for Matchings}

Let $G = (L,R,E)$ be a bipartite graph with $|L| = |R| = n$. How can we check if there is a perfect matching in $G$? There's actually a really easy way to compute algorithm that can allows us to check with near perfect accuracy if there is one. For now, let $G$ be the graph shown below:

\begin{center}
    \includegraphics[width=0.3\textwidth]{Lectures/Example Graph.png}
\end{center}

Let $A$ be the $n \times n$ adjacency matrix corresponding to $G$. The rows will represent vertices in $L$, and the columns will represent those in $R$. We let $A_{ij} = 1$ if $(i,j) \in E$, and 0 otherwise. For the above graph, it is

\[A = \begin{pmatrix}
    1 & 0 & 0 \\ 1 & 1 & 0 \\ 0 & 1 & 1
\end{pmatrix}\]

Now let's replace every non-zero value in $A$ with some random, non-zero real number. This will give us a new, modified adjacency matrix $\tilde{A}$:

\[\begin{pmatrix}
    7 & 0 & 0 \\ e & 13 & 0 \\ 0 & 1 & \pi
\end{pmatrix}\]

Now, we compute the determinant of $\tilde{A}$, which is $7 \cdot 13 \cdot \pi = 91\pi \neq 0$. Because the determinant is non-zero, we conclude that $G$ has a perfect matching. 

More explicitely, the algorithm to check for the existence of a perfect matching is as follows:

\begin{enumerate}
    \item Compute the adjacency matrix $A$
    \item For each edge $(i,j)$, pick a random number $\tilde{A}_{ij} \in \{1,2,\ldots, M\}$. For all other $(i,j)$, set $\tilde{A}_{ij} = 0$
    \item If $\det{\tilde{A}} = 0$, we have no perfect matching. If it is not zero, we do have a perfect matching
\end{enumerate}

One may look at this algorithm and ask if the choice of $\tilde{A}_{ij}$ for edges $(i,j)$ matters. Indeed it does. If $G$ has a perfect matching, not all choices will lead to a matrix with non-zero determinant, however, most choices will. If $G$ has no perfect matching, no matter what choices we make, the determinant will be 0. We formalize this in the following theorem:

\begin{theorem}
    If $G$, with $|L| = |R| = n$, has a perfect matching, and we let $\tilde{A}_{ij} \in \{1,\ldots, M\}$ for all edges $(i,j)$, then the probability that the above algorithm tells us $G$ has a perfect matching is at least $1 - \dfrac{n}{M}$.

    If $G$ has no perfect matching, the probability the algorithm tell us $G$ has no perfect matching is 1.
\end{theorem}

To prove this, we need to understand what the determinant function is representing when it is applied to an adjacency matrix. Given an $n \times n$ matrix $M$, the formula for the determinant is 

\[\det{M} = \sum_{\sigma \in S_n} (-1)^{\operatorname{sgn}(\sigma)}\prod_{i=1}^n M_{i,\sigma(i)}\]

\begin{example}
    Taking 

    \[M = \begin{pmatrix}
        a & b \\ c & d
    \end{pmatrix}\]

    Then as the only permutations in $S_2$ are the identity permuation and the transposition $(1 \ 2)$, we get that 

    \[\det{M} = M_{11}M_{22} + (-1)M_{12}M_{21} = ad - bc\]

    which we know to be the classic formula for the determinant of a $2 \times 2$ matrix. 
\end{example}

What happens when $M$ is an adjacency matrix? Well, each permutation in the sum will correspond to a hypothetical pairing of vertices in $L$ and $R$, meaning each term in the corresponding product is a hypothetical edge. Edges that do not exist in $G$ will be ommitted, as the corresponding entry in the matrix is 0.

Notice that if a perfect matching exists, there will be a permutation $\sigma$ that corresponds to it, meaning that the pairings $(i,\sigma(i))$ correspond to the edges in the matching. If $G$ has no perfect matching, then for all $\sigma \in S_n$, there is an $i$ such that $(i,\sigma(i)) \notin E$. Thus, $\prod_i \tilde{A}_{i,\sigma(i)} = 0$, and so $\det{\tilde{A}} = 0$. 

Now suppose that $G$ has a perfect matching. We define a matrix $M(x_{11},x_{12},x_{13},\ldots, x_{nn})$ by

\[M_{ij} = \begin{cases}
    x_{ij} & \text{ if } (i,j) \in E \\
    0 & \text{ otherwise}
\end{cases}\]

This is technically a function of $n^2$ variables that maps to a modified adjacency matrix. We then define $P(X_{11}, \ldots, X_{nn})$ to be the polynomial $\det{M}$. It suffices to show that $P(x_{11},\ldots,x_{nn})$ is a non-zero polynomial. The rough proof of this is as follows: take the permutation $\sigma$ corresponding to the perfect matching. We know this gives a non-zero term in $P$. Moreover, as this permuation does not appear in another term in the sum, it cannot be cancelled out by another term, hence we get a non-zero output to $P$. This is formalized as follows:

\begin{lemma}[Schwarz-Zippel Lemma]
    Let $|A| = M$ be a subset of $\R$. If $Q(y_1,\ldots, y_k)$ is a non-zero polynomial of degree at most $d$, then 

    \[\Pr[Q(\vec{y}) = 0] \leq \frac{d}{M}\]

    where $\vec{y} \in A^k$ is chosen uniformly.
\end{lemma}

\begin{proof}
    We proceed by induction on $k$, the number of variables in $Q$. First suppose that $k = 1$. Then we know that $Q$ has at most $d$ roots as it is of degree at most $d$. The probability of one of $M$ chosen numbers being one of these roots is at most $\dfrac{d}{M}$, as desired.

    Now let $k > 1$ and suppose the claim holds for $k-1$. We may write $Q$ as 

    \[Q(y_1, \ldots, y_{k-1},z) = Q_0(y_1,\ldots,y_{k-1}) + Q_1(y_1,\ldots,y_{k-1})z + \ldots + Q_t(y_1,\ldots,y_{k-1})z^t\]

    where $Q_i(y_1, \ldots, y_{k-1})$ is non-zero with degree at most $d-i$. We define a new set 

    \[B = \{\vec{y} : Q(\vec{y},z) = 0\}\]

    Now, observe that if all of $Q$ is non-zero, so too must $Q_t$. Using the induction hypothesis and conditional probability, we conclude that 

    \begin{align*}
        \Pr_{\substack{\vec{y} \in A^{k-1} \\ z \in A}}[Q(\vec{y},z) = 0] & = \Pr_{\vec{y} \in A^{k}}[\vec{y} \in B] + \Pr[Q(\vec{y},z) = 0 | \vec{y} \notin B] \cdot \Pr[\vec{y} \notin B] \\
        & = \frac{d - t}{M} + \frac{t}{M} \\
        & = \frac{d}{M}
    \end{align*}

    The first term on the second line comes from the fact that there are $d-t$ roots of $Q_t$, while the second term follows from there being $t$ roots of $Q$ that are not roots of $Q_t$. This completes the proof. 
\end{proof}

There is also a combinatorial way of phrasing this lemma, where we say that

\[|\{\vec{y} \in A^k : Q(\vec{y}) = 0\}| \leq dM^{k-1}\]
\chapter{Week 2}

\section{The Essential PDEs}

In this section we introduce the PDEs that will be studied in this class, all of which come from physics. We will study them (usually) in their simplest form, using the least number of variables. 

\subsection*{The Simple Transport Equation}

We consider a 1-dimensional system in which a fluid is traveling through a pipe at a constant rate. If we let $u(t,x)$ represent the amount of fluid moving through the pipe at position $x$ and at time $t$, then we get the PDE

\[ u_t + cu_x = 0 \]

where $c$ is any constant. We call this the \textbf{simple transport equation}. This is a constant coefficient first order PDE, which we know to have solutions given by $u(t,x) = f(x-ct)$ for any function $f$. 

Visually, this actually makes physical sense. For a solution $f(x-ct)$, we see that as $t$ increases, the function literally moves at a constant rate along the $x$-axis.

\subsection*{The Wave Equation}

We consider a flexible, elastic, and homogeneous string of length $\ell$ that moves up and down (transversely). Let $u(t,x)$ denote the height/displacement of the string. We then set

\[ c = \sqrt{\frac{T}{\rho}} \]

where $T$ is the tension in the string, and $\rho$ is the density of the string. Then we get a PDE given by

\[ u_{tt} - c^2u_{xx} = 0 \]

This is called the \textbf{wave equation}, and plays an important role in general relativity, fluid dynamics, and electromagnetism through Maxwell's Equations. 

There are also many variations of the wave equation that are of relevance:

\begin{enumerate}[label=(\roman*)]
    \item Air resistance: given a constant $r > 0$ representing friction (or air resistance), we get

    \[ u_{tt} - c^2u_{xx} + ru_t = 0 \]

    \item Elastic force in the transverse direction: suppose there was a force pushing back against the transverse movement of the string, given by $k > 0$. We get

    \[ u_{tt} - c^2u_{xx} + ku = 0 \]

    This is called the \textbf{Klein-Gordon Equation}.

    \item External forcing: Suppose $f(t,x)$ represents an external force in the system. Then we get

    \[ u_{tt} - c^2u_{xx} = f(t,x) \]
\end{enumerate}

The higher dimensional versions of these equations also show up often; the 3D version is seen in Maxwell's Equations. Furthermore, we can also consider $c$ to be a function $c(t,x,y,z)$, which shows up in fluid dynamics and general relativity.

\subsection*{The Diffusion/Heat Equation}

Suppose we have a chemical substance diffusing in a fluid, or say we are heating up an metal object. Given a function $u(t,x)$ representing the spread of the substance or heat with respect to position and time, then we have

\[ u_t - ku_{xx} = 0 \]

where $k > 0$ is a constant related to the properties of the fluid or material. This is called the \textbf{heat/diffusion equation}; we will just call it the heat equation. The higher dimensional forms are also relevant and play an interesting role in statistics through Brownian Motion.

\subsection*{The Laplace Equation}

Suppose $u$ solves the heat or wave equation and is now settled and in a ``stationary" state. We thus set time, and its related derivatives, to 0, and we get

\[ u_{xx} = 0 \]

The higher dimensional forms also exist and are extremely relevant to complex analysis. This is called the \textbf{Laplace Equation}, and the solutions to this equation are called \textbf{harmonic functions}.

\section{Initial and Boundary Conditions}

Solutions to PDEs are generally a large class of functions; as we have seen previously, the set of solutions is often an infinite dimensional vector space. To get a unique solution to a PDE, we often require additional conditions to be present. There are two common ways this can be achieved, which we discuss now.

\subsection*{Initial Conditions}

If we have, say, a time variable $t$, then we may specify the solution at a specific time $t_0$. Oftentimes we use $t_0 = 0$, but this $t_0$ may be any value. 

\begin{example}
    Consider the heat equation $u_t = ku_{xx}$ with the initial condition that 

    \[ u(t_0,x) = \phi(x) \]

    where $\phi$ is given. In this case, $\phi$ represents the distribution of heat at time $t_0$. If $t_0 = 0$, then this is the initial heat distribution.
\end{example}

\begin{example}
    Consider the wave equation $u_{tt} - c^2u_{xx} = 0$ with the initial conditions that 

    \[ u(t_0,x) = \phi(x) \]
    \[ \pardif{u}{t}(t_0,x) = \psi(x) \]

    where $\phi, \psi$ are given. Note that we must specify the derivative with respect to $t$ at $t_0$ because we use the second partial derivative of $t$ in the wave equation. In this case, $\phi$ represents the height of the string at position $x$, while $\psi$ represents its momentum. 
\end{example}

\subsection*{Boundary Conditions}

Suppose we have an unknown function $u(t,x_1, \ldots, x_{n-1}): D \to \mathbb{R}$ or $\mathbb{C}$, where $D \subset \mathbb{R}^n$. As an explicit example, one can think of a 1-dimensional pipe with the heat equation, or a 2-dimensional disk (like that of the surface of a drum) with the wave equation. We will often require that some condition holds along the \textit{boundary} of these domains, such as the ends of the pipe or the rim of the drum's surface. These conditions are called boundary conditions, and there are 3 standard ways we can define them:

\begin{enumerate}[label=(\roman*)]
    \item \textbf{Dirichlet Conditions} $u\big|_{\partial D} = \phi(x)$
    \item \textbf{Neumann Conditions} $\pardif{u}{n}\bigg|_{\partial D} = \psi(x)$
    \item \textbf{Robin Conditions} $\left(\pardif{u}{n}\bigg|_{\partial D} + \alpha u\right)\bigg|_{\partial D} = \chi(x)$
\end{enumerate}

where $\phi, \psi, \chi$ are given.

\begin{remark}
    The term $\pardif{u}{n}$ is the normal derivative, the directional derivative in the direction that is normal to the curve at the desired point. 
\end{remark}

If the given function $\phi,\psi,\chi$ is the zero function, then we call the condition a \textbf{homogeneous condition}, otherwise, it is an \textbf{inhomogeneous condition}.

\begin{example}
    Consider the problem of the vibrating string. If we include a Dirichlet condition, then this can be thought of as holding the ends of the string fixed while the rest of the string vibrates. Think of this like the strings of a guitar.

    If instead we include a Neumann condition, it can be thought of as tying the ends of the string to vertical poles and pulling the string tight, meaning only the ends of the string move freely. 
\end{example}

\begin{example}
    Consider the problem of heat distribution on a metal rod. Including a Dirichlet condition may be thought of as fixing the rod's temperature along the boundary. 

    Including a homogeneous Neumann condition may be thought of as insulating the rod, ensuring that no heat leaves the system. 
\end{example}

Note that boundary conditions can also exist at infinity. For instance, suppose $D = \mathbb{R}$. Then for the 1-dimensional heat equation, we may require that

\[ \lim_{x \to \pm\infty} u(t,x) = 0 \]

\section{Well-Proved Problems}

For a PDE to be representative of a physical problem; that is, a problem that can exist in the real world and can be done in a physics class, three conditions must hold:

\begin{enumerate}
    \item \textbf{Existence}: There is a solution to the PDE (under reasonable assumptions)
    \item \textbf{Uniqueness}: There is at most one solution to the PDE
    \item \textbf{Stability}: Small changes to the inputs must yield only small changes to the solution
\end{enumerate}

\section{Constant Coefficient 2nd Order PDE in 2 Variables}

These types of PDEs will form the basis of our study of PDEs throughout the course, as they form the general class of PDEs from which the simple transport, wave, heat, and Laplace equations arise from.

Such equations are of the form

\[ a_{11}u_{xx} + 2a_{12}u_{xy} + a_{22}u_{yy} + a_1u_x + a_2u_y + a_0u = 0 \]

where $a_{11}, a_{12}, a_{22}, a_1, a_2, a_0$ are real coefficients. Like in ODEs where we cared more about the terms with first derivatives compared to other terms, we now care about the terms with second derivatives more than lower order terms. 

Understanding the type of 2nd order PDE we're dealing with is of great importance and allows us to better understand how to solve them. There is a relatively simple algorithm for determining its type, and how to simplify the equation based on its type:

\begin{theorem}
    Given a constant coefficient 2nd order PDE in two variables, there exists a change of variables

    \[ x' = b_1x + b_2y \]
    \[ y' = b_3x  +b_4y \]

    where $b_1,b_2,b_3,b_4$ are real constants with

    \[ \det\begin{pmatrix}
        b_1 & b_2 \\ b_3 & b_4
    \end{pmatrix} \neq 0 \]

    such that

    \begin{enumerate}[label=(\roman*)]
        \item \textbf{Elliptic Case}: If $a_{12}^2 < a_{11}a_{22}$, then the PDE reduces to

        \[ u_{x'x'} + u_{y'y'} + \text{ lower order terms} = 0 \]

        \item \textbf{Hyperbolic Case}: If $a_{12}^2 > a_{11}a_{22}$, then the PDE reduces to 

        \[ u_{x'x'} - u_{y'y'} + \text{ lower order terms} = 0 \]

        \item \textbf{Parabolic Case}: If $a_{12}^2 = a_{11}a_{22}$, then the PDE reduces to

        \[ u_{x'x'} + \text{ lower order terms } = 0 \ \text{OR} \ \text{lower order terms} = 0 \]
    \end{enumerate}
\end{theorem}

Notice that the elliptic case is the 2D Laplace equation, while the hyperbolic case is the wave equation.

\begin{example}
    Consider $u_{xx} - 5u_{xy} = 0$. We have that $a_{11} = 1, a_{12} = \dfrac{5}{2}, a_{22} = 0$, hence

    \[ a_{12}^2 = \frac{25}{5} > 0 = a_{11}a_{22} \]

    so this is hyperbolic.
\end{example}

While such equations won't be the main focus of the course, it should be noted that it is possible to formulate this idea for second order PDEs with variable coefficients. Say we have

\[ yu_{xx} - 2u_{xy} + xu_{yy} = 0 \]

If we specify a point $(x,y)$, then we get constant coefficients and the theorem applies, meaning we can determine the PDE's type, \textit{at that point}. This means the PDE will change type depending on the point in the plane. 

In the above case, we have that $a_{12}^2 = 1$, while $a_{11}a_{22} = yx$. Thus, we get that

\[ 1 = yx \implies \text{ parabolic} \]
\[ 1 > yx \implies \text{ hyperbolic} \]
\[ 1 < yx \implies \text{ elliptic} \]

This can be seen in the graph below, where the red area represents when the PDE is elliptic, while the black represents where it is hyperbolic; the intersection is where it is parabolic:

\begin{center}
    \includegraphics[width=0.7\textwidth]{Week 2/PDE type graph.png}
\end{center}

Let's find the change of variables for the equation

\[ u_{xx} - 5u_{xy} = 0 \]

First consider 

\[ x^2 - 5xy \]

Completing the square yields

\begin{align*}
    x^2 - 5xy & = x^2 - 5xy + \left(\frac{5}{2}\right)^2y^2 - \left(\frac{5}{2}\right)^2y^2 \\
    & = \left(x - \frac{5}{2}y\right)^2 - \left(\frac{5}{2}y\right)^2
\end{align*}

Surprisingly, this idea also works for partial derivatives:

\begin{align*}
    u_{xx} - 5u_{yy} & = \pardif{}{x}\pardif{}{x}u - 5\pardif{}{x}\pardif{}{y}u \\
    & = \left(\pardif{}{x} - \frac{5}{2}\pardif{}{y}\right)^2u - \left(\frac{5}{2}\pardif{}{y}\right)^2u
\end{align*}

Now, we need $(x',y')$ such that

\[ \pardif{}{x'} = \pardif{}{x} - \frac{5}{2}\pardif{}{y} \quad \pardif{}{y'} = \frac{5}{2}\pardif{}{y} \]

We see that

\[ \pardif{}{x} = \pardif{}{x'} + \pardif{}{y'} \quad \pardif{}{y} = \frac{2}{5}\pardif{}{y'} \]

Recalling the definition of $x',y'$, we get by the Chain Rule that

\[ \pardif{}{x} = \pardif{}{x'}b_1 + \pardif{}{y'}b_3 \]
\[ \pardif{}{y} = \pardif{}{x'}b_2 + \pardif{}{y'}b_4 \]

so we have that $b_1 = 1, b_2 = 0, b_3 = 1, b_4 = \frac{2}{5}$. Thus,

\[ x' = x \]
\[ y' = x  +\frac{2}{5}y \]
\chapter{Week 3}

\section{Posets}

\begin{definition}
    A \textbf{poset}, or partially ordered set, is a set $S$ along with a binary relation $\leq$ satisfying the following:
    \begin{enumerate}[label=(\roman*)]
        \item For all $a \in S$, $a \leq a$ (Reflextive)
        \item For all $a,b,c \in S$, $(a \leq b) \wedge (b \leq c) \implies a \leq c$ (Transitive)
        \item For all $a,b \in S$, $(a \leq b) \wedge (b \leq a) \implies a = b$ (Antysymmetric)
    \end{enumerate}

    If the condition 

    \begin{enumerate}
        \item[(iv)] For all $a,b \in S$, $a \leq b$ or $b \leq a$
    \end{enumerate}

    is also satisfied, then we call it a \textbf{totally ordered set}.
\end{definition}

\begin{example}
    The following are all examples of posets and totally ordered sets
    \begin{enumerate}
        \item $(\R, \leq)$ is totally ordered 
        \item $\R^2$ with coordinate wise $\leq$ is a poset 
        \item $(\{a,b,c,\ldots,x,y,z\}, \leq)$ is a totally ordered set
        \item $(\{\text{words in the English language}\}, \text{lexical order})$
        \item $(\mathcal{P}(A), \subseteq)$
        \item $(\N \setminus \{0\}, |)$. 
    \end{enumerate}
\end{example}

For our purposes, the poset $(\mathcal{P}(X), \subseteq)$ is of relevance. 

\begin{definition}
    A \textbf{chain} in a poset $(S, \leq)$ is a subset $A \subseteq S$ such that for all $a,a' \in A$, either $a \leq a'$ or $a' \leq a$. 
\end{definition}

We can actually draw chains as graphs, as seen here:


The opposite of a chain is a subset where no two elements are comparable. These are also important.

\begin{definition}
    An \textbf{anti-chain} is a subset $A \subseteq S$ such that for all $a,a' \in A$, $a \nleq a'$ and $a' \nleq a$. 
\end{definition}

\section{Dilworth's Theorem}

Dilworth's Theorem is a result that pertains to the largest anti-chain in a poset. 

\begin{definition}
    For a poset $(S,\leq)$, a \textbf{chain cover} of $S$ is a set of chains $C_1,\ldots, C_m$ such that 

    \[\cup_i C_i = S\]

    In this case, we say the chain cover has size $m$. 
\end{definition}

\begin{theorem}[Dilworth's Theorem]
    The size of the largest anti-chain is equal to the smallest number of chains that cover the poset.
\end{theorem}

To prove this, we require some additional results:

\begin{prop}
    Given a poset $(S,\leq)$, for all anti-chains $A \subseteq S$, for all chain cover $C_1,\ldots, C_m$, we have that $|A| \leq m$. 
\end{prop}

\begin{proof}
    By definition of an anti-chain, at most one element of $A$ can be in each $C_i$. The claim then follows as there are $m$ chains. 
\end{proof}

Another important result needed for Dilworth is related to vertex covers.

\begin{definition}
    Given a graph $G = (V,E)$, a subset $U \subseteq V$ is a \textbf{vertex cover} of $G$ is for all $(u,v) \in E$, wither $u \in U$ or $v \in U$. 
\end{definition}

\begin{theorem}[K\"{o}nig's Theorem]
    In a bipartite graph $(L,R,E)$, the maximum size of a matching $M$ is equal to the minimum size of a vertex cover.
\end{theorem}

\begin{proof}
    Suppose the maximum mathcing $M$ has size $|M|$. It is evident that there can be no vertex cover $U$ with size smaller than $M$, since a cover must contain a vertex from each edge in $M$.

    To find a vertex cover of size $|M|$, we use the same method as in Theorem 2.9, except we now set the capacity function of all edges in the graph to $\infty$. 
    
    Now let $A$ be the cut of the vertex set with smallest cut value. Write 

    \[A = \{s\} \cup A_L \cup A_R\]

    Clearly there is no edges $(u,v)$ for which $u \in A_L$ and $v \in R\setminus A_R$. Thus,

    \[|M| = \operatorname{CutValue}(A) = |L \setminus A_L| + |A_R|\]

    From here, it is easy to see that $(L \setminus A_L) \cup A_R$ is a vertex cover of size $|M|$, as for any edge $(u,v)$, we either have $u \in L \setminus A_L$, or $v \in A_R$. 
\end{proof}

We are now prepared to prove Dilworth's Theorem:

\begin{proof}[Proof of Dilworth's Theorem]
    Suppose $|S| = n$, and let $S^-, S^+$ be two copies of $S$. We define a bipartite graph $G = (L,R,E)$ as 

    \[\begin{cases}
        L = S^- \\ R = S^+ \\ E = \{(x^-, y^+): x \leq y, x \neq y\}
    \end{cases}\]

    By K\"{o}nig's Theorem, there is a maximum matching $M$ and minimum vertex cover $U$, where $|M| = |U| = m$. We will show that there is a chain cover of $S$ with size $n - m$, and that there is an anti-chain of size $n - m$. 

    For the first claim, we consider the following algorithm: start with $n$ chains of the form $\{s\}$, one for each $s \in S$. Then, if $(x^-, y^+) \in M$, then we merge the chains containing $x$ and $y$. We repeat this process until we get a chain cover. Observe that because we have $m$ matchings, we would have to do $m$ mergings, thus creating a chain cover of size $n - m$, as desired. 

    For the second claim, we let $U = U^- \cup U^+$, such that $U^- \subseteq S^-$, $U^+ \subseteq S^+$. Define 

    \[A = \{x \in S : x^- \notin U^- \wedge x^+ \notin U^+\}\]

    we claim $A$ is an anti-chain. Suppose $x < y$. Note that $(x^-, y^+) \in E$. By the definition of our vertex cover, $x^- \in U$ or $y^+ \in U$. If $x^- \in U$, then $x \notin A$. Similarly, if $y^+ \in U$, then $y \notin A$. Thus, both $x,y$ cannot be in $A$, so $A$ is indeed an anti-chain. 

    Finally, we show that $|A| \geq n-m$. Indeed,

    \[|A| = |\{x \in S : x^- \notin U^- \wedge x^+ \notin U^+\}| \geq |S| - |U^-| - |U^+| = n-m\]
\end{proof}

\section{Sperner's Theorem \& LYM Inequality}

\subsection{Sperner's Theorem}

Let $X = [n] = \{1,\ldots,n\}$ and consider the poset $(\mathcal{P}(X), \subseteq)$. Sperner's Theorem is a result about the largest anti-chain in this poset:

\begin{theorem}[Sperner's Theorem]
    The longest anti-chain in this poset has size ${n \choose \left\lfloor\frac{n}{2}\right\rfloor}$
\end{theorem}

To prove this, we first need to define some new objects. Let $\sigma \in S_n$ be a uniformly random permutation. For $B \subseteq [n]$, we define an event 

\[E_B = \{\sigma(1), \sigma(2), \ldots, \sigma(|B|)\} = B\]

which means that the first $|B|$ elements of the permutation are equal to the elements of $B$, but not necessarily in the same order. 

\begin{example}
    $E_{\{1\}}$ is the event that $\sigma(1) = 1$, which has probability $\dfrac{1}{n}$. 

    $E_{\{1,3\}}$ is the event that either $\sigma(1) = 1$ and $\sigma(2) = 3$, or the other way around. The probability of this event is $\dfrac{2}{n(n-1)}$. 
\end{example}

In general,

\[\Pr(E_B) = \frac{1}{{n \choose |B|}} = \frac{|B|!}{n(n-1)\cdots (n-|B|+1)}\]

Now let $A$ be an anti-chain. Consider the events $\{E_B: B \in A\}$. As the $B$'s are not subsets of each other, the $E_B$'s are disjoint events:

\begin{prop}
    If $B,B'$ are incomparible, then $E_B, E_{B'}$ are disjoint events. 
\end{prop}

\begin{proof}
    If $\sigma \in E_B, E_{B'}$, then 

    \[\{\sigma(1), \sigma(2), \ldots, \sigma(|B|)\} = B\]
    \[\{\sigma(1), \sigma(2), \ldots, \sigma(|B'|)\} = B'\]

    so $B,B'$ are comparible. 
\end{proof}

With this in mind, we can now prove our theorem:

\begin{proof}[Proof of Sperner's Theorem]
    Let $A$ be an anti-chain and consider $\{E_B : B \in A\}$. As the events are disjoint, we have that 

    \[S = \sum_{B \in A} \frac{1}{{n \choose |B|}} \leq 1\]

    As ${n \choose \left\lfloor \frac{n}{2}\right\rfloor}$ is the largest binomial coefficient, we get that 

    \begin{align*}
        & \sum_{B \in A} \frac{1}{{n \choose \left\lfloor \frac{n}{2}\right\rfloor}} \leq S \leq 1 \\
        \implies & |A| \frac{1}{{n \choose \left\lfloor \frac{n}{2}\right\rfloor}} \leq 1 \\
        \implies & |A| \leq {n \choose \left\lfloor \frac{n}{2}\right\rfloor}
    \end{align*}
\end{proof}

\subsection{The LYM Inequality}

Another relevant result about posets of $\mathcal{P}([n])$ revolves around elements of an anti-chain of of a given size $k$. For an anti-chain $A$, we define 

\[A_k = A \cap {[n] \choose k}\]

where ${[n] \choose k}$ is the set of subsets of $[n]$ of size $k$. 

\begin{theorem}[Lubell-Yamamoto-Meshalkin Inequality]
    \[\sum_{k=1}^n \frac{|A_k|}{{n\choose k}} \leq 1\]
\end{theorem}

This means that the sum of the ratios of the elements of $A$ of size $k$, to all subsets of $[n]$ of size $k$, can never exceed 1. 

We will prove LYM using a stronger statement, that for any anti-chain $A$, 

\[\sum_{B \in A} \frac{1}{{n \choose |B|}} \leq 1\]

\begin{proof}
    let $\sigma \in S_n$ be chosen uniformly. Then let 

    \[C_\sigma : \varnothing \subset \{\sigma(1)\} \subset \{\sigma(1), \sigma(2)\} \subset \cdots \sigma \{\sigma(1), \ldots, \sigma(n)\} = [n]\]

    be the maximal chain built from $\sigma$. Now for each $B \subseteq [n]$, we define $E_B$ to be the event in which $B$ appears in this chain. In other words,

    \[E_B = \{B = \{\sigma(1), \ldots, \sigma(|B|)\}\}\]

    We now compute the probability of $E_B$. There are $n!$ permutations of $[n]$. Of these, $|B|!(n-|B|)!$ of them send the first $|B|$ entries to elements in $B$, and the remaining entries are elements of $[n] \setminus B$. Thus,

    \[\Pr(E_B) = \frac{|B|!(n-|B|)!}{n!} = \frac{1}{{n \choose |B|}}\]

    Now note that these events are pairwise disjoint. Indeed if $B \neq B'$ and $E_B, E_{B'}$ occur for the same permutation $\sigma$, then $B,B'$ would appear as initial segments of the chain $C_\sigma$. It follows that either $B \subset B'$ or $B' \subset B$, which contradicts that $A$ is an anti-chain. Thus, For any $B \in A$, we have that 

    \[1 \geq \Pr\left(\bigcup_{B \in A} E_B\right) = \sum_{B \in A}\Pr(E_B) = \sum_{B \in A} \frac{1}{{n\choose |B|}}\]
\end{proof}
\chapter{Week 4}

\section{A Magic Trick Using Matches}

To begin this lecture, we present a fun application of matchings that may prove useful at a party: some math magic. 

This trick requires two magicians, $M_1$ and $M_2$, and a volunteer. The volunteer will choose 5 cards at random from a deck of cards, and give them to $M_2$. $M_2$ will then read out exactly 4 cards of the cards they were given in some order. $M_1$, who never directly sees the cards, is then able to correctly identify the missing 5th card. How are they able to do this? 

Let's simplify this trick a little to get a better idea of what's going on. Suppose that the volunteer only picks up two cards, and also tosses a fair coin. $M_2$ will read out both cards, and $M_1$ just needs to guess which side the coin has landed on. This may seem impossible, but note that $M_2$ has seen the coin flip, and has had time to converse with $M_1$. In conversing ahead of time, they could come up with some order that $M_2$ could read out the cards in, so that $M_1$ knows the outcome of the coin flip. For example, if the coin lands on heads, then $M_2$ will read out the cards in ascending order by rank (or if they are the same rank, using CHaSeD order for suit), and if the coin lands on tails, they read them in descending order by rank (or reverse CHaSeD order). 

The key observation to make here is that the order of the input is a set (the subset of cards drawn and outcome of the coin flip), but the output given to $M_1$ is a sequence (a sequence of cards). The ordering of that sequence is what allows $M_2$ to encode information about the fifth card. 

To encode it, we will use a bipartite graph. Identify the deck with $[52]$, and let $G = (L,R,E)$ be bipartite, with 

\[L = {[52] \choose 5},\ \text{ the set of possible 5-cards received by $M_2$}\]
\[R = \{(x_1,x_2,x_3,x_4) \in [52]^4: x_i \neq x_j \ \forall i \neq j\},\ \text{ the ordered sequences of cards $M_2$ can read out}\]

and $E$ is defined so that $A \in L$ is joined to $b = (x_1,x_2,x_3,x_4) \in R$ if and only if $b \subset A$. Notice that for each $A \in L$, $\deg(A) = {5 \choose 4} \cdot 4! = 120$, and for each $b \in R$, $\deg(b) = 48$. We can use a perfect matching in $G$ to find an encoding for $M_2$ to read our the four cards so that $M_1$ knows exactly what the fifth card is. 

\begin{theorem}
    In the graph $G$ described above, there is a matching which saturates all vertices of $L$. Equivalently, there is an injective map $f: L \hookrightarrow R$ such that $f(A)$ is an ordered 4-tuple of cards from $A$ for all $A \in L$.
\end{theorem}

\begin{proof}
    We prove this via Hall's Theorem. Let $S \subset L$, $N(S)$ the set of neighbours, and $E_S, E_{N(S)}$ the corresponding edge sets. As every vertex of $S$ has degree 120,

    \[|E_S| = 120|S|\]

    and as every vertex of $N(S)$ has degree 48,

    \[|E_{N(S)}| = 48|N(S)|\]

    Now, $E_S \subseteq E_{N(S)}$, so 

    \[120|S| \leq 48|N(S)| \implies |N(S)| \geq \frac{120}{48}|S| \geq |S|\]

    as required. Hall's Theorem now applies.
\end{proof}

So when $M_2$ receives a hand $A \in L$, they read out the 4-card ordered tuple $b \in R$ that $A$ is matched to. $M_1$ recognizes this, identifies what $A$ is using the matching, and reads out the fifth card!

Note that this only guarantees that a matching exists, not how to create it. There is some leeway in how you do this, and there are many methods, such as mnemonics. 

\section{The Erd\"{o}s-Ko-Rado (EKR) Theorem}

\subsection{Intersecting Families}

\begin{definition}
    A family of sets $\mathcal{F}$ is \textbf{intersecting} if for all $A,B \in \mathcal{F}$, $A \cap B \neq \varnothing$.
\end{definition}

An intersecting family of sets is simply a family of pairwise disjoint sets. We ask the following question: If $\mathcal{F} \subset \mathcal{P}([n])$ is intersecting, then how large can $|\mathcal{F}|$ be?

\begin{example}
    Consider the family of sets

    \[\{A \in \mathcal{P}([n]) : \{1,2\} \subseteq A, \{2,3\} \subseteq A, \text{ or } \{1,3\} \subseteq A\}\]

    This is an intersecting family of sets, and its size is $2^{n-1}$
\end{example} 

\begin{theorem}
    If $\mathcal{F} \subset \mathcal{P}([n])$ is intersecting, then $|\mathcal{F}| \leq 2^{n-1}$. 
\end{theorem}

\begin{proof}
    We partition $\mathcal{P}([n])$ into pairs $(A, A^c)$, of which there are $2^{n-1}$. Notice that any intersecting family $\mathcal{F}$ must contain at most one set from each pair, as if it contained both sets in a pair their intersection would be empty. Thus, $|\mathcal{F}| \leq 2^{n-1}$. 
\end{proof}

Let's explore a slightly more complicated question: If $\mathcal{F} \subseteq {[n] \choose k}$, how large can $|\mathcal{F}|$ be? Notice that if $k > n/2$, then ${[n] \choose k}$ is already an intersecting family, so let's assume $k < n/2$. Consider the family 

\[\{A \in {[n] \choose k}: 1 \in A\}\]

Then this is an intersecting family of size ${n-1 \choose k-1}$, and this grows at a rate of $\Theta(n^{k-1})$. Another intersecting example is the family

\[\{A \in {[n] \choose k} : A \cap \{1,2,3\} = \{1,2\} \text{ or } \{1,3\} \text{ or } \{1,3\}\}\]

This is also intersecting, and is of size $3{n-3 \choose k-2} \ll {n-1 \choose k-1}$, and growing at a rate of $\Theta(n^{k-2})$ when $k \ll n$. The size we found in our first example is actually the maximum size for such a family, an important result known as the Erd\"{o}s-Ko-Rado (EKR) Theorem.

\begin{theorem}[Erd\"{o}s-Ko-Rado Theorem]
    If $k < \frac{n}{2}$ and $\mathcal{F} \subseteq {[n] \choose k}$ is intersecting, then 

    \[|\mathcal{F} \leq {n-1 \choose k-1}\]
\end{theorem}

\subsection{Necklaces}

To prove EKR, we will use a probabilistic method that involves a special object called a \textbf{necklace}.

\begin{definition}
    Let $\Sigma$ be an alphabet. Two sequences $(a_1,\ldots,a_n), (b_1,\ldots,b_n)$ of elements of $\Sigma$ are equivelent if there is an $i \in [n]$ such that 

    \[a_1 = b_i, a_2 = b_{i+1}, \ldots a_{n-i+1} = b_n, a_{n-i+2} = b_1, \ldots, a_n = b_{i-1}\]

    in other words, one can be made into the other by a shift of indices. An equivalence class of sequences of length $n$ over $\Sigma$ is called a \textbf{necklace}, and is denoted $[a_1,\ldots,a_n]$. 
\end{definition}

\begin{proof}[Proof of EKR]
    Let $\sigma \in S_n$ be chosen uniformly, and let

    \[N = [\sigma(1), \ldots, \sigma(n)]\]

    be the necklace corresponding to $\sigma$. For each $\mathcal{F} \subseteq {[n] \choose k}$, let $\mathcal{F}_N$ be the set of all $A \in \mathcal{F}$ such that $A$ is also a contiguous subsequence of $N$. We then define the random variable $X = |\mathcal{F}_N|$. For each $A \in \mathcal{F}$, we define an indicator function 

    \[X_A = \begin{cases}
        1 & A \in \mathcal{F}_N \\ 0 & A \notin \mathcal{F}_N
    \end{cases}\]

    If follows that $X = \sum_{A \in \mathcal{F}} X_A$. What is the expectation of $X$? Notice that there are $n$ cyclic starting positions in $N$ and each position gives a uniformly randon size $k$-subset of $N$. Thus, 

    \[\mathbb{E}[X_A] = \Pr[A \in \mathcal{F}_N] = \frac{n}{{n\choose k}}\]

    Thus, 

    \[\mathbb{E}[X] = \sum_{A \in \mathcal{F}}\mathbb{E}[X_A] = \sum_{A \in \mathcal{F}}\frac{n}{{n\choose k}} = |\mathcal{F}| \cdot \frac{n}{{n \choose k}}\]

    We now make a key observation: As $k < \frac{n}{2}$, then $X \leq k$. Indeed, notice that two contiguous segments of $N$ must intersect to both be in $\mathcal{F}$. If we had $k + 1$ contigous segments of length $k$, we remove a point not in any of these segments (which is possible as $k < \frac{n}{2})$. We get a line with $k+1$ distinct length $k$ intervals on this line. The left and rightmost intervals are disjoint, as required. Thus as $X \leq k$, so too is its expectation. We conclude that 

    \[|\mathcal{F}| \cdot \frac{n}{{n\choose k}} \leq k\]

    \begin{align*}
        \implies |\mathcal{F}|& \leq \frac{{n \choose k} \cdot k}{n} \\
        & = \frac{n! \cdot k}{k!(n-k)!n} \\
        & = \frac{(n-1)!}{(k-1)! (n-k)!} \\
        & = {n-1 \choose k-1}
    \end{align*}
\end{proof}

\section{Fermat's Little Theorem}

Recall this basic theorem of number theory:

\begin{theorem}[Fermat's Little Theorem]
    Let $p$ be prime. Then for any integer $a$,

    \[a^n \equiv a \mod{n}\]

    Equivalently, if $n \nmid a$, then 

    \[a^{n-1} \equiv 1 \mod{n}\]
\end{theorem}

We can actually prove this theorem using necklaces. Consider necklaces over $\Sigma = [a]$ of length $n$. For now we will let $a = 2$.

\begin{example}
    If $a = 2$, then 

    \begin{itemize}
        \item If $n = 2$, there are 3 unique necklaces: $[0,0],[0,1],[1,1]$ 
        \item If $n = 3$, there are 4 unique necklaces: $[0,0,0,0],[0,0,1],[0,1,1],[1,1,1]$
        \item If $n = 4$, there are 6 unique necklaces: $[0,0,0,0],[0,0,0,1],[0,0,1,1],[0,1,0,1],[0,1,1,1],[1,1,1,1]$. 
    \end{itemize}

    In general, there are $2^n$ strings of length $n$, and each necklace is equivalent to $n$ of them by rotation.
\end{example}

For a string $x$, we will define $R^ix$ to be that string rotated/shifted by $i$ places; we let $\Z_n$ act on the string by shifting. We let 

\[\operatorname{Period}(x) : \{i \in \mathbb{Z}_n : R^ix = x\}\]

be the values of $i$ that fix $x$. This is a group, a subgroup of $\Z_n$. We have that 

\[|\mathcal{O}(x)| = \frac{n}{|\operatorname{Period}(x)|}\]

If we let $n$ be prime, then the only way for $x$ to be fixed is if all its elements are the same. Thus,

\[|\operatorname{Period}(x)| = \begin{cases}
    n & x \in \{(0,0,\ldots,0),(1,1,\ldots,1)\} \\
    1 & \text{ otherwise}
\end{cases}\]

and so we have that the orbit is $1$ in the first case, and $n$ in the second case. Partitioning $\Sigma^n$ into its orbits (which we know to be disjoint), we get 2 orbits of size 1 (one for $\{0,0,\ldots,0\}$, and one for $\{1,1,\ldots,1\}$), and some number of orbits of size $n$, which we denote by $\alpha$. It follows that 

\[|\{0,1\}^n = 2^n = 1 + 1 + n\alpha \implies \alpha = \frac{2^n - 2}{n}\]

Therefore, the total number of necklaces of $\Sigma$ of length $n$ is 

\[\alpha + 2 = \frac{2^n - 2}{n} + 2\]

because we have a one-to-one correspondence between a necklace and its orbit. The fraction $\alpha$ must be an integer, so we have that 

\[2^n - 2 \equiv 0 \mod{n} \iff 2^n \equiv 2 \mod{n}\]

which completes the proof of Fermat's Little Theorem for $a = 2$. This generalizes for any $a$.

We can further generalize this for composite values of $n$. Suppose that $n = pq$ for primes $p,q$. We determine the number of necklaces of size $n$ over $[a]$. The orbit sizes of the action of $\Z_n$ on $[a]^n$ must divide $n$, meaning they are either $1,p,q$, or $pq = n$. Strings with orbit size $pq$ are the constant strings, and there are $a$ of them. 

For orbits of size $q$, if $x \in [a]^n$ is in this orbit, then 

\[|\operatorname{Period}(x)| = n/q = p \iff \operatorname{Period}(x) = \{0,q,2q,\ldots,(p-1)q\}\]

because $\operatorname{Period}(x)$ is a subgroup of $\Z_n$ of size $p$. $x$ has this period if and only if the first $q$ elements repeat themselves. Such elements may be arbitrarily chosen from $[a]$ in $a^q$ ways. In $a$ of these choices, all elements of the string are the same, so the size of the period is 1. Thus, the number of orbits of size $q$ is $a^1 - a$. Similarly, there are $a^p - a$ orbits of size $p$. Continuing with the process done previously gives a generalization of FLT for these values of $n$. 
\chapter{Week 5}

\section{Counting Necklackes Using Burnside's Lemma}

We continue our study of necklaces by deriving a way to count the number of necklaces over $[a]$ of length $n$ using Burnside's Lemma.

\subsection{Burnside's Lemma}

Burnside's Lemma is a theorem in abstract algebra, and describes a formula for the number of orbits of a given group action in terms of the number of fixed points. For a group $G$ acting on a set $S$, and $g \in G$, we let 

\[\operatorname{Fix}(G) = |\{s \in S : gs = s\}|\]

Be the number of elements fixed by $g$. 

\begin{theorem}[Burnside's Lemma]
    The number of orbits of the given group action is 

    \[\frac{1}{|G|} \sum_{g \in G} \operatorname{Fix}(g)\]
\end{theorem}

\newpage

\begin{proof}
    We have that 

    \begin{align*}
        \sum_{g \in G} \operatorname{Fix}(g) & = \sum_{g \in G}\sum_{s \in S} \mathrm{1}_{gs = s} \\
        & = \sum_{s \in S}\sum_{g \in G} \mathrm{1}_{gs = s} \\
        & = \sum_{s \in S} |\operatorname{Stab}_G(s)| \\
        & = \sum_{s \in S} \frac{|G|}{|\mathcal{O}(s)|} \\
        & = |G| \sum_{s \in S} \frac{1}{|\mathcal{O}(s)|} \\
        & = |G| \sum_{\text{orbits $\mathcal{O}$}} \sum_{s \in \mathcal{O}} \frac{1}{|\mathcal{O}(s)|} \\ 
        & = |G| \sum_{\text{orbits $\mathcal{O}$}} 1 \\
        & = |G| \cdot \text{number of orbits}
    \end{align*}

    and rearranging completes the proof.
\end{proof}

\subsection{Counting Necklaces and FLT}

Now, consider necklaces over $[a]$ of length $n$. We let $\Z_n$ act on $[a]^n$ by rotation/shifting by $i$. Notice that each orbit is a necklace. By Burnside, we have that the number of such orbits is 

\[\frac{1}{|G|} \sum_{g \in G} \operatorname{Fix}(g)\]

where $\operatorname{Fix}(i)$ is the number of strings that are $i$-periodic (shifting by $i$ does not affect the string). In particular, it is the number of strings that are $\gcd(1,n)$-periodic. This follows from B\'{e}zout's Theorem modulo $n$. Thus, 

\begin{align*}
    \text{number of necklaces} & = \frac{1}{n}\sum_{i=0}^{n-1} a^{\gcd(i,n)} \\
    & = \frac{1}{n}\sum_{d | n} (\text{number of $i$ such that $gcd(i,n) = 1$}) a^d \\
    & = \frac{1}{n} \sum_{d | n} \phi(n/d)a^d
\end{align*}

where $\phi$ is the Euler Totient Function. This also gives us a generalization of FLT: as this value is an integer, we get that 

\[n \bigg| \sum_{d | n}\phi(n/d)a^d\]

\section{Catalan Numbers and The Ballot Theorem}

Consider an $n \times n$ grid. Starting at $(0,0)$, we may move up or to the right by 1. How many ways can we end up at $(n,n)$? Given that we have to move up $n$ times, and move right $n$ times, giving us a total of $2n$ moves, we get that there are 

\[{2n \choose n} \approx \Theta\left(\frac{1}{\sqrt{n}}2^{2n}\right)\]

such ways. 

Now let's restrict our paths so that they cannot go above the diagonal $y = x$. How many paths are there now? The answer is approximately $1/n$ of all paths. In particular, we get 

\[\frac{1}{n+1}{2n \choose n}\]

This value is called the \textbf{$n$-th Catalan Number}, and they show up in a wide variety of combinatorial problems. 

To derive this value, we first need to consider a special theorem that, once again, involves necklaces. Consider a necklaces of beads, each labelled either $+1$ or $-1$. We say a bead is \textbf{special} if, starting at that bead, the partial sums in the clockwise direction are positive. 

\begin{example}
    Suppose we have the necklace

    \[[+1, -1, +1, +1, -1, +1, +1, +1, -1, -1, +1, -1, +1]\]

    Then the first $+1$ is not special, since the 2nd partial sum is 0, 2hile the second $+1$ is special. 
\end{example}

Suppose I know that the necklace has a certain number of beads labelled $+1$, and a certain number labelled $-1$. How can I arrange them to maximize the number of special beads?

\begin{theorem}[The Ballot Theorem]
    Given $a$ beads labelled $+1$, and $b$ labelled $-1$, the maximum number of special beads is $\max(a-b,0)$. 
\end{theorem}

\begin{proof}
    We can always find two adjacent beads labelled $+1$ and $-1$ (assuming $a,b \neq 0$). Delete these beads, and repeat this process until there are either all positive beads or all negative beads. If there are all positive beads, then there are $a-b$ of them, and all are special. If all are negative, then none are special. 
\end{proof}

Using this theorem, we prove our above claim about paths which don't cross the diagonal: Consider necklaces with $(n+1)$ beads labelled $+1$ and $n$ labelled $-1$. Each necklace has $2n+1$ representations as a string in $\{+1, -1\}^{2n+1}$ (by rotation) with exactly $n+1$ beads labelled as $+1$'s. By the Ballot Theorem, exactly one out of each string representation has all partial sums positive. To construct a string with the desired number of beads, we just consider choose whwere the $n+1$ positive beads will go in the $2n+1$ positions. Thus, the number of desired walks is 

\[\frac{1}{2n+1}{2n+1 \choose n+1} = \frac{1}{n+1}{2n \choose n}\]

\section{Probabilistic Results}

We now move to discussing some results in probability. These will become useful in discussing some results in graph theory later on. 

\subsection{Markov's Inequality}

We let $X$ be a nonnegative random variable. For our purposes, we assume that it is discrete. 

\begin{theorem}[Markov's Inequality]
    \[\Pr[X \geq t] \leq \frac{\mathbb{E}[X]}{t}\]
\end{theorem}

\begin{proof}
    \begin{align*}
        \mathbb{E}[X] & = \sum_a \Pr[X = a] \cdot a \\
        & = \sum_{a \geq t} \Pr[X = a] \cdot a + \sum_{0 \leq a \leq t} \Pr[X = a] \cdot a \\
        & \geq t \cdot \Pr[X \geq a]
    \end{align*}
\end{proof}

\begin{example}
    Let $X_1,\ldots,X_n$ be independent coin flips, with $\Pr[X_i = 1] = 0.1, \Pr[X_i = 0] = 0.9$, and let $X = \sum X_i$. Then by Markov's Inequality

    \[\Pr[X > 0.2n] \leq \frac{1}{2}\]
\end{example}

\subsection{Chebyshev's Inequality}

Let $\mathbb{E}[X] = \mu$, and let $\sigma = \operatorname{Var}(X) = \mathbb{E}[(X-\mu)^2]$. If we apply Markov to $(X - \mu)^2$, then 

\[\Pr[(X - \mu)^2 \geq t^2] \leq \frac{\mathbb{E}[(X-\mu)^2]}{t^2} = \frac{\sigma}{t^2}\]

More explicitely, 

\begin{theorem}[Chebyshev's Inequality]
    \[\Pr[|X - \mu| \geq t] \leq \frac{\sigma}{t^2}\]
\end{theorem}

\begin{example}
    Using the example above, with $X = \sum X_i$, we have that 

    \[\mathbb{E}[X] = \sum \mathbb{E}[X_i] = 0.1n\]

    \begin{align*}
        \sigma & = \mathbb{E}[(X-\mu)^2] \\
               & = \mathbb{E}[(\sum X_i - 0.1n)^2] \\
               & = \mathbb{E}[(\sum X_i - 0.1)^2] \\
               & = \mathbb{E}[\sum_i \sum_j(X_i - 0.1)(X_j - 0.1)] \\
               & = \sum_i \sum_j \mathbb{E}[(X_i - 0.1)(X_j - 0.1)] \\
               & = \sum_i \mathbb{E}[(X_i - 0.1)^2] + \sum_{i \neq j}\mathbb{E}[X_i -0.1]\mathbb{E}[X_j - 0.1] \\
               & \leq n
    \end{align*}

    Thus, we have that $\Pr[|X - 0.1| > t] \leq \frac{n}{t^2}$, and 

    \[\Pr[X > 0.2n] \leq \Pr[|X - 0.1n| > 0.1n] \leq \frac{n}{(0.1n)^2} = O\left(\frac{1}{n}\right)\]
\end{example}
\chapter{Week 6}

\section{The Wave Equation, with a Source}

We now consider the wave equation with a source, which is the problem

\[ \begin{cases}
    u(x,t): \R \times (0,\infty) \to \R \\
    u_{tt} - c^2u_{xx} = f(x,t) \\
    u(x,0) = \phi(x) \quad u_t(x,0) = \psi(x)
\end{cases} \]

We claim that the unique solution to this problem is given by 

\[ u(x,t) = \frac{1}{2}[\phi(x+ct) + \phi(x-ct)] + \frac{1}{2c}\int_{x-ct}^{x+ct}\psi(s) \ ds + \frac{1}{2c}\iint_{\Delta(x,t)}f(y,s) \ dy \ ds \]

where $\Delta(x,t)$ is the path light cone from before. We may assume that $\phi, \psi = 0$ by the same logic as before. While there are many solving methods, like using modified versions of Duhamel's Principle, we will exploit the wave equations relatively simple structure to get a solution. 

We will use characteristic coordinates. We define

\[ \xi = x + ct, \quad \eta = x -ct \]

\begin{align*}
    \pardif{}{x} & = \pardif{}{\xi}\pardif{\xi}{x} + \pardif{}{\eta}\pardif{\eta}{x} = \pardif{}{\xi} + \pardif{}{\eta} \\
    \pardif{}{t} & = \pardif{}{\xi}\pardif{\xi}{t} + \pardif{}{\eta}\pardif{\eta}{t} = c\pardif{}{\xi} - c\pardif{}{\eta}
\end{align*}

so

\[ x = \frac{1}{2}(\xi + \eta) \quad t = \frac{1}{2c}(\xi - \eta) \]

and we may rewrite our PDE as

\begin{align*}
    u_{tt} - c^2u_{xx} = f(x,t) & \iff \left(c\pardif{}{\xi} - c\pardif{}{\eta}\right)^2u - c^2\left(\pardif{}{\xi} + \pardif{}{\eta}\right)^2u = f \\
    & \iff \left(c^2\pardif{}{\xi}^2 - 2c^2\frac{\partial^2}{\partial \xi \partial \eta } + c^2 \pardif{}{\eta}^2 - c^2 \pardif{}{\xi}^2 - 2c^2 \frac{\partial^2}{\partial \xi \partial \eta} - c^2 \pardif{}{\eta}^2\right)u = f(x,t) \\
    & \iff -4c^2 \frac{\partial^2}{\partial \xi \partial \eta}u = f
\end{align*}

Thus, our solution in $(\xi, \eta)$ will look something like

\[ u(\xi, \eta) = -\frac{1}{4c^2}\int_?^\xi\int_?^\eta f \ d\xi \ d\eta + \cdots  \]

Let's now consider the path light cone in these coordinates, which is shown below:

\begin{center}
    \includegraphics[width=0.5\textwidth]{Week 6/Characteristic Triangle.png}
\end{center}

The left side is when $\eta$ is constant, while the right side is when $\xi$ is constant. We have that $u(x,0) = u_t(x,0) = 0$ along the line $\{t = 0\} \iff \{\xi = \eta\}$. Thus, $u|_{(\xi,\xi)} = 0$, and so do $u_\xi|_{(\xi,\xi)}, u_\eta|_{(\xi,\xi)}$. 

We fix $(x_0,t_0)$ where we originally wanted to evaluate $u(x,t)$. Set $\eta_0 = x_0 - ct_0, \xi_0 = x_0+ct_0$. Then by the FTC, we get that

\begin{align*}
    u(\xi_0,\eta_0) & = \int_{\eta_0}^{\xi_0} u_\xi(\xi,\eta_0) \ d\xi  \\
    & = \int_{\eta_0}^{\xi_0}\int_{\xi}^{\eta_0} u_{\xi\eta}(\xi,\eta) \ d\eta \ d\xi \\
    & = -\frac{1}{4c^2}\int_{\eta_0}^{\xi_0}\int_\xi^{\eta_0} f \ d\eta \ d\xi \tag{by the PDE}
\end{align*}

Note that $\xi \in (\eta_0, \xi_0)$, with $\xi_0 > \eta_0$, meaning that $\xi > \eta_0$. Hence we flip the integral to give us

\[ u(\xi_0, \eta_0) = \frac{1}{4c^2}\int_{\eta_0}^{\xi_0}\int_{\eta_0}^\xi f \ d\eta \ d\xi = \frac{1}{4c^2}\iint_? f \cdot A \ dx \ dt \]

Where the last equal sign is the change of variables back to $(x,t)$. The $A$ is the determinant of the transformation, given by 

\begin{align*}
    A & = \left|\det\begin{pmatrix}
        \pardif{\xi}{x} & \pardif{\xi}{t} \\
        \pardif{\eta}{x} & \pardif{\eta}{t}
    \end{pmatrix}\right| \\
    & = \left|\det\begin{pmatrix}
        1 & c \\ 1 & -c
    \end{pmatrix}\right| \\
    & = |-c - c| \\
    & = 2c
\end{align*}

so we get

\[ \frac{1}{2c} \iint_? f \ dx \ dt \]

What are we integrating? The triangle from before! The left side is when $\eta = \eta_0$, and the right side is when $\xi = \xi_0$. For the outside integral, we integrate over $\xi$, which goes from $\eta_0$ to $\xi_0$. For the inside integral, we integrate over $\eta$, which goes from $\eta_0$ to $\xi$. 

In $(x,t)$ coordinates, our triangle goes from $0$ to $t$ on the $t$-axis. For $x$, the left side of the triangle is given by the line $y - cs = x - ct$ and the right side is given by $y + cs = x + ct$. Solving for $x$ means we are integrating from $x - c(t-s)$ to $x + c(t-s)$. Thus, our integral becomes 

\[ \int_0^t\int_{x-c(t-s)}^{x-c(t-s)} f(y,s) \ dy \ ds \]

\begin{example}
    Solve

    \[ \begin{cases}
        u_{tt} - c^2u_{xx} = \cos(x) \\
        u(x,0) = \sin(x) \quad u_t(x,0) = 1 + x
    \end{cases} \]

    Using the formula we've just derived, we get that

    \[ u(x,t) = \frac{1}{2}[\sin(x+ct) + \sin(x-ct)] + \frac{1}{2c}\int_{x-ct}^{x+ct} (1 + s) \ ds + \frac{1}{2c}\int_0^t\int_{x-c(t-s)}^{x+c(t-s)}\cos(y) \ dy \ ds  \]
\end{example}

Like the heat equation, we can use this formula to solve for inhomogeneous boundary conditions. Reflection tricks will allow us to handle source terms on the half-line and homogeneous boundary conditions, and you can then go on to solve even more general problems. 

\begin{remark}
    Everything up to this point will appear on the first term test. 
\end{remark}

\section{Working on a Finite Interval}

We now seek to solve both the wave and heat equations where the spatial variable $x$ is defined on a finite interval $(0, \ell)$. Doing this requires us to introduce a new solving technique, and make some assumptions about our initial conditions that will be made rigorous in a later section.

\subsection{The Wave Equation on a Finite Interval with Dirichlet B.C}

We now wish to solve the wave equation on a finite interval. We will consider the equation with Dirichlet boundary conditions:

\[ \begin{cases}
    u_{tt} - c^2u_{xx} = 0 & 0 < x < \ell, \quad t > 0 \\
    u(0,t) = u(\ell,t) = 0 \\
    u(x,0) = \phi(x) \quad u_t(x,0) = \psi(x)
\end{cases} \]

The fact that $x$ is defined on a finite interval is going to make things more complicated, and thus we require a new solving method: \textbf{separation of variables}.

To begin, let's make an educated guess (commonly called an \textit{Ansatz}) about the shape of our solution. Suppose that

\[ u(x,t) = X(x)T(t) \]

Can we find a solution that looks like this? If so, then we have that

\[ u_{tt} = X(x)T''(t) \]
\[ u_{xx} = X''(x)T(t) \]
\[ \implies X(x)T''(t) - c^2X''(x)T(t) = 0 \]

In this case, we can push all the $x$ terms on one side and all $t$ terms on the other side. We get

\[ X(x)T''(t) = c^2X''(t)T(t) \]

and assuming $X,T$ don't vanish, 

\[ -\frac{T''(t)}{c^2T(t)} = -\frac{X''(x)}{X(x)} \]

\begin{remark}
    The minus sign is convention and will make the next part easier.
\end{remark}

So we have that a function of $t$ is equal to a function of $x$. Clearly the only way this is possible is when these functions are both equal to a \textit{constant} value.

\[ -\frac{T''(t)}{c^2T(t)} = -\frac{X''(x)}{X(x)} = \lambda \in \R \]

From this, we get two equations:

\begin{equation*}
    X''(x) + \lambda X(x) = 0
\end{equation*}
\begin{equation*}
    T''(t) + c^2\lambda T(t) = 0
\end{equation*}

Both of these are ODEs that we already know how to solve. The only thing we need to do now is check that they solve the boundary conditions from before. To do this, we take cases:

\begin{enumerate}
    \item If $\lambda = 0$, then $X''(x) = 0$, and so

    \[ X(x) = Ax + B \]

    for some constants $A,B$. Plugging in the boundary conditions, we get that

    \[ 0 = X(0) = B \implies B = 0 \]
    \[ 0 = X(\ell) = A\ell + B \implies A \ell = 0 \implies A = 0 \]

    But then $X(x) = 0$, which results in the degenerative case that $u = 0$. We thus reject this case.

    \item If $\lambda < 0$, then

    \[ X(x) = Ae^{-\sqrt{-\lambda}x} + Be^{\sqrt{-\lambda}x} \]

    for some constants $A,B$. Plugging in the boundary conditions, we get that

    \[ 0 = X(0) = A + B \implies A = -B \]
    \begin{align*}
        0 = X(\ell) & = Ae^{-\sqrt{-\lambda}\ell} + Be^{\sqrt{-\lambda}\ell} \\
        & = -Be^{-\sqrt{-\lambda}\ell} + Be^{\sqrt{-\lambda}\ell} \\
        & = -B(e^{-\sqrt{-\lambda}\ell} - e^{\sqrt{-\lambda}\ell}
    \end{align*}

    We assume $B \neq 0$ (otherwise we get the degenerative case), and so we get that

    \begin{align*}
        e^{-\sqrt{-\lambda}\ell} - e^{\sqrt{-\lambda}\ell} = 0 & \implies -\sqrt{-\lambda} = \sqrt{-\lambda} \\
        & \implies \lambda = 0
    \end{align*}

    a contradiction. We again reject this case.

    \item If $\lambda > 0$, then

    \[ X(x) = A\cos(\sqrt{\lambda}x) + B\sin(\sqrt{\lambda}x) \]

    for some constants $A,B$. Plugging in the boundary conditions, we get that

    \[ 0 = X(0) = A \implies A = 0 \]
    \[ 0 = X(\ell) = B\sin(\sqrt{\lambda}\ell) \]

    We again take $B \neq 0$, and so we have that

    \begin{align*}
        \sin(\sqrt{\lambda}\ell) = 0 & \implies \sqrt{\lambda}\ell \ \text{ is a multiple of $\pi$}. \\
        & \implies \sqrt{\lambda}\ell = n\pi \\
        & \implies \lambda\ell^2 = (n\pi)^2 \\
        & \implies \lambda = \left(\frac{n\pi}{\ell}\right)^2 \tag{$n = 1, 2 ,3, \ldots$}
    \end{align*}
\end{enumerate}

We thus attain a solution when $\lambda > 0$. We denote

\[ \lambda_n = \left(\frac{n\pi}{\ell}\right)^2 \]
\[ X_n(x) = \sin\left(\frac{n\pi x}{\ell}\right) \]

The $\lambda_n$'s are called \textbf{eigenvalues}, while the $X_n(x)$'s are called \textbf{eigenfunctions}, which are related to the concept of eigenvectors (we will explain this later). It should also be noted that, without loss of generality, we're assume the constant is 1 since we can always absorb it into $T_n(t)$. 

Now to deal with $T$. First fixing $n$, we get

\[ T_n''(t) + c^2\left(\frac{n\pi}{\ell}\right)^2T_n(t) = 0 \]

This is the same ODE that we solved for when $\lambda > 0$. Thus,

\[ T_n(t) = A_n\cos\left(\frac{n\pi c}{\ell}t\right) + B_n\sin\left(\frac{n\pi c}{\ell}t\right) \]

Combining, we get

\[ u_n(x,t) = T_n(t)X_n(x) = \left(A_n\cos\left(\frac{n\pi c}{\ell}t\right) + B_n\sin\left(\frac{n\pi c}{\ell}t\right)\right)\sin\left(\frac{n\pi x}{\ell}\right) \]

PDEs are linear, so we can try adding up these $u_n$'s and get

\[ u(x,t) \overset{?}{=} \sum_{n=1}^\infty \left(A_n\cos\left(\frac{n\pi c}{\ell}t\right) + B_n\sin\left(\frac{n\pi c}{\ell}t\right)\right)\sin\left(\frac{n\pi x}{\ell}\right) \]

We're not even sure this sum makes sense, but, it actually does, and it is in fact the general solution for suitable choices of $A_n$ and $B_n$. 

Under this assumption that it solves the wave equation with a Dirichlet boundary condition, we get a strange corollary. Given initial conditions $u(x,0) = \phi(x)$ and $u_t(x,0) = \psi(x)$, we'd get that

\[ \phi(x) = \sum_{n=1}^\infty A_n\sin\left(\frac{n\pi x}{\ell}\right) \]
\[ \psi(x) = \sum_{n=1}^\infty (u_n)_t(x,0) = \sum_{n=1}^\infty B_n \frac{\pi nc}{\ell}\sin\left(\frac{n\pi x}{\ell}\right) \]

This leads to a question: is it possible to represent functions $\phi(x), \psi(x)$ in terms of a sum of the above form? Is such a representation unique? We will answer these questions later. 

For now, let's try and understand what this solution looks like, starting with the $X_n$'s, the spatial components. On $(0,\ell)$, $X_n(x)$ will have $n-1$ zeroes, and oscillates more and more rapidly as we increase $n$. The figure below shows $X_1, X_2,$ and $X_3$ when $\ell = 1$.

\begin{center}
    \includegraphics[width=0.7\textwidth]{Week 6/Xn.png}
\end{center}

What about the time components $T_n$? This will oscillate from 1 to $-1$, and do so faster as $n$ increases. Upon combining these values, $u_n(x,t)$ will look like the graph of $X_n(x)$, which will oscillate between $X_n(x)$ and $-X_n(x)$ faster and faster as $t$ increases. 

\begin{center}
    \includegraphics[width=0.7\textwidth]{Week 6/oscillating_string.png}
\end{center}

This is in contrast to the situation on the infinite or half-lines. Before, waves could ``escape" by moving off to infinity. Now, they are stuck, and thus continue to affect the string forever.

\subsection{The Heat Equation on a Finite Interval with Dirichlet B.C}

Now let's try the same strategy for the heat equation problem

\[ \begin{cases}
    u_t - ku_{xx} = 0 \\ u(0,t) = u(\ell,t) = 0 \\ u(x,0) = \phi(x)
\end{cases} \]

The strategy is the exact same. By making an initial guess that $u$ is some product of a function of $x$ and a function of $t$, we get that

\[ u_t - ku_{xx} = 0 \iff X(x)T'(t) - kX''(x)T(t) = 0 \]

and so

\[ \frac{T'(t)}{kT(t)} = \frac{X''(x)}{X(x)} = - \lambda \]

for a constant $\lambda$. We thus need to solve

\begin{equation*}
    X''(x) + \lambda X(x) = 0 \quad X(0) = X(\ell) = 0
\end{equation*}

\begin{equation*}
    T'(x) = K\lambda T(t) = 0 
\end{equation*}

We've done some of this in our analysis of the wave equation already: For $n = 1,2,3, \ldots$,

\[ X_n(x) = \sin\left(\frac{n\pi x}{\ell}\right), \quad \lambda_n = \left(\frac{n\pi}{\ell}\right)^2 \]

What does change is our equation for $T$. Plugging in what we know gives

\[ T'(t) + k\left(\frac{n\pi}{\ell}\right)^2T_n(t) = 0 \implies T_n(t) = A_n\exp\left(-k\left(\frac{n\pi}{\ell}\right)^2t\right) \]

and so 

\[ u_n(x,t) = A_n\exp\left(-k\left(\frac{n\pi}{\ell}\right)^2t\right)\sin\left(\frac{n\pi x}{\ell}\right) \]
\[ u(x,t) \overset{?}{=} \sum_{n=1}^\infty u_n(x,t) \]

With this solution, we get that $\phi(x) = u(x,0) \overset{?}{=} \sum_{n=1}^\infty A_n\sin\left(\frac{n\pi x}{\ell}\right)$. This looks different compared to the wave equation. While the $X_n$ looks the same, the $T_n$ is an exponential with negative exponent, meaning that it decreases in time, with the decrease going faster as $n \to \infty$. How fast is this decrease? Recall that for $x \in \R$, for a fixed $x$,

\[ u(x,t) = \frac{1}{\sqrt{t}}\left(\frac{1}{\sqrt{4\pi k}}\int_{-\infty}^\infty \phi(y) \ dy \right) + o\left(\frac{1}{\sqrt{t}}\right) \]

so decay on the infinite rod will occur in polynomial time. Since we have an exponential on the finite rod, we get exponential decay, meaning heat dissipates much faster in the finite case compared to the infinite case. 

\subsection*{An Aside Regarding Eigenfunctions}

We call the $\lambda_n$'s \textbf{eigenvalues} and the $X_n$'s \textbf{eigenfunctions}, which act like eigenvectors. These terms may seem strange as they feel out of place here. However, they are at home in this context.

Recall from linear algebra that an $n \times n$ real valued matrix $A$ is \textbf{self-adjoint} if $A = A^T$; $A$ is also called a \textbf{symmetric} matrix. In other words, for any vectors $\vec{v}, \vec{w} \in \R^n$,

\[ (A\vec{v}) \cdot \vec{w} = \vec{v} \cdot (A\vec{w})  \]

The spectral theorem tells us that there is an orthonormal basis of eigenvectors for such operators. We thus get vectors

\[ \vec{v_1}, \vec{v_2}, \ldots, \vec{v_n} \]

for which

\[ \vec{v_1} \cdot \vec{v_j} = \begin{cases}
    1 & i = j \ \text{ and } \ A\vec{v_i} = \lambda_i \vec{v_i} \ \text{ for some $\lambda_i \in \R$} \\
    0 & \text{otherwise}
\end{cases} \]

Now, if we take $A = \partial_x^2$ to be the second derivative of a function with respect to $x$, and let $x \in (0,\ell)$, then for functions $f(x),g(x)$ with

\[ f(0) = f(\ell) = 0 = g(0) = g(\ell) \]

the dot product is given by 

\[ (f,g) = \int_0^\ell f(x)g(x) \ dx \]

and one can see that $(\partial_x^2f,g) = (f,\partial_x^2g)$. Another version of the spectral theorem says that there is an orthonormal basis 

\[ x_1, \ldots, x_n \]

such that we get eigenvalues $\lambda_i$ for each $i$, and

\[ AX_n = \lambda_nx_n \]

Thus, $\phi(x) = \sum A_n X_n(x)$, exactly like we predicted before!



\chapter{Week 7}

\section{Working on a Finite Interval, Continued}

Last week we considered solutions to the wave and heat equations on finite intervals $(0, \ell)$ with Dirichlet boundary conditions. We begin this week with a study of the same problems but with Neumann boundary conditions.

\subsection{The Wave Equation on a Finite Interval with Neumann B.C}

We seek to solve the problem

\[ \begin{cases}
    u(x,t): (0,\ell) \times (0, \infty) \to \R \\
    u_{tt} - c^2u_{xx} = 0 \\
    u(x,0) = \phi(x), \quad u_t(x,0) = \psi(x) \\
    u_x(0,t) = u_x(\ell,t) = 0
\end{cases} \]

Like before, we make a guess that $u(x,t) = T(t)X(x)$, and we wish to find $\lambda \in \R$ such that

\[ X'' + \lambda X = 0 \]

has a nonzero solution satisfying our boundary conditions: $X'(0) = X'(\ell) = 0$. 

\begin{remark}
    We assume here that $\lambda \in \R$, but what if $\lambda \in \C$? Given possibly complex-valued functions $f(x),g(x)$ which satisfy the boundary conditions (which may be Dirichlet or Neumann, it doesn't matter), we get that

    \[ f \cdot g = \int_0^\ell f(x) \overline{g(x)} \ dx \]

    When boundary conditions are considered, integrating by parts shows that

    \begin{align*}
        \int_0^\ell f''(x) \overline{g(x)} \ dx & = f'(x)g(x) \bigg|_0^\ell - \int_0^\ell f'(x)\overline{g'(x)} \ dx \\
        & = - \int_0^\ell f'(x)\overline{g'(x)} \ dx \\
        & = -f(x)g'(x)\bigg|_0^\ell + \int_0^\ell f(x) \overline{g''(x)} \ dx \\
        & = \int_0^\ell f(x) \overline{g''(x)} \ dx
    \end{align*}

    Now, given $X'' + \lambda X = 0$ for some $\lambda \in \C$ with $X$ satisfying the boundary conditions, we get that

    \begin{align*}
        \int_0^\ell X''\overline{X} \ dx & = -\lambda\int_0^\ell |X|^2 \ dx \tag{$X'' = -\lambda X$} \\
        & = \int_0^\ell X \overline{X''} \ dx \tag{by the above work} \\
        & = -\overline{\lambda}\int_0^\ell |X|^2 \ dx 
    \end{align*}

    Thus, $\lambda = \overline{\lambda}$, and so $\lambda \in \R$. 
\end{remark}

We again need to consider cases regarding the value of $\lambda$:

\begin{enumerate}
    \item For $\lambda < 0$, the general solution to $X'' + \lambda X = 0$ is 

    \[ X(x) = Ae^{\sqrt{-\lambda}x}- Be^{-\sqrt{-\lambda}x} \]

    For constants $A,B$. We have that

    \[ X'(x) = A\sqrt{-\lambda}e^{\sqrt{-\lambda}x} - B\sqrt{-\lambda}e^{-\sqrt{-\lambda}x} \]

    \begin{align*}
        & X'(0) = A\sqrt{-\lambda} - B\sqrt{-\lambda} = 0 \\
        \implies & A = B
    \end{align*}

    \[ X'(\ell) = A\sqrt{-\lambda}e^{\sqrt{-\lambda}\ell} - A\sqrt{-\lambda}e^{-\sqrt{-\lambda}\ell} \]

    Assuming that $A \neq 0$, we have

    \begin{align*}
        & 0 = e^{\sqrt{-\lambda}\ell} - e^{-\sqrt{-\lambda}\ell} \\
        \implies & e^{-\sqrt{-\lambda}\ell} = e^{\sqrt{-\lambda}\ell} \\
        \implies & = -\sqrt{-\lambda} = \sqrt{-\lambda} \\
        \implies & \lambda = 0
    \end{align*}

    So we reject this case as it leads to a contradiction. 

    \item For $\lambda = 0$, then $X'' = 0$, so 

    \[ X(x) = Ax + B \]

    for constants $A,B$. We have that $X'(x) = A \implies X'(0) = A$, so $A = 0$. Moreover, $X'(\ell) = A = 0$. This is different from our analysis of Dirichlet boundary conditions, as $\lambda = 0$ yields a valid solution. Without loss of generality we take 

    \[ X(x) = 1 \]

    which is an eigenfunction with eigenvalue 0. 

    \item For $\lambda > 0$, we get the general solution

    \[ X(x) = A\sin(\sqrt{\lambda}x + B\cos(\sqrt{\lambda}x) \]

    for constants $A,B$. We have that

    \[ X'(x) = \sqrt{\lambda}A\cos(\sqrt{\lambda}x) - B\sqrt{\lambda}\sin(\sqrt{\lambda}x) \] 

    \begin{align*}
        & X'(0) = \sqrt{\lambda}A = 0 \\
        \implies & A = 0
    \end{align*}

    \[ X'(\ell) = -B\sqrt{\lambda}\sin(\sqrt{\lambda}\ell) = 0 \]

    $B = 0$ yields the degenerative case, so $\sin(\sqrt{\lambda}\ell) = 0$. We have seen this before, and can conclude that

    \[ \lambda = \left(\frac{\pi n}{\ell}\right)^2 \quad n = 1,2,3,\ldots \]

    which is the same as the Dirichlet boundary conditions. There is a small change in $X_n$, since we use $\cos$ instead of $\sin$:

    \[ X_n(x) = \cos\left(\frac{n\pi x}{\ell}\right) \]
\end{enumerate}

It should be noted that the $\lambda = 0$ solution will be treated as the solution for $n = 0$, which makes sense given that $\lambda_0 = 0$, lining up with $X_0(x) = 1$ being the eigenfunction with eigenvalue 0. 

For the function $T(t)$, we have that

\[ T_n'' + c^2\lambda_nT_n(t) = 0 \]

For $n = 0$, $T_0''(t) = 0$, so 

\[ T_0(t) = A_0t + B_0 \]

and when $n > 0$, we get

\[ T_n'' + c^2\left(\frac{n\pi}{\ell}\right)T_n = 0 \]

and so we have

\[ T_n(t) = A_n\sin\left(\frac{n\pi ct}{\ell}\right) + B_n\cos\left(\frac{n\pi ct}{\ell}\right) \]

meaning our final guess for the general solution is given by  

\[ u(x,t) = \frac{1}{2}(A_0t + B_0) + \sum_{n=1}^\infty A_n\sin\left(\frac{n\pi ct}{\ell}\right) + B_n\cos\left(\frac{n\pi ct}{\ell}\right) \]

where that factor for $1/2$ is there for the sake of simpler calculations. 

It should be noted that $\lim_{t \to \infty} u(x,t) = \infty$. What does this mean? Recalling that under Neumann conditions our string is held taught and fixed onto infinitely tall poles so that we can only move the endpoints, any action on the string, like flicking it upwards, will cause the string to just keep going in that direction forever. 

\subsection{The Heat Equation on a Finite Interval with Neumann B.C}

Now let's solve the heat equation on the same domain, with Neumann boundary conditions:

\[ \begin{cases}
    u_t - ku_{xx} = 0 \\
    u_x(0,t) = u_x(\ell,t) = 0 \\
    u(x,0) = \phi(x)
\end{cases} \]

Recalling that the spatial part $X(x)$ matches that of the wave equation, we get that 

\[ Xn(x) = \cos\left(\frac{n\pi x}{\ell}\right) \quad n = 0,1,2,3,\ldots \]
\[ \lambda_n = \left(\frac{n\pi}{\ell}\right)^2 \]

The equation for $T_n(t)$ has changed:

\[ T_n'(t) + k\lambda_nT_n(t) = 0 \]

So we have cases: If $n = 0$, then $T_0'(t) = 0$, meaning $T_0(t) = A_0$ for some constant $A_0$. 

If $n > 0$, we get 

\[ T_n(t) = A_n\exp(-k\lambda_nt) \]

Thus, we get that

\[ u(x,t) = A_0 + \sum_{n=1}^\infty A_n\exp\left(-kt\left(\frac{n\pi}{\ell}\right)^2\right)\cos\left(\frac{n\pi x}{\ell}\right) \]

Note that, like with Dirichlet conditions, as $t \to \infty$ the exponential goes to 0, meaning the sum will vanish and we are left with $A_0$. This makes sense, and means that our rod will, after a long enough time, have even temperature across its surface given by that constant. 

\subsection*{Other Boundary Conditions}

we can go through this exact same process for many different boundary conditions, though usually we wont be able to solve them exactly. Despite this, we will still get values that work, and we can still say things about them, which is what matters most in these contexts. 

\section{Fourier Transform}

In our previous derivations of the heat and wave equations on finite intervals, our results indicated that certain functions could be represented as the infinite sum of the sine and cosine functions. In this section, we demonstrate how this is true by attaining the necessary coefficients in these series.

\subsection{Fourier Sine Series}

We begin with the Fourier Sine Series, which works for Dirichlet boundary conditions. Suppose we have a function $\phi(x): (0,\ell) \to \R$. Can we find $\{A_n\}_{n=1}^\infty$ such that

\[ \phi(x) = \sum_{n=1}^\infty A_n\sin\left(\frac{n\pi x}{\ell}\right) \]

If we can solve the wave equation with Dirichlet boundary conditions, then we must be able to do this.

To begin answering our question, we need an important result:

\begin{theorem}
    For positive integers $n \neq m$, we have

    \[ \int_0^\ell \sin\left(\frac{n\pi x}{\ell}\right)\sin\left(\frac{m\pi x}{\ell}\right) \ dx = 0 \]
\end{theorem}

\begin{proof}
    Take $X_n = \sin\left(\dfrac{n\pi x}{\ell}\right)$. Then $X_n'' = -\lambda_nX_n$, where

    \[ \lambda_n = \left(\frac{n\pi}{\ell}\right)^2 \]

    We have that

    \[ \int_0^\ell X_n''X_m \ dx = -\lambda_n\int_0^\ell X_nX_m \ dx \]

    But using integration by parts, we get that

    \begin{align*}
        \int_0^\ell X_n''X_m \ dx & = -\int_0^\ell X_n'X_m' \ dx \\
        & = \int_0^\ell X_nX_m'' \ dx \\
        &=  -\lambda_m\int_0^\ell X_nX_m \ dx
    \end{align*}

    $n \neq m$, so $\lambda_m \neq \lambda_n$. Thus, the only possibility is that

    \[ \int_0^\ell X_nX_m \ dx = 0 \]

    as desired. 
\end{proof}

What about the case that $n = m$? Note that

\[ \sin^2\theta = \frac{1}{2} - \frac{1}{2}\cos(2\theta) \]

Using this fact, we get that

\begin{align*}
    \int_0^\ell \sin^2\left(\frac{n\pi x}{\ell}\right) \ dx & = \int_0^\ell \frac{1}{2} \ dx - \frac{1}{2}\int_0^\ell \cos\left(\frac{2n\pi x}{\ell}\right) \ dx \\
    & = \frac{\ell}{2} - \frac{1}{2}\left[\frac{\ell}{2n \pi}\sin\left(\frac{2n\pi x}{\ell}\right)\right]_0^\ell \\
    & = \frac{\ell}{2} - \frac{1}{2}\left[\frac{\ell}{2n\pi}\sin(2n\pi) - \frac{\ell}{2n\pi}\sin(0)\right] \\
    & = \frac{\ell}{2}
\end{align*}

With all this in hand, we can begin deriving our coefficients. We have that

\[ \phi(x) \overset{?}{=} \sum_{n=1}^\infty A_n\sin\left(\frac{n\pi x}{\ell}\right) \]

To find the coefficients of this series, we can take the dot product of it and the sine value. This gives us

\begin{align*}
    \int_0^\ell \phi(x) \sin\left(\frac{m \pi x}{\ell}\right) \ dx
    & = \int_0^\ell \left(\sum_{n=1}^\infty A_n \sin\left(\frac{n\pi x}{\ell}\right)\right)\sin\left(\frac{m\pi x}{\ell}\right) \ dx \\
    & = \sum_{n=1}^\infty A_n\int_)^\ell \sin\left(\frac{n\pi x}{\ell}\right)\sin\left(\frac{m\pi x}{\ell}\right) \ dx \\
    & = A_m\int_0^\ell \sin^2\left(\frac{m\pi x}{\ell}\right) \\
    & = \frac{A_m \ell}{2}
\end{align*}

Thus, we conclude that

\[ A_m = \frac{2}{\ell}\int_0^\ell \phi(x) \sin\left(\frac{m\pi x}{\ell}\right) \ dx \]

This gives us the coefficients of $\phi$ in the wave equation! Recall the solution on a finite interval with Dirichlet boundary conditions given by

\[ u(x,t) = \sum_{n=1}^\infty \left(A_n\cos\left(\frac{n\pi ct}{\ell}\right) + B_n\sin\left(\frac{n\pi ct}{\ell}\right)\right)\sin\left(\frac{n\pi x}{\ell}\right) \]

We have that

\[ \phi(x) = u(x,0) = \sum_{n=1}^\infty A_n\sin\left(\frac{n\pi x}{\ell}\right) \]

meaning

\[ A_n = \frac{2}{\ell}\int_0^\ell \phi(x) \sin\left(\frac{n\pi x}{\ell}\right) \ dx \]

Moreover, we can also find the coefficients for $\psi(x)$! We have

\begin{align*}
    \psi(x) = u_t(x,0) & = \sum_{n=1}^\infty \left(\frac{-A_n n \pi c}{\ell}\sin\left(\frac{n\pi c \cdot 0}{\ell}\right) + \frac{B_nn\pi c}{\ell}\cos\left(\frac{n\pi c \cdot 0}{\ell}\right)\right) \sin\left(\frac{n \pi x}{\ell}\right) \\
    & = \sum_{n=1}^\infty \frac{B_n n \pi c}{\ell}\sin\left(\frac{n \pi x}{\ell}\right)
\end{align*}

And so we get that

\[ \frac{B_nn\pi c}{\ell} = \frac{2}{\ell}\int_0^\ell \psi(x)\sin\left(\frac{n\pi x}{\ell}\right) \ dx \]
\[ \implies B_n = \frac{2}{n\pi c}\int_0^\ell \psi(x)\sin\left(\frac{n\pi x}{\ell}\right) \ dx  \]

\subsection{Fourier Cosine Series}

We now find the coefficients of the Fourier Cosine Series. Recall that, using Neumann boundary conditions, saw that

\[ \phi(x) = \frac{1}{2}A_0 + \sum_{n=1}^\infty A_n\cos\left(\frac{n\pi x}{\ell}\right) \]

What are the values of $A_0, A_1, A_2, \ldots$?

First, we need a result:

\begin{theorem}
    For $n \neq m$, we have that

    \[ \int_0^\ell \cos\left(\frac{n\pi x}{\ell}\right)\cos\left(\frac{m\pi x}{\ell}\right) \ dx = 0 \]
\end{theorem}

\begin{proof}
    By setting $X_n = \cos\left(\dfrac{n\pi x}{\ell}\right)$, we again get that $X_n'' = -\lambda_nX_n$, where $\lambda_n = \left(\dfrac{n\pi}{\ell}\right)^2$. Repeating the method from the sine version of this theorem suffices. 
\end{proof}

This works for $n,m \in \{0,1,2,3,\ldots\}$. To find the coefficients we again take the dot product, but now of $\phi$ with cosine. 

\begin{align*}
    \int_0^\ell \phi(x) \cos\left(\frac{m\pi x}{\ell}\right) \ dx & = \int_0^\ell \left(\frac{1}{2}A_0 + \sum_{n=1}^\infty A_n\cos\left(\frac{n\pi x}{\ell}\right)\right)\cos\left(\frac{m\pi x}{\ell}\right) \ dx \\
    & = \frac{1}{2}\int_0^\ell A_0\cos\left(\frac{m\pi x}{\ell}\right) \ dx + \sum_{n=1}^\infty \int_0^\ell A_n \cos\left(\frac{n\pi x}{\ell}\right)\cos\left(\frac{m\pi x}{\ell}\right) \ dx
\end{align*}

If $m = 0$, the infinite sum will vanish by our above theorem, leaving us with

\[ \frac{\ell}{2}A_0 \]

While if $m \neq 0$, the first term will vanish, also by the above theorem, and so will all terms in the infinite sum except $m = n$, leaving us with

\[ A_m\int_0^\ell \cos^2\left(\frac{m\pi x}{\ell}\right) \ dx \]

Noting that

\[ \int_0^\ell \cos^2\left(\frac{m\pi x}{\ell}\right) \ dx = \frac{\ell}{2} \]

We conclude that 

\[ A_m = \begin{cases}
    \frac{2}{\ell}\int_0^\ell \phi(x) \ dx & m = 0 \\
    \frac{2}{\ell}\int_0^\ell \phi(x) \cos\left(\frac{m\pi x}{\ell}\right) \ dx & m \neq 0
\end{cases} \]

Which we can combine to just say that 

\[ A_n = \frac{2}{\ell}\int_0^\ell \phi(x) \cos\left(\frac{n\pi x}{\ell}\right) \ dx \]

The $B_n$ coefficients can be derived from this like we did before. Note the value $A_0$, which is the average of the initial heat distribution. This makes sense, because in the limit, heat on a rod will become equally distributed; the heat at each point will correspond to the average value of heat on the rod.

\subsection{Full Fourier Series: Periodic Boundary Conditions}

Consider the heat equation, but instead of using a straight rod, we consider a \textit{circular} rod.

\begin{center}
    \includegraphics[width=0.35\textwidth]{Week 7/circle.jpg}
\end{center}

Now, there is no ``boundary" for us to work with. How can we deal with this?

Let's consider our domain of $x$ to be $(-\ell, \ell)$. The condition now is that $u(x,t)$ is \textbf{periodic} in $x$ with period $2\ell$. We can express this as

\[ u(-\ell,t) = u(\ell,t) \]
\[ u_x(\ell,t) = u_x(\ell,t) \]

Solving this will require separation of variables again. Upon doing this, we will get 3 equations:

\[ X'' + \lambda X = 0 \quad x \in (-\ell,\ell) \]
\[ X(\ell) = X(-\ell) \]
\[ X'(\ell) = X'(-\ell) \]

with $\lambda \in \R$ (the proof of this is like what we did before. 

There are a few options for $X$, we can have 

\[ X_n = \sin\left(\frac{n\pi x}{\ell}\right) \quad \lambda_n = \left(\frac{n\pi}{\ell}\right)^2 \quad n = 1,2,3, \ldots \]
\[ X_n = \cos\left(\frac{n\pi x}{\ell}\right) \quad \lambda_n = \left(\frac{n\pi}{\ell}\right)^2 \quad n = 0,1,2,3, \ldots \]

If we are doing the heat equation, we will have that 

\[ \phi(x) \overset{?}{=} \frac{1}{2}A_0 + \sum_{n=1}^\infty \left(A_n\cos\left(\frac{n\pi x}{\ell}\right) + B_n\sin\left(\frac{n\pi x}{\ell}\right)\right) \]

which is the \textbf{full Fourier series} for $x \in (-\ell,\ell)$. In order to find the coefficients $A_n,B_n$, it would be really nice if certain integrals cancelled out like before. In particular, it would be nice if

\begin{enumerate}
    \item $\int_{-\ell}^\ell \sin\left(\frac{n\pi x}{\ell}\right)\cos\left(\frac{m\pi x}{\ell}\right) \ dx = 0$
    \item $\int_{-\ell}^\ell \sin\left(\frac{n\pi x}{\ell}\right)\sin\left(\frac{m\pi x}{\ell}\right) \ dx = 0$
    \item $\int_{-\ell}^\ell \cos\left(\frac{n\pi x}{\ell}\right)\cos\left(\frac{m\pi x}{\ell}\right) \ dx = 0$
\end{enumerate}

Luckily they are all in fact true!

\chapter{Week 8}

\section{Fourier Transform, Continued}

\subsection{Full Fourier Series, Continued}

Towards the end of last week, we introduced the full Fourier series

\[ \phi(x) = \frac{1}{2}A_0 + \sum_{n=1}^\infty \left(A_n\cos\left(\frac{n\pi x}{\ell}\right) + B_n\sin\left(\frac{n\pi x}{\ell}\right)\right) \]

This week we will find those coefficients $A_n,B_n$. 

To do this we need to 3 integrals to vanish:

\begin{enumerate}
    \item $\int_{-\ell}^\ell \sin\left(\frac{n\pi x}{\ell}\right)\cos\left(\frac{m\pi x}{\ell}\right) \ dx = 0$ 
    \item $\int_{-\ell}^\ell \sin\left(\frac{n\pi x}{\ell}\right)\sin\left(\frac{m\pi x}{\ell}\right) \ dx = 0$
    \item $\int_{-\ell}^\ell \cos\left(\frac{n\pi x}{\ell}\right)\cos\left(\frac{m\pi x}{\ell}\right) \ dx = 0$
\end{enumerate}

To see why they vanish, recall that if a function $f(x)$ is \textit{odd}, then

\[ \int_{-\ell}^\ell f(x) \ dx = 0 \]

For all $n,m$, we have that

\[ \sin\left(\frac{n\pi x}{\ell}\right)\cos\left(\frac{m\pi x}{\ell}\right)\]

is odd; the product of an odd function, $\sin$, and an even function, $\cos$, is always odd. Thus, its integral from $-\ell$ to $\ell$ will vanish. 

For the other two functions, they are either the product of two even functions, as seen with $\cos$, or two odd functions, as seen with $\sin$, meaning both are even functions. This is not a bad thing, though, as for any even function $f(x)$,

\[ \int_{-\ell}^\ell f(x) \ dx = 2\int_0^\ell f(x) \ dx \]

Thus, we get that for $n \neq m$,

\[ \int_{-\ell}^\ell \cos\left(\frac{n\pi x}{\ell}\right)\cos\left(\frac{m\pi x}{\ell}\right) \ dx = 2\int_0^\ell \cos\left(\frac{n\pi x}{\ell}\right)\cos\left(\frac{m\pi x}{\ell}\right) \ dx = 0 \]
\[ \int_{-\ell}^\ell \sin\left(\frac{n\pi x}{\ell}\right)\sin\left(\frac{m\pi x}{\ell}\right) \ dx = 2\int_0^\ell \sin\left(\frac{n\pi x}{\ell}\right)\sin\left(\frac{m\pi x}{\ell}\right) \ dx = 0 \]

which follows from the Fourier cosine and sine series respectively. The last thing we need to do is consider when $n = m$ in both cases. 

For $\cos$, if $n \neq 0$, we get

\begin{align*}
    \int_{-\ell}^\ell \cos^2\left(\frac{n\pi x}{\ell}\right) \ dx & = 2\int_0^\ell \cos^2\left(\frac{n\pi x}{\ell}\right) \ dx \\ 
    & = 2\left(\frac{\ell}{2}\right) \tag{derived previously} \\
    & = \ell
\end{align*}

and if $n = 0$, we get

\[ \int_{-\ell}^\ell \cos(0) \ dx = \int_{-\ell}^\ell 1 \ dx = 2\ell \]

For $\sin$, we only consider when $n \neq 0$, giving us

\begin{align*}
    \int_{-\ell}^\ell \sin^2\left(\frac{n\pi x}{\ell}\right) \ dx & = 2\int_0^\ell \sin^2\left(\frac{n\pi x}{\ell}\right) \ dx \\
    & = 2\left(\frac{\ell}{2}\right) \\
    & = \ell
\end{align*}

Upon taking dot products and simplifying, we will get that

\[ A_n = \frac{1}{\ell}\int_{-\ell}^\ell \phi(x) \cos\left(\frac{n\pi x}{\ell}\right) \ dx \]
\[ B_n = \frac{1}{\ell}\int_{-\ell}^\ell \phi(x) \sin\left(\frac{n\pi x}{\ell}\right) \ dx \]

\subsection{Understanding Convergence}

Throughout our study of these infinite sums, we've mad a bit of an assumption in saying that that they do in fact converge and thus are well-defined. While full rigour is not necessarily what we are going for in MAT311 (we leave that to those working in real and functional analysis), we should probably get some understanding of when these series converge. To do this we will state 3 theorems that give different flavours of convergence.

\begin{theorem}
    Suppose $f$ is a $C^1$ function that satisfies either the Dirichlet, Neumann, or Periodic boundary conditions. Then we have that $f$ is equal to the respective Fourier series (sine, cosine, or full), and

    \[ \lim_{N \to \infty} \max_x\left|f(x) - \sum_{n=1}^N (\cdots)(x)\right| = 0 \]

    so $f(x) = \sum_{n=1}^\infty (\cdots)(x)$
\end{theorem}

This means that if $f$ satisfies the boundary conditions of our PDE, and is $C^1$, then it is of the form we've previously defined and equal to the infinite sum (meaning said sum converges).

\begin{theorem}
    Suppose $f$ is continuous and $f'$ is piecewise continuous, all on some interval $[0,\ell]$. Then for $x \in (0,\ell)$, 

    \[ f(x) = \sum_{n=1}^\infty (\cdots)(x) \]
\end{theorem}

This is a weaker result but is still quite useful, the only downside being that we cannot use it to deal with the endpoints $x = 0, \ell$. 

Here's one last theorem that gives another version of convergence of the sum. 

\begin{theorem}
    Suppose that 

    \[ \int f^2 \ dx < \infty \]

    Then

    \[ \lim_{N \to \infty}\int \left|f(x) - \sum_{n=1}^N (\cdots)(x)\right|^2 \ dx = 0 \]
\end{theorem}

With these in mind, we can now compute some actual Fourier series.

\begin{example}
    Compute the Fourier cosine series of $f(x) = 1$ on $(0, \ell)$.

    For $n \neq 0$, the coefficients are given by

    \begin{align*}
        A_n & = \frac{2}{\ell}\int_0^\ell 1 \cdot \cos\left(\frac{n\pi x}{\ell}\right) \ dx \\
        & = \frac{2}{\ell}\left[\frac{\ell}{n\pi}\sin\left(\frac{n\pi x}{\ell}\right)\right]_0^\ell \\
        & = 0
    \end{align*}

    while for $n = 0$, 

    \[ A_0 = \frac{2}{\ell}\int_o^\ell 1 \ dx = 2 \]

    Thus, we get that

    \[ 1 = \frac{1}{2}A_0 + 0 + \cdots = 1 \]

    This is a silly example, but it is still instructive.
\end{example}

\begin{example}
    Compute the Fourier sine series of $f(x) = 1$ on $(0, \ell)$. 

    The coefficients are given by

    \begin{align*}
        A_n & = \frac{2}{\ell}\int_0^\ell 1 \cdot \sin\left(\frac{n\pi x}{\ell}\right) \ dx \\
        & = \frac{2}{\ell}\left[\frac{-\ell}{n\pi}\cos\left(\frac{n\pi x}{\ell}\right)\right]_0^\ell \\
        & = \frac{-2}{n\pi}(\cos(n\pi) - 1) \\
        & = \frac{-2}{n\pi}[(-1)^n - 1] \\
        & = \begin{cases}
        0 & n \ \text{ is even} \\
        \frac{4}{n\pi} & n \ \text{ is odd}
        \end{cases}
    \end{align*}

    Hence we have that

    \[ 1 \overset{?}{=} \frac{4}{\pi}\left(\sin(x) + \frac{1}{3}\sin(3x) + \frac{1}{5}\sin(5x) + \cdots\right) \]

    Theorem 8.2 tells us that this equality is true when $x \in (0,\ell)$, but not when $x = 0, \ell$. Indeed,

    \[ 1 \neq \frac{4}{\pi}(0 + 0 + \cdots) = 0 \]

    so we get that 1 does not satisfy the Dirichlet boundary conditions in this case. 
\end{example}
\chapter{Week 9}

\section{More Examples of Fourier Series}

Let's compute some additional examples of Fourier series.

\begin{example}
    compute the Fourier sine series for $f(x) = x$ on $[0, \ell]$. 

    We have that

    \begin{align*}
        A_m & = \frac{2}{\ell}\int_0^\ell x\sin\left(\frac{m\pi x}{\ell}\right) \ dx \\
        & = \frac{-2}{\ell}\int_0^\ell x\left(\frac{\ell}{m\pi}\right)\dif{}{x}\left[\cos\left(\frac{m\pi x}{\ell}\right)\right] \ dx \\
        & = \frac{-2x}{m\pi} \cos\left(\frac{m\pi x}{\ell}\right)\bigg|_0^\ell + \frac{2}{m\pi}\int_0^\ell \dif{}{x}(x) \cos\left(\frac{m\pi x}{\ell}\right) \ dx \\
        & = \frac{-2\ell}{m\pi}\cos(m\pi) + \frac{2\ell}{(m\pi)^2} \sin\left(\frac{m \pi x}{\ell}\right)\bigg|_0^\ell \\
        & = \frac{-2\ell}{m\pi}(-1)^m \\
        & = \frac{2\ell}{m\pi}(-1)^{m+1}
    \end{align*}

    This tells us that

    \[ x \overset{?}{=} \frac{2\ell}{\pi}\left[\sin\left(\frac{\pi x}{\ell}\right) - \frac{1}{2}\sin\left(\frac{2\pi x}{\ell}\right) + \frac{1}{3}\sin\left(\frac{3\pi x}{\ell}\right) - \cdots \right] \]

    Recall that $f(x) = x$ is smooth on $x \in (0,\ell)$, so the series converges at these points. If $x = \ell$, the sum on the right vanishes and we do not get equality. We can see this in the figure below, where we show the series with 1 (green), 5 (blue), 10 (orange), and 100 (purple) terms, converging to the red line, which is $f(x)$. Notice how all of the functions vanish at $\ell$, which in the graph is set to 20.

    \begin{center}
        \includegraphics[width=0.6\textwidth]{Week 9/sine_series_at_x.png}        
    \end{center}
\end{example}

\begin{example}
    Compute the Fourier cosine series for $f(x) = x$ on $[0, \ell]$. 

    For the $m = 0$ case, we get

    \[ A_0 = \frac{2}{\ell}\int_0^\ell x \cdot 1 \ dx = \frac{2}{\ell}\frac{x^2}{2}\bigg|_0^\ell = \frac{\ell^2}{\ell} = \ell \]

    For $m > 0$, we get

    \begin{align*}
        A_m & = \frac{2}{\ell}\int_0^\ell x \cos\left(\frac{m \pi x}{\ell}\right) \ dx \\
        & = \frac{2}{\ell}\int_0^\ell x \frac{\ell}{m\pi} \dif{}{x}\left[\sin\left(\frac{m \pi x}{\ell}\right)\right] \ dx \\
        & = \frac{2x}{m\pi}\sin\left(\frac{m\pi x}{\ell}\right)\bigg|_0^\ell - \frac{2}{m\pi}\int_0^\ell \dif{}{x}(x)\sin\left(\frac{m\pi x}{\ell}\right) \ dx \\
        & = \frac{2\ell}{(m\pi)^2} \cos\left(\frac{m\pi x}{\ell}\right)\bigg|_0^\ell \\
        & = \frac{2\ell}{(m\pi)^2}((-1)^m - 1) \\
        & = \begin{cases}
            0 & m \text{ is even} \\
            -\dfrac{4\ell}{(m\pi)^2} & m \text{ is odd}
        \end{cases}
    \end{align*}

    Thus,

    \[ x \overset{?}{=} \frac{1}{2}\ell - \frac{4\ell}{\pi^2}\left[\cos\left(\frac{\pi x}{\ell}\right) + \frac{1}{3^2}\cos\left(\frac{3\pi x}{\ell}\right) + \frac{1}{5^2}\cos\left(\frac{5\pi x}{\ell}\right) + \cdots \right] \]

    This is in fact an equality when $x \in (0,\ell)$, but there's no guarantee it works on the endpoints. The figure below shows the sums with 1 (green), 3 (blue), 5 (orange), and 10 (purple) terms. Notice how quickly the convergence occurs compared to the sine series.

    \begin{center}
        \includegraphics[width=0.6\textwidth]{Week 9/cosine_series_of_x.png}
    \end{center}
\end{example}

\begin{example}
    Find the full Fourier series for $f(x)$ on $(-\ell, \ell)$. 

    As $x$ is odd, we get that

    \[ A_0 = \frac{1}{\ell}\int_{-\ell}^\ell x \ dx = 0 \]

    For $m > 0$, we get that

    \[ A_m = \frac{1}{\ell}\int_{-\ell}^\ell x \cos\left(\frac{m\pi x}{\ell}\right) \ dx = 0 \]

    since $x$ is odd and $\cos$ is even. $\sin$ is odd, so we get that

    \begin{align*}
        B_m & = \frac{1}{\ell}\int_{-\ell}^\ell x \sin\left(\frac{m\pi x}{\ell}\right) \ dx \\
        & = \frac{2}{\ell}\int_0^\ell x \sin\left(\frac{m\pi x}{\ell}\right) \ dx \\
        & = \frac{2\ell}{m\pi}(-1)^{m+1}
    \end{align*}

    which follows from the sine series calculation above. thus, the full Fourier series for $f(x) = x$ is just the sine series, taken on $(-\ell, \ell)$. 
\end{example}

\begin{example}
    Solve $u_{tt} - c^2u_{xx} = 0$ given

    \[ \begin{cases}
        u(0,t) = u(0,\ell) = 0 \quad t > 0 \\
        u(x,0) = x, u_t(x,0) = 0
    \end{cases} \]

    We know that

    \[ u(x,t) = \sum_{n=1}^\infty \left(A_n\cos\left(\frac{n\pi ct}{\ell}\right) + B_n\sin\left(\frac{n\pi ct}{\ell}\right)\right)\sin\left(\frac{n\pi x}{\ell}\right) \]

    and that

    \[ x = \sum_{n=1}^\infty A_n \sin\left(\frac{n\pi x}{\ell}\right), \quad 0 = \sum_{n=1}^\infty \frac{n\pi c}{\ell}B_n \sin\left(\frac{n\pi x}{\ell}\right) \]

    Thus, we get that

    \[ A_n = \frac{2\ell}{\pi}\frac{(-1)^{n+1}}{n}, \quad B_n = 0 \]

    Hence, we conclude that

    \[ u(x,t) = \frac{2\ell}{\pi}\sum_{n=1}^\infty \frac{(-1)^{n+1}}{n}\cos\left(\frac{n\pi ct}{\ell}\right)\sin\left(\frac{n\pi x}{\ell}\right) \]
\end{example}

We conclude this section with 2 remarks. 

First, suppose we wish to expand a function $f(x)$ into its Fourier sine or cosine series, but it is defined on some interval $(a,b)$ instead of $(0,\ell)$. To solve this, take

\[ g(x) = f(x \pm c ) \]

where $g$ is defined on $x \in (0, \ell)$. In this case, we take

\[ g(x) = f(x+a), \quad x \in (0,b-a) \]

We can then expand $g$ into its sine or cosine series. For example, if we expand it into its sine series, we get

\[ g(x) = \sum_{n=1}^\infty A_n\sin\left(\frac{n\pi x}{b-a}\right), \quad A_n = \frac{2}{b-a}\int_0^{b-a}g(x) \sin\left(\frac{n\pi x}{b-a}\right) \ dx \]
\[ \implies f(x) = g(x-a) = \sum_{n=1}^\infty A_n\sin\left(\frac{n\pi(x-a)}{b-a}\right), \quad A_n = \frac{2}{b-a}\int_0^{b-a}g(x-a)\sin\left(\frac{n\pi(x-a)}{b-a}\right) \ dx  \]

Our second remark concerns the example of the full Fourier series for $f(x) = x$. The fact that it is just the sine series is not a coincidence, bur rather a general fact of odd functions. 

Similarly, for even functions $\phi(x)$, we have that

\[ \int_{-\ell}^\ell \sin\left(\frac{n\pi x}{\ell}\right) \ dx = 0 \]
\[ \frac{1}{\ell}\int_{-\ell}^\ell \phi(x) \cos\left(\frac{n\pi x}{\ell}\right) = \frac{2}{\ell}\int_0^\ell \phi(x)\cos\left(\frac{n\pi x}{\ell}\right) \ dx \]

Thus, the full Fourier series of $\phi(x)$ is just the cosine series. 

Finally, recall that 

\[ \phi(x) = \frac{1}{2}[\phi(x) + \phi(-x)] + \frac{1}{2}[\phi(x) - \phi(x)] \]

where the first component is an even function and the second component is an odd function (check this). Thus, the full Fourier series of any function corresponds to the Fourier cosine series of the even part and the Fourier sine series of the odd part. 

\section{Complex Exponentials \& Fourier Series}

Recall that the sine and cosine functions may be written using complex exponentials:

\[ \sin\theta = \frac{e^{i\theta} - e^{-i\theta}}{2} \quad \cos\theta = \frac{e^{i\theta}+e^{-i\theta}}{2} \]

This means that any expansions involving sine and cosine, like in Fourier series, can be written instead as expansions in $e^{\pm\frac{in\pi x}{\ell}}$. This can be tedious, but we can use the same methods from before, and in fact, these methods become much easier to do once we incorporate complex exponentials!

Let's derive the full Fourier series using $\{e^{\frac{i\pi nx}{\ell}}\}_{n=-\infty}^\infty$. The new orthogonality condition is that for $n \neq m$

\[ \int_{-\ell}^\ell e^{\frac{in\pi x}{\ell}}\overline{e^{\frac{im\pi x}{\ell}}} \ dx = \int_{-\ell}^\ell e^{\frac{in\pi x}{\ell}}e^{-\frac{im\pi x}{\ell}}  \ dx \]

We get that

\[ \int_{-\ell}^\ell e^{\frac{i(n-m)\pi x}{\ell}} \ dx = \int_{-\ell}^\ell \frac{\ell}{(n-m)} \dif{}{x} e^{\frac{i(n-m)\pi x}{\ell}} \ dx = 0 \]

because $e^{\frac{ik\pi x}{\ell}}$ is periodic over $(-\ell, \ell)$ whenever $k$ is an integer. Moreover, we have that

\[ \int_{-\ell}^\ell e^{\frac{in\pi x}{\ell}}\overline{e^{\frac{in\pi x}{\ell}}} \ dx = \int_{-\ell}^\ell 1 \ dx = 2\ell \]

Now, the full Fourier series for $f(x)$ on $(-\ell, \ell)$ is given by

\[ f(x) = \sum_{n=-\infty}^\infty C_ne^{\frac{in\pi x}{\ell}} \]

We see that

\[ \int_{-\ell}^\ell f(x)e^{-\frac{in\pi x}{\ell}} \ dx = \sum_{n=-\infty}^\infty C_n\int_{-\ell}^\ell e^{\frac{in\pi x}{\ell}}e^{-\frac{in\pi x}{\ell}} \ dx = 2\ell C_n \]

Thus,

\[ C_m = \frac{1}{2\ell}\int_{-\ell}^\ell f(x) e^{-\frac{im\pi x}{\ell}} \quad m = 0,1,2,3, \ldots \]

The $=$ sign has the same conditions as for regular Fourier series.

\begin{example}
    Compute the full Fourier series for $f(x) = x$ on $(-\ell,\ell)$ using the complex exponential. 

    We have that, for $n = 0$,

    \[ C_0 = \frac{1}{2\ell}\int_{-\ell}^\ell x \ dx = 0 \]

    and for $n \neq 0$, we get

    \begin{align*}
        C_n & = \frac{1}{2\ell}\int_{-\ell}^\ell x e^{-\frac{in\pi x}{\ell}} \ dx \\
        & = \frac{1}{2\ell}\int_{-\ell}^\ell x \left(-\frac{\ell}{in\pi}\right) \dif{}{x}(e^{-\frac{in\pi x}{\ell}}) \ dx \\
        & = \frac{-x}{2in\pi}e^{-\frac{in\pi x}{\ell}}\bigg|_{-\ell}^\ell + \frac{1}{2in\pi}\int_{-\ell}^\ell e^{-\frac{in\pi x}{\ell}}\ dx \\
        & = \frac{-\ell}{2in\pi}e^{-in\pi} - \frac{\ell}{2in\pi}e^{in\pi} + \frac{1}{2in\pi}\int_{-\ell}^\ell e^{-\frac{in\pi x}{\ell}}\ dx \\
        & = \frac{-\ell}{2in\pi}e^{-inx}(1 + e^{2\pi in)} \\
        & = \frac{-\ell}{in\pi}e^{in\pi} \\
        & = \frac{-\ell}{in\pi}(-1)^n
    \end{align*}
\end{example}

\section{The Full Version of the Heat Equation on a Finite Interval}

To wrap up our study of Fourier series, let's go all in and solve the heat equation in its most general of forms: for $x \in (0, \ell)$,

\[ \begin{cases}
    u_t - ku_{xx} = f(x,t) \\
    u(0,t) = h(t), u(\ell,t) = j(t) \\
    u(x,0) = \phi(x)
\end{cases} \]

To solve this, we could try separation of variables. Setting $u(x,t) = X(x)T(t)$, we get

\[ X(x)T'(t) - kX''(x)T(t) = f(x,t) \]

and there isn't much else that we can do. Instead, let's try a different approach.

\begin{remark}
    What we are going to do works on $(0,\ell)$, but is not guaranteed to work on the end points.
\end{remark}

We write

\[ u(x,t) = \sum_{n=1}^\infty u_n(t) \sin\left(\frac{n\pi x}{\ell}\right) \]
\[ f(x,t) = \sum_{n=1}^\infty f_n(t) \sin\left(\frac{n\pi x}{\ell}\right) \]

We wish to derive equations for each $A_n(t)$. A naive approach would be to apply the PDE to this series, giving us

\[ \left(\pardif{}{t} - k\pardif{^2}{x^2}\right)\left[\sum_{n=1}^\infty u_n(t) \sin\left(\frac{n\pi x}{\ell}\right)\right] = \sum_{n=1}^\infty f_n(t) \sin\left(\frac{n\pi x}{\ell}\right) \]

where $f_n(t)$ are the coefficients in $f$'s Fourier sine series. This seems like a good approach, but there are issues along the boundaries, and so this will not always work. For example, we have previously seen that

\[ 1 = \sum_{n \text{ odd}} \frac{4}{n\pi} \sin(nx) \]

but deriving gives us

\[ 0 = \sum_{n \text{ odd}} \frac{4}{\pi}\cos(nx) \]

which is clearly not true since the right hand side is not going to 0. 

Not all hope is lost, as we still have one more trick up our sleeves. We will instead try to find the coefficients \textit{directly}. Instead of applying the PDE to the sum, let's plug the PDE into it:

\[ \int_0^\ell (u_t - ku_{xx}) \sin\left(\frac{n\pi x}{\ell}\right) \ dx = \int_0^\ell f(x,t) \sin\left(\frac{n\pi x}{\ell}\right) = f_n(t) \]

Let's break up the right side into two equations.

\[ \frac{2}{\ell}\int_0^\ell u_t\sin\left(\frac{n\pi x}{\ell}\right) \ dx - \frac{2k}{\ell}\int_0^\ell u_{xx}\sin\left(\frac{n\pi x}{\ell}\right) \ dx \]

We compute each of these individually: Using reverse differentiation under the integral sign, we get

\begin{align*}
    \frac{2}{\ell}\int_0^\ell u_t\sin\left(\frac{n\pi x}{\ell}\right) \ dx & = \dif{}{t}\frac{2}{\ell}\int_0^\ell u \sin\left(\frac{n\pi x}{\ell}\right) \ dx \\
    & = \dif{}{t} u_n
\end{align*}

Moreover, applying IBP twice gives us

\begin{align*}
    \frac{-2k}{\ell}\int_0^\ell u_{xx}\sin\left(\frac{n\pi x}{\ell}\right) \ dx & = \frac{-2k}{\ell}u_x\sin\left(\frac{n\pi x}{\ell}\right)\bigg|_0^\ell + \frac{2k}{\ell}\int_0^\ell u_x\left(\frac{n\pi}{\ell}\right)\cos\left(\frac{n\pi x}{\ell}\right) \ dx \\
    & = \frac{2kn\pi}{\ell^2}u\cos\left(\frac{n\pi x}{\ell}\right)\bigg|_0^\ell + \frac{2kn\pi}{\ell^2}\left(\frac{n\pi}{\ell}\right)\int_0^\ell u \sin\left(\frac{n\pi x}{\ell}\right) \ dx \\
    & = \frac{(-1)^n2kn\pi}{\ell^2}j(t) - \frac{2kn\pi}{\ell^2}h(t) + k\left(\frac{n\pi}{\ell}\right)^2\left(\frac{2}{\ell}\int_0^\ell u\sin\left(\frac{n\pi x}{\ell}\right) \ dx \right) \\
    & = \frac{(-1)^n2kn\pi}{\ell^2}j(t) - \frac{2kn\pi}{\ell^2}h(t) + k\left(\frac{n\pi}{\ell}\right)^2u_n
\end{align*}

Combining, we get an ODE,

\[ \dif{}{t}u_n(t) + k\left(\frac{n\pi}{\ell}\right)^2u_n(t) = f_n(x,t) + \frac{2kn\pi}{\ell^2}(h(t) - j(t)(-1)^n) \]

which we can solve using an integrating factor. We write

\[ \dif{}{t}u_n(t) + k\left(\frac{n\pi}{\ell}\right)^2u_n(t) = e^{-tk\left(\frac{n\pi}{\ell}\right)^2}\dif{}{t}[e^{tk\left(\frac{n\pi}{\ell}\right)^2}u_n(t)] \]

and setting $P_n(t) = f_n + \frac{2kn\pi}{\ell^2}(h(t) - j(t)(-1)^n)$, we get

\begin{align*}
    & \implies \dif{}{t}[e^{tk\left(\frac{n\pi}{\ell}\right)^2}u_n(t)] = e^{tk\left(\frac{n\pi}{\ell}\right)^2}P_n(t) \\
    & \implies e^{tk\left(\frac{n\pi}{\ell}\right)^2}u_n(t) = u_n(0) + \int_0^t e^{sk\left(\frac{n\pi}{\ell}\right)^2}P_n(s) \ ds \\
    & \implies u_n(t) = e^{-tk\left(\frac{n\pi}{\ell}\right)^2}u_n(0) + e^{-tk\left(\frac{n\pi}{\ell}\right)^2}\int_0^t e^{sk\left(\frac{n\pi}{\ell}\right)^2}P_n(s) \ ds 
\end{align*}

where $u_n(0) = \frac{2}{\ell}\int_0^\ell \phi(x)\sin\left(\frac{n\pi x}{\ell}\right) \ dx$.

If we take $f = h = j = 0$, then $P_n(t) = 0$, and get a much more simplified equation. 

This process works for any case where separation of variables was used, including Neumann B.C, periodic B.C, the wave equation, Schrodinger's Equation, etc.
\chapter{Week 10}

\section{The Laplace Equation}

We now pivot towards studying the Laplace equation. In dimensions 1,2, and 3, the Laplace equation is written as

\[ u_{xx} = 0 \quad u_{xx} + u_{yy} = 0 \quad u_{xx} + u_{yy} + u_{zz} = 0 \]

We may also write this equation as 

\[ \Delta u = 0 \]

where $\Delta$ is the \textbf{Laplacian} operator. 

This equation deals with situations where a solution is in an \textit{equilibrium/stationary} state, that is, they are independent of time. Take for example the heat equation in 1 dimension,

\[ u_t - ku_{xx} = 0 \]

If our solution is in a stationary state, then it is independent of time, meaning the $u_t$ is gone, leaving us with the 1 dimensional Laplace equation. 

In a sense, the Laplace equation is the most relevant PDE in mathematical physics, playing an important role in electrostatics, steady fluid flow, and Brownian motion, among other areas. Studying this equation in detail - like we did for the heat and wave equations - is somewhat outside the scope of the course, but we can still look at it and understand some interesting properties.

\subsection{The Maximum/Minimum Principle}

Like the heat equation, the Laplace equation has a Max/Min Principle, though, it is a little more complex compared to before. 

Our domain is now not a square of values, but rather any open set $D \subset \R^2$ with a ``nice" boundary (``nice" just means that it doesn't cause us any problems). We suppose that, inside of $D$,

\[ \Delta u = 0 \]

and that $u$ extends continuously to $\partial D$. We know that $D \cup \partial D$ is a closed and bounded subset of $\R^2$, so the Extreme Value Theorem tells us that $u$ attains a maximum and minimum on $D$. Take a wild guess where they are...

\begin{theorem}[The Maximum/Minimum Principle for Laplace Equations]
    Let $D \subset \R^2$ and $u$ be as above, then

    \[ \max_D u = \max_{\partial D}u \quad \text { and } \quad \min_D u = \min_{\partial D} u \]
\end{theorem}

One may notice that this is somewhat related to the 2nd Derivative Test: If a critical point likes within the interior, the $\nabla u = 0$. If that point is a strict extremum, then $\nabla^2 u$ has all positive or negative eigenvalues (depending on if it's a max or min). Now $\Delta = \operatorname{tr}\nabla^2 u$. However, if we have a strict max/min, this value will be positive or negative, and not 0, meaning the point cannot be an extremum. 

\subsubsection*{Uniqueness of the Dirichlet Problem}

The Maximum/Minimum Principle can allow us to prove, like in studying the heat equation, that the Laplace equation has unique solutions in certain scenarios. Let's consider the Laplace equation with Dirichlet boundary conditions. We let $D$ be a bounded, open subset of $\R^2$ and let

\[ h: \partial D \to \C \]

be a smooth function, and $F: D \to \C$ be a smooth too. We seek to find a solution to

\[ \begin{cases}
    u: D \to \C \\ \Delta u = F \\ u|_{\partial D} = h
\end{cases} \]

For simplicity, we can take $F = 0$. 

\begin{prop}
    The above problem has at most 1 solution.
\end{prop}

\begin{proof}
    Suppose that both $u_1, u_2$ solve the problem. We define

    \[ w = u_1 - u_2 \]

    We see that

    \begin{align*}
        \Delta w & = \Delta(u_1 - u_2) \\
        & = \Delta u_1 - \Delta u_2 \\ 
        & = 0 \\
        w|_{\partial D} & = (u_1 - u_2)|_{\partial D} \\
        & = u_1|_{\partial D} - u_2|_{\partial D} \\
        & = 0
    \end{align*}

    so the problem for $w$ is

    \[ \begin{cases}
        \Delta w = 0 \\
        w|_{\partial D} = 0
    \end{cases} \]

    so the Maximum/Minimum Principle applies. We get that

    \[ \max_D w = \max_{\partial D} w = 0 \]
    \[ \min_D w = \min_{\partial D} w = 0 \]

    hence $w = 0$, meaning $u_1 = u_2$.
\end{proof}

\subsection{Invariance Under Rigid Motion}

Another important property of the Laplace equation is that is invariant under \textit{rigid motions}, that is, the symmetries of the plane. There are 2 such symmetries:

\begin{enumerate}[label=(\roman*)]
    \item Translation: $(x,y) \mapsto (x+a, y+b)$ for constants $a,b$. 
    \item Rotation: $(x,y) \mapsto (x\cos\alpha + y\sin\alpha, x\sin\alpha + y\cos\alpha)$ for some $\alpha \in[0,2\pi)$.
\end{enumerate}

What's special about this is that $\Delta$ is the only differential operator with this property. This makes is perfect for studying \textit{isotropic} situations in engineering, where there is no preferred direction. 

The fact that we have rotational invariance suggests that the Laplace equation would have a simpler form when converted to \textit{polar coordinates}. Let's compute it. Given polar coordinates $(r, \theta)$, we have

\[ x = r\cos\theta, \quad y = r\sin \theta \]
\[ r = \sqrt{x^2 + y^2}, \quad \theta = \arctan\left(\frac{y}{x}\right) \]


\chapter{Week 11}

\section{The Laplace Equation, Continued}

\subsection{Poisson's Formula}

Recall that our solution from last week for the Dirichlet problem on a ball of radius 1 was written as 

\[ u(r,\theta) = \sum_{n=-\infty}^\infty A_nr^{|n|}e^{in\theta} \]

with 

\[ A_n = \frac{1}{2\pi}\int_{-\pi}^\pi h(\theta)e^{-in\theta} \ d\theta \]

We can rewrite our solution in a different way that will lead to a much more powerful formula and a better visualizer of what is actually going on. To do this, let's plug in the formula for $A_n$ into the formula for $u$:

\[ u = \sum_{n=-\infty}^\infty \left[\frac{1}{2\pi}\int_{-\infty}^\infty h(\phi)e^{-in\phi} \ d\phi\right]r^{|n|}e^{i\theta} \]

Swapping the sum and integral yields

\[ u = \frac{1}{2\pi}\int_{-\infty}^\infty \sum_{n=-\infty}^\infty\left[r^{|n|}e^{in(\theta - \phi)}\right]h(\phi) \ d\phi \]

Notice that the $r^{|n|}$ terms give us a geometric series. This is because, as $x^2 + y^2 < 1$ in the interior, so must $r$, and thus $|re^{\pm i (\theta - \phi)}| < 1$. We get that

\begin{align*}
    \sum_{n=-\infty}^\infty r^{|n|}e^{in(\theta - \phi)} & = 1 + \sum_{n=1}^\infty r^n e^{in(\theta - \phi)} + \sum_{n=-\infty}^-1 r^{-n}e^{in(\theta - \phi)} \\
    & = 1 + \sum_{n=1}^\infty (re^{i(\theta - phi)})^n + \sum_{n=1}^\infty (re^{-i(\theta - \phi)})^n \\
    & = 1 + \frac{re^{i(\theta - \phi)}}{1 - re^{i(\theta - \phi)}} + \frac{re^{-i(\theta - \phi)}}{1 - re^{-i(\theta - \phi)}} \\
    & = \frac{1 - re^{-i(\theta - \phi)} - re^{i(\theta - \phi)} + r^2 + re^{i(\theta - \phi)} - r^2 + re^{-i(\theta - \phi)} - r^2}{1 - re^{-i(\theta - \phi)} - re^{i(\theta - \phi)} + r^2} \intertext{By applying $re^{i\theta} = r\cos(\theta) + ri\sin(\theta)$, we get} \\
    & = \frac{1 - 2r\cos(\theta - \phi) + r^2 + 2r\cos(\theta - \phi) - r^2}{1 - 2r\cos(\theta - \phi) + r^2} \\
    & = \frac{1 - r^2}{1 - 2r\cos(\theta - \phi) + r^2} 
\end{align*}

Thus, we have that 

\[ u(r,\theta) = \frac{(1 - r^2)}{2\pi}\int_{-\pi}^\pi \frac{h(\phi)}{1 - 2r\cos(\theta - \phi) + r^2} \ d\phi \]

This formula can be generalized to circles of any radius $a > 0$, giving us

\[ u(r,\theta) = \frac{(a - r^2)}{2\pi}\int_{-\pi}^\pi \frac{h(\phi)}{a - 2ar\cos(\theta - \phi) + r^2} \ d\phi \]

In this form, we have what is known as \textbf{Poisson's Formula}. The special thing about this formula in comparison to the previous equation is that it shows how the value of a point inside the circle can be computed using exclusively the values on the boundary of the circle. To see this, we can rewrite this further using a more geometric argument. 

Let $\vec{x} \in B_a(0)$ be in the interior of the ball of radius $a$, and consider the figure below:

\begin{center}
    \includegraphics[width=0.5\textwidth]{Week 11/Poisson_formula_circle.png}
\end{center}

The Law of Cosines says that

\[ |\vec{x} - \vec{x'}|^2 = a^2 + r^2 - 2ar\cos(\theta - \phi) \]

Furthermore, we write $r = |\vec{x}|^2$, and we are now integrating over all $|\vec{x'}|^2 = a$. The value $h(\phi)$ now becomes $u(\vec{x'})$ since it corresponds to evaluating $u$ on the boundary. Finally, our change of variables means that $d\phi = a \ ds'$. Thus, we get that

\[ u(\vec{x}) = \frac{a^2 - |\vec{x}|^2}{2\pi a}\int_{|\vec{x'}|^2 = a} \frac{u(\vec{x'})}{|\vec{x} - \vec{x'}|^2} \ ds' \]

\subsubsection*{The Maximum Principle}

We can use Poisson's Formula to prove the Maximum Principle for the Laplace Equation. Using the geometric version of the formula, take $\vec{x} = 0$. Then we have that

\begin{align*}
    u(\vec{x}) & = u(\vec{0}) \\
    & = \frac{a^2}{2\pi a}\int_{|\vec{x'}| = a} \frac{u(\vec{x'})}{|\vec{x'}|^2} \ ds' \\
    & = \frac{a^2}{2\pi a}\int_{|\vec{x'}| = a} \frac{u(\vec{x'})}{a^2} \ ds' \\
    & = \frac{1}{2\pi a}\int_{|\vec{x'}| = a} u(\vec{x'}) \ ds'
\end{align*}

The astute reader, or simply anyone whose read a complex analysis book, may notice that this is eerily familiar to the Cauchy Integral Formula! Indeed, they are the same, both indicating that the value of $u$ at $\vec{0}$ is equal to the average value of $u$ along a circle of radius $a$ around the zero vector. 

The final piece of the puzzle is to notice that if $u(\vec{x})$ is harmonic, then so is $u(\vec{x} + \vec{x_0})$ for any vector $\vec{x_0}$. This amounts to just shifting the circle around until it is centered at $\vec{x} + \vec{x_0}$. This is the mathematical way of saying that a system is in equilibrium. If the values in a system can be computed using the values around it by taking their average, then clearly that system isn't moving or changing over time. 

Let's now apply this to the Maximum Principle. Our proof will be a rough sketch that requires some additional details to be complete. The curious reader is encouraged to seek out these final details in an analysis class. 

We let $\Omega$ be an open and bounded domain such that $\Delta u = 0$ inside $\Omega$. Suppose that there is a point $p$ in the interior of $\Omega$ for which $u$ attains its maximum. By the above discussion, $u(p)$ is the average of $u$ along a circle $C$ of radius $a$. But $p$ is the maximum, so it must be larger than the average. The only way both statements are true is if $u$ also attains its maximum at every point in $C$. This works for circles of any radius, so $u$ attains its maximum at every point in the \textit{disc} centered at $p$ with radius $a$. We can then keep drawing circles until we eventually cover $\Omega$ with a finite number of them, at which point one must intersect with the boundary of $\Omega$; any point in the intersection will be where $u$ attains its maximum, and this point is on $\partial\Omega$, as desired. 

\subsection{The Dirichlet Problem in the Exterior of a Ball}

Let's now consider the opposite problem from the ball: our domain is now everything \textit{except} for a ball.

\[ \begin{cases}
    \Delta u = 0 \quad x^2 + y^2 > 1 \\ u|_{x^2 + y^2 = 1} = h(\theta)
\end{cases} \]

We begin with our \textit{master formula} we derived last week:

\[ u = A_0 + B_0\log(r) + \sum_{\substack{n=-\infty \\ n \neq 0}}^\infty (A_nr^{|n|} + B_nr^{-|n|})e^{in\theta} \]

But we've run into a problem. In the region we're working on, $r > 1$, meaning everything is well-defined and we cannot remove any constants. Not all hope is lost, however, as there is some information we know. Applying the boundary condition yields 

\[ u|_{r = 1} = h(\theta) = A_0 + \sum_{\substack{n=-\infty \\ n \neq 0}}^\infty (A_n + B_n)e^{in\theta} \]

So $A_0$, $A_n + B_n$ satisfy the equations for $h$'s Fourier coefficients in the full Fourier series:

\[ A_0 = \frac{1}{2\pi}\int_{-\pi}^\pi h(\theta) \ d\theta \quad A_n + B_n = \frac{1}{2\pi}\int_{-\pi}^\pi h(\theta)e^{-in\theta} \ d\theta \]

However, this doesn't give us a unique solution. To rectify this, we need to add an additional boundary condition \textit{at infinity}. Let's say that $u$ remains bounded as $r \to \infty$. Then because $\log(r) \to \infty$ and $r^{|n|} \to \infty$ as $r \to \infty$, we require that $B_0 = 0$ and $A_n = 0$ for all $n \neq 0$. This now gives us a unique solution given by 

\[ A_n = \frac{1}{2\pi}\int_{-\pi}^\pi h(\theta)e^{-in\theta} \ d\theta \]

This idea of including boundary conditions at infinity is not a new one. In fact, it was implicitely used in our study of the heat equation. All of our solutions went to 0 as $x$ went to $\pm \infty$. 

\subsection{The Dirichlet Problem in an Annulus}

We now shift focus to solving the Laplace Equation in an \textbf{annulus}, which may be thought of in this context as a disc with a hole cut out of it. Given $0 < a < b < \infty$, the problem becomes

\[ \begin{cases}
    \Delta u = 0 \quad a^2 < x^2+ y^2 < b^2 \\
    u|_{r=a} = h(\theta), \quad u|_{r=b} = g(\theta)
\end{cases} \]

Again, since $r$ is always greater than 0, we cannot remove any terms from the master solution, but applying the boundary conditions does give information that can help us get to a unique solution. We have 

\[ h(\theta) = u|_{r=a} = A_0 + B_0\log(a) + \sum_{\substack{n=-\infty \\ n\neq 0}}^\infty (A_n a^{|n|} + B_na^{-|n|})e^{in\theta} \]
\[ g(\theta) = u|_{r=b} = A_0 + B_0\log(b) + \sum_{\substack{n=-\infty \\ n\neq 0}}^\infty (A_n b^{|n|} + B_nb^{-|n|})e^{in\theta} \]

From this, we get two systems of equations:

\begin{align*}
    A_0 + B_0\log(a) & = h_0 \\
    A_0 + B_0\log(b) & = g_0
\end{align*}

\begin{align*}
    A_na^n + B_na^{-n} & = h_n \\
    A_nb^n + B_nb^{-n} & = g_n
\end{align*}

corresponding to the matrices

\[ \begin{bmatrix}
        1 & \log(a) \\ 1 & \log(b) 
    \end{bmatrix}\begin{bmatrix}
        A_0 \\ B_0
    \end{bmatrix} \quad \text{ and } \quad \begin{bmatrix}
        a^n & a^{-n} \\ b^n & b^{-n}
    \end{bmatrix}\begin{bmatrix}
        A_0 \\ B_0
    \end{bmatrix} \]  

These indeed have unique solutions since, for the first matrix, $\log(a) \neq \log(b)$, and for the second one, 

\[ \frac{a^n}{b^n} \neq \frac{b^n}{a^n} \]

\section{Fourier Transform}

Earlier we introduced Fourier Series, where the coefficients had discrete inputs in $\Z$. Fourier Transform is very similar, but allows for continuous input values. 

\subsection{Definition and the Inversion Formula}

We consider a function $f: \R \to \C$ such that

\[ \int_{-\infty}^\infty |f(x)| \ dx < \infty \]

Then we may define the \textbf{Fourier Transform} of $f$ as the map $\hat{f}: \R \to \C$ given by 

\[ \hat{f}(\xi) = \int_{-\infty}^\infty e^{-ix\xi}f(x) \ dx \]

Alternatively, we can define a function that sends $f$ to its Fourier transform

\[ \mathcal{F}[f](\xi) = \int_{-\infty}^\infty e^{-ix\xi} f(\xi) \ d\xi = \hat{f}(\xi) \]

This is almost identical to full Fourier series, only this time we can plug in any real number. This requires us to use an integral instead of a sum. 

One of the benefits of Fourier Transform, is that, like Fourier Series, an analytic operation on the original function corresponds to an algebraic operation on its Fourier Transform, and vice versa. To see this, we first introduce a fundamental theorem:

\begin{theorem}[Fourier Inversion Theorem]
    Suppose $f: \R \to \C$ such that 

    \[ \int_{-\infty}^\infty |f(x)| \ dx < \infty \quad \text{ and } \quad \int_{-\infty}^\infty |\hat{f}(\xi)| \ d\xi < \infty \]

    Then we have that

    \[ f(x) = \frac{1}{2\pi}\int_{-\infty}^\infty e^{ix\xi}\hat{f}(\xi) \ d\xi \]
\end{theorem}

This is analagous to the formula for Fourier Series:

\[ f(\theta) = \sum_{n=-\infty}^\infty \hat{f}(n)e^{in\theta} \]

Using this formula, we can see that

\begin{align*}
    \dif{}{x}f & = \frac{1}{2\pi}\int_{-\infty}^\infty \pardif{}{x}\left[e^{ix\xi}\hat{f}(\xi)\right] \ d\xi \\
    & = \frac{1}{2\pi}\int_{-\infty}^\infty i\xi e^{ix\xi}\hat{f}(\xi) \ d\xi
\end{align*}

Using the $\mathcal{F}$ function, this says that

\[ \mathcal{F}[\partial_xf](x) = (i\xi)\hat{f}(\xi) \]

We can also go backwards. We have

\begin{align*}
    \mathcal{F}[-ixf](\xi) & = \int_{-\infty}^\infty e^{-ix\xi}(-ix)f(x) \ dx \\
    & = \pardif{}{\xi}\int_{-\infty}^\infty e^{ix\xi}f(x) \ dx \\
    & = \pardif{}{\xi}\hat{f}(\xi)
\end{align*}

So $\partial_\xi \hat{f}(\xi) = \mathcal{F}[-ixf](\xi)$. 

\subsection{Schwartz Space}

There is a special class of functions which the Fourier Transform respects, thus making this class really useful for our purposes. 

We define the \textbf{Schwartz Space} $\mathcal{S}(\R)$ to be the set of all functions $f: \R \to \C$ such that $f$ is smooth, and for all positive integers $i,j$,

\[ \max_{x \in \R} |x|^i|\partial_x^j f| < C(i,j) \]

for some constant $C$ that may or may not depend on $i,j$. In other words, $\mathcal{S}(\R)$ is the set of all functions that, alongside all derivatives, goes to 0 faster than any polynomial. 

\begin{example}
    \begin{enumerate}[label=(\roman*)]
        \item If $f$ is smooth and compactly supported, then $f \in \mathcal{S}(\R)$
        \item $f(x) = e^{-x^2} \in \mathcal{S}(\R)$.
    \end{enumerate}
\end{example}

A non-example is the function $f(x) = x$. 

The reason that $\mathcal{S}(\R)$ is good for our purposes is found in the following theorem:

\begin{theorem}
    If $f \in \mathcal{S}(\R)$, then so is $\hat{f}$. 
\end{theorem}

This also tells us that if $f \in \mathcal{S}(\R)$, then 

\[ \int_{-\infty}^\infty |f| \ dx < \infty \]

Finally, we have that if $\mathcal{F}[f] = f = \hat{f}$, we get 4 equations:

\[ \mathcal{F}[\partial_x f](\xi) = i\xi \hat{f}(\xi) \]
\[ \mathcal{F}[\partial_\xi \hat{f}](x) = -ixf(x) \] 
\[ \mathcal{F}[ix f](\xi) = -\partial_\xi \hat{f}(\xi) \]
\[ \mathcal{F}[i\xi \hat{f}](x) = -\partial_x f(x) \]  

\subsection{Solving PDEs}

Solving certain PDEs using Fourier Transform can be quite simple in the event that certain functions are in Schwartz Space. We demonstrate some examples of this here. 

\subsubsection{The Heat Equation}

Suppose we wish to solve the heat equation on an infinite rod.

\[ \begin{cases}
    \partial_t u - \partial_x^2 u = 0 \quad x \in (-\infty, \infty) \\
    u(x,0) = \phi(x)
\end{cases} \]

and suppose that $\phi(x) \in \mathcal{S}(\R)$. Let $\hat{u}$ be the Fourier Transform of $u$. As the PDE is equal to 0, so too must the Fourier Transform of the PDE. We will take the Transform with respect to $x$.

\[ \mathcal{F}_x[\partial_t u - \partial_x^2 u] = 0 \implies \mathcal{F}_x[\partial_t u] - \mathcal{F}_x[\partial_x^2 u] = 0 \]

Computing, we have that 

\begin{align*}
    \mathcal{F}_x[\partial_t u] & = \int_{-\infty}^\infty e^{-ix\xi}\pardif{}{t}u(x,t) \ dx \\
    & = \pardif{}{t}\int_{-\infty}^\infty e^{-ix\xi}u(x,t) \ dx \\
    & = \pardif{\hat{u}}{t}(\xi,t)  \intertext{By the properties of Fourier Transform described above, we get that}\\
    \mathcal{F}_x[\partial_x^2 u] & = (i\xi)\mathcal{F}_x[\partial_x u] \\
    & = (i\xi)^2 \hat{u} \\
    & = -\xi^2 \hat{u}
\end{align*}

Thus,

\[ \pardif{\hat{u}}{t}(\xi,t) + \xi^2 \hat{u}(\xi,t) = 0 \]

This is an ODE which has a solution given by 

\[ \hat{u}(\xi,t) = A(\xi)e^{-t\xi^2} \] 

One can think of this as the solution $A_ne^{-tn^2}$ from Fourier Series. To solve for $A(\xi)$, we see that 

\[ A(\xi) = \hat{u}(\xi,0) = \hat{\phi}(\xi) \]

so we get that $u(\xi,t) = \hat{\phi}(\xi)e^{-t\xi^2}$. As $\hat{\phi} \in \mathcal{S}(\R)$, and $e^{-t\xi^2}$ is exponential, we get that $u(\xi,t) \in \mathcal{S}(\R)$, meaning the we can use Fourier Inversion to conclude that 

\begin{align*}
    u(x,t) & = \frac{1}{2\pi}\int_{-\infty}^\infty e^{ix\xi}\hat{u}(\xi,t) \ d\xi \\
    & = \frac{1}{2\pi}\int_{-\infty}^\infty e^{ix\xi - t\xi^2}\hat{\phi}(\xi) \ d\xi
\end{align*}

If you expand and simplify this equation, you will see that this is identical to the derivation we did many weeks ago. 

\subsubsection{The Wave Equation}

Sometimes this method can produce strange results. Consider solving the wave equation 

\[ \begin{cases}
    \partial_t^2 u - \partial_x^2 u = 0 \\
    u(x,0) = \phi(x), \quad \partial_t u (x,0) = \psi(x)
\end{cases} \]

with $\phi, \psi \in \mathcal{S}(\R)$. Using the same process, we get that 

\begin{align*}
    & \mathcal{F}_x[\partial_t^2 u - \partial_x^2 u] = 0 \\
    \implies & \frac{\partial^2}{\partial t^2}\hat{u}(\xi,t) - (i\xi)^2 \hat{u}(\xi,t) = 0 \\
    \implies & \partial_t^2 \hat{u} + \xi^2 \hat{u} = 0 \\
    \implies & \hat{u}(\xi,t) = A(\xi)\cos(\xi t) + B(\xi)\sin(\xi t)
\end{align*}

With $A(\xi) = \hat{u}(\xi,0) = \hat{\phi}(\xi)$. Moreover, 

\begin{align*}
    \hat{\psi}(\xi) & = \mathcal{F}_x[\partial_t u]\bigg|_{t = 0} \\
    & = \pardif{}{t}[A(\xi)\cos(\xi t) + B(\xi)\sin(\xi t)]\bigg|_{t = 0} \\
    & = \xi(B(\xi))
\end{align*}

meaning $B(\xi) = \dfrac{\hat{\psi}(\xi)}{\xi}$. This presents us with a problem when $\xi = 0$. What this means is that, since we already derived the solutions to this equation previously, this method isn't necessary. 

\subsubsection{Another Example}

Consider the problem 

\[ \begin{cases}
    \partial_t u + \partial_x^4 u = 0 \\ u(x,0) = \phi(x)
\end{cases} \] 

where $\phi$ is smooth and compactly supported, so it's in $\mathcal{S}(\R)$. We have that 

\begin{align*}
    & \mathcal{F}_x[\partial_t u + \partial_x^4 u] = 0 \\
    \implies & \mathcal{F}_x[\partial_t u] + \mathcal{F}_x[\partial_x^4] = 0 \\
    \implies & \pardif{\hat{u}}{t}(\xi,t) + (i\xi)^4\hat{u}(\xi,t) = 0 \\
    \implies & \partial_t \hat{u}(\xi,t) + \xi^4 \hat{u}(\xi,t) = 0 \\
    \implies & \hat{u}(\xi,t) = A(\xi)e^{-t\xi^4}
\end{align*}

where $A(\xi) = \hat{\phi}(\xi)$. As $\hat{u} \in \mathcal{S}(\R)$, by Fourier Inversion, we get 

\[ u(x,t) = \frac{1}{2\pi}\int_{-\infty}^\infty e^{ix\xi - t\xi^4}\hat{\phi}(\xi) \ d\xi \] 
\chapter{Week 12}

\section{Equivalence of Heat Equation Solutions}

To wrap up our discussion on the Fourier Transform, and to conclude our discussion on PDEs as a whole, we will show that the solutions to the heat equation derived using the heat kernel, and the solutions derived using Fourier Transform, are equivalent. In particular, recall that the problem

\[\begin{cases}
    u_t - ku_{xx} = 0 \quad x \in (-\infty,\infty) \\
    u(x,0) = \phi(x)
\end{cases}\]

has a solution given by 

\[u(x,t) = \frac{1}{\sqrt{4\pi t}}\int_{-\infty}^\infty e^{\frac{-(x-y)^2}{4t}}\phi(y) \ dy \]

We also know that, using Fourier Transform,

\[u(x,t) = \frac{1}{2\pi}\int_{-\infty}^\infty e^{ix\xi - t\xi^2}\hat{\phi}(\xi) \ d\xi\]

where $\hat{\phi}(xi) = \mathcal{F}[\phi](\xi) = \int_{-\infty}^\infty e^{ix\xi} f(x) \ dx$. 

Before we can show they are equivalent, we first need to understand how Fourier Transform interacts with certain operations. 

\subsection*{Dilation}

Given an $a > 0$, we define the \textbf{dilation} function $D_a: \R \to \R$ as the map $x \mapsto ax$. For any function $f: \R \to \C$, the dilated version of $f$ is the map $f \circ D_a$. In other words, the point $f(x)$ is sent to $f(ax)$. 

The Fourier Transform of a dilated function has a nice relationship to the Fourier Transform of the orignal function. We have 

\begin{align*}
    \mathcal{F}[f(ax)](\xi) & = \int_{-\infty}^\infty e^{-ix\xi}f(ax) \ dx \intertext{Setting $y = ax, dy = a \ dx$, we get} \\
    & = \int_{-\infty}^\infty e^{-i\frac{y}{a}\xi}f(y) \frac{dy}{a} \\
    & = \frac{1}{a}\int_{-\infty}^\infty e^{-iy\left(\frac{\xi}{a}\right)}f(y) \ dy \\
    & = \frac{1}{a}\mathcal{F}[f](\xi/a)
\end{align*}

Thus, dilating $f$ results in a sort of ``reverse dilation'' of $\hat{f}$. 

\subsection*{Convolution}

Recall that for functions $f(x),g(x)$, their convolution is given by

\[(f*g)(x) = \int_{-\infty}^\infty f(x-y)g(y) \ dy\]

This operator is both commutative and associative. What's nice about convolution is that, under Fourier Transform, it turns into a pointwise product:

\begin{theorem}
    If $f,g \in \mathcal{S}(\R)$, 

    \[\mathcal{F}[f * g](\xi) = \mathcal{F}[f](\xi) \cdot \mathcal{F}[g](\xi)\]
\end{theorem}

\begin{proof}
    We have 

    \begin{align*}
        \int_{-\infty}^\infty e^{-ix\xi}(f*g)(x) \ dx & = \int_{-\infty}^\infty e^{-ix\xi}\left[\int_{-\infty}^\infty f(x-y)g(y) \ dy\right] \ dx \intertext{Switching the order of integration,} \\
        & = \int_{-\infty}^\infty \left[\int_{-\infty}^\infty e^{-ix\xi}f(x-y) \ dx\right] g(y) \ dy \intertext{Setting $z = x-y, dx = dx$, we get} \\
        & = \int_{-\infty}^\infty \left[\int_{-\infty}^\infty e^{-i(z+y)\xi}f(z) \ dz\right] g(y) \ dy \\
        & = \int_{-\infty}^\infty \left[\int_{-\infty}^\infty e^{-iz\xi}f(z) \ dz\right]e^{-iy\xi}g(y) \ dy \\
        & = \int_{-\infty}^\infty \hat{f}(\xi) e^{-iy\xi}g(y) \ dy \\
        & = \hat{f}(\xi) \int_{-\infty}^\infty e^{-iy\xi}g(y) \ dy \\
        & = \hat{f}(\xi) \cdot \hat{g}(\xi)
    \end{align*}
\end{proof}

\subsection*{Fourier Transform of the Gaussian}

The last thing we do before showing equivalence is compute the Fourier Transform of the Gaussian function $e^{-x^2}$. Becuase of the non-elementary nature of the integral of the Gaussian over $\R$, solving this will require some trickery: First we write $H(\xi) = \mathcal{F}[e^{-x^2}](\xi)$. Then it follows that 

\[H(\xi) = \int_{-\infty}^\infty e^{-ix\xi}e^{-x^2} \ dx\]

Taking the derivative of $H$, we get the following:

\begin{align*}
    H'(\xi) & = \dif{}{\xi}\int_{-\infty}^\infty e^{-ix\xi}e^{-x^2} \ dx \\
    & = \int_{-\infty}^\infty \left[\pardif{}{\xi}e^{-ix\xi}\right]e^{-x^2} \ dx \\
    & = \int_{-\infty}^\infty (-ix)e^{-ix\xi}e^{-x^2} \ dx \intertext{As $\dif{}{x}(e^{-x^2}) = -2xe^{-x^2}$, we can rewrite this as} \\
    & = \frac{i}{2}\int_{-\infty}^\infty e^{-ix\xi}\left(\dif{}{x}e^{-x^2}\right) \ dx \\
    & \overset{\text{IBP}}{=} \frac{i}{2}e^{-ix\xi}e^{-x^2}\bigg|_{-\infty}^\infty - \frac{i}{2}\int_{-\infty}^\infty \left(\pardif{}{x} e^{-ix\xi}\right)e^{-x^2} \ dx \\
    & = -\frac{i}{2}\int_{-\infty}^\infty \left(\pardif{}{x} e^{-ix\xi}\right)e^{-x^2} \ dx \\
    & = -\frac{i}{2}\int_{-\infty}^\infty (-i\xi)e^{-ix\xi}e^{-x^2} \ dx \\
    & = -\frac{\xi}{2}\int_{-\infty}^\infty e^{-ix\xi}e^{-x^2} \ dx \\
    & = -\frac{\xi}{2}H(\xi)
\end{align*}

Thus, we get an ODE 

\[\dif{H}{\xi} + \frac{\xi}{2}H = 0\]

which we know has a solution given by 

\[H(\xi) = A\exp\left(-\frac{\xi^2}{4}\right)\]

for a constant $A$. What is $A$?

\[A = H(0) = \int_{-\infty}^\infty e^{-x^2} \ dx = \sqrt{\pi}\]

so we conclude that $H(\xi) = \sqrt{\pi}\exp\left(-\frac{\xi^2}{4}\right)$.

\subsection*{Showing Equivalence}

We are now ready to show that the two types of solutions are in fact equivalent. We start with 

\[u(x,t) = \frac{1}{2\pi}\int_{-\infty}^\infty e^{ix\xi - t\xi^2}\hat{\phi}(\xi) \ d\xi\]

and we let $\hat{u}(x,t) = \mathcal{F}_x[u](\xi,t)$ be the Fourier Transform of $u$ in $x$. By Fourier Inversion, we get that 

\[\hat{u}(\xi,t) = e^{-t\xi^2}\hat{\phi}(\xi)\]

Now, we know from before that for a convolution $f * g$, 

\[\mathcal{F}[f*g](\xi) = \hat{f}(\xi) \cdot \hat{g}(\xi)\]

Thus,

\[(f*g)(x) = \mathcal{F}^{-1}[\hat{f}(\xi)\cdot\hat{g}(\xi)]\]

We can wrtie this in a more suggestive way by setting $h(\xi) = \hat{f}(\xi), k(\xi) = \hat{g}(\xi)$. Then it follows that 

\[(\mathcal{F}^{-1}(h)*\mathcal{F}^{-1}(k)) = \mathcal{F}^{-1}(h(\xi) \cdot k(\xi))\]

Using this on $u$, we get that 

\[u(x,t) = (\mathcal{F}^{-1}(e^{-t\xi^2}) * \mathcal{F}^{-1}(\hat{\phi}(\xi)))(x) = (\mathcal{F}^{-1}(e^{-t\xi^2})*\phi)(x)\]

So we just need to determine what $\mathcal{F}^{-1}(e^{-t\xi^2})$ is. Recall that 

\[\mathcal{F}[e^{-x^2}] = \sqrt{\pi}e^{-\frac{\xi^2}{4}}\]

As $-\dfrac{\xi^2}{4} = -\left(\dfrac{\xi}{2}\right)^2$, we need to turn $\dfrac{\xi}{2}$ into $\sqrt{t}\xi$. Doing this requires a multiplication by $2\sqrt{t}$, or a division by $\dfrac{1}{2\sqrt{t}}$. It follows by our discussion on dilations that 

\[\mathcal{F}\left[e^{-\left(\frac{1}{2\sqrt{t}}x\right)^2}\right] = 2\sqrt{t}\sqrt{\pi}e^{-t\xi^2} = \sqrt{4\pi t}e^{-t\xi^2}\]

Thus, 

\[\mathcal{F}^{-1}(e^{-t\xi^2}) = \frac{1}{\sqrt{4\pi t}}e^{\frac{-x^2}{4t}}\]

and we get that 

\[u(x,t) = \frac{1}{\sqrt{4\pi t}}\int_{-\infty}^\infty e^{\frac{-(x-y)^2}{4t}}\phi(y) \ dy\]

which is precisely what we got when we used the heat kernel!

\chapter*{Beyond PDEs}

Our disucssion of PDEs has been, in the broader context of the subject, somewhat restricted to a small class of really nice PDEs and certain PDE techniques. There are plenty of more interesting things to discover in this field. Reading deeper into the course textbook (which was mentioned in the opening remarks of these notes) is a good start, as it not only contains a lot of other PDE related topics, but also goes into the theory as to why certain techniques, which we took for granted, work. 

For those who wish to see some related concepts outside of PDEs, I would recommend looking into the field of \textit{numerical analysis}, which study algorithms that can approximate things like solutions to polynomials, ODEs \& PDEs, etc. Some relevant courses for the interested reader at UofT include 

\begin{itemize}
    \item MAT264H5: Introduction to Numerical Analysis 
    \item CSC336H1: Numerical Methods
    \item CSC436H1: Numerical Algorithms
    \item CSC456H1: High-Performance Scientific Computing
    \item CSC466H1: Numerical Methods for Optimization Problems
\end{itemize}

while a lot of these courses are tailored towards computer science students and do require some knowledge of computers and coding, they're still quite interesting and might be the place to go if you want to see how PDEs are used in applications. 

For those who are more interested in PDE theory, 2 recommended courses at UTM include MAT357H1: Foundations of Real Analysis, and MAT436H1: Introduction to Linear Operators, both of which have sections dedicated to the study of PDEs and the theory surrounding them. 

\end{document}
