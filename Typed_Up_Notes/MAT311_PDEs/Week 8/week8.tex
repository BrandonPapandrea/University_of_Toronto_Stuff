\chapter{Week 8}

\section{Fourier Transform, Continued}

\subsection{Full Fourier Series, Continued}

Towards the end of last week, we introduced the full Fourier series

\[ \phi(x) = \frac{1}{2}A_0 + \sum_{n=1}^\infty \left(A_n\cos\left(\frac{n\pi x}{\ell}\right) + B_n\sin\left(\frac{n\pi x}{\ell}\right)\right) \]

This week we will find those coefficients $A_n,B_n$. 

To do this we need to 3 integrals to vanish:

\begin{enumerate}
    \item $\int_{-\ell}^\ell \sin\left(\frac{n\pi x}{\ell}\right)\cos\left(\frac{m\pi x}{\ell}\right) \ dx = 0$ 
    \item $\int_{-\ell}^\ell \sin\left(\frac{n\pi x}{\ell}\right)\sin\left(\frac{m\pi x}{\ell}\right) \ dx = 0$
    \item $\int_{-\ell}^\ell \cos\left(\frac{n\pi x}{\ell}\right)\cos\left(\frac{m\pi x}{\ell}\right) \ dx = 0$
\end{enumerate}

To see why they vanish, recall that if a function $f(x)$ is \textit{odd}, then

\[ \int_{-\ell}^\ell f(x) \ dx = 0 \]

For all $n,m$, we have that

\[ \sin\left(\frac{n\pi x}{\ell}\right)\cos\left(\frac{m\pi x}{\ell}\right)\]

is odd; the product of an odd function, $\sin$, and an even function, $\cos$, is always odd. Thus, its integral from $-\ell$ to $\ell$ will vanish. 

For the other two functions, they are either the product of two even functions, as seen with $\cos$, or two odd functions, as seen with $\sin$, meaning both are even functions. This is not a bad thing, though, as for any even function $f(x)$,

\[ \int_{-\ell}^\ell f(x) \ dx = 2\int_0^\ell f(x) \ dx \]

Thus, we get that for $n \neq m$,

\[ \int_{-\ell}^\ell \cos\left(\frac{n\pi x}{\ell}\right)\cos\left(\frac{m\pi x}{\ell}\right) \ dx = 2\int_0^\ell \cos\left(\frac{n\pi x}{\ell}\right)\cos\left(\frac{m\pi x}{\ell}\right) \ dx = 0 \]
\[ \int_{-\ell}^\ell \sin\left(\frac{n\pi x}{\ell}\right)\sin\left(\frac{m\pi x}{\ell}\right) \ dx = 2\int_0^\ell \sin\left(\frac{n\pi x}{\ell}\right)\sin\left(\frac{m\pi x}{\ell}\right) \ dx = 0 \]

which follows from the Fourier cosine and sine series respectively. The last thing we need to do is consider when $n = m$ in both cases. 

For $\cos$, if $n \neq 0$, we get

\begin{align*}
    \int_{-\ell}^\ell \cos^2\left(\frac{n\pi x}{\ell}\right) \ dx & = 2\int_0^\ell \cos^2\left(\frac{n\pi x}{\ell}\right) \ dx \\ 
    & = 2\left(\frac{\ell}{2}\right) \tag{derived previously} \\
    & = \ell
\end{align*}

and if $n = 0$, we get

\[ \int_{-\ell}^\ell \cos(0) \ dx = \int_{-\ell}^\ell 1 \ dx = 2\ell \]

For $\sin$, we only consider when $n \neq 0$, giving us

\begin{align*}
    \int_{-\ell}^\ell \sin^2\left(\frac{n\pi x}{\ell}\right) \ dx & = 2\int_0^\ell \sin^2\left(\frac{n\pi x}{\ell}\right) \ dx \\
    & = 2\left(\frac{\ell}{2}\right) \\
    & = \ell
\end{align*}

Upon taking dot products and simplifying, we will get that

\[ A_n = \frac{1}{\ell}\int_{-\ell}^\ell \phi(x) \cos\left(\frac{n\pi x}{\ell}\right) \ dx \]
\[ B_n = \frac{1}{\ell}\int_{-\ell}^\ell \phi(x) \sin\left(\frac{n\pi x}{\ell}\right) \ dx \]

\subsection{Understanding Convergence}

Throughout our study of these infinite sums, we've mad a bit of an assumption in saying that that they do in fact converge and thus are well-defined. While full rigour is not necessarily what we are going for in MAT311 (we leave that to those working in real and functional analysis), we should probably get some understanding of when these series converge. To do this we will state 3 theorems that give different flavours of convergence.

\begin{theorem}
    Suppose $f$ is a $C^1$ function that satisfies either the Dirichlet, Neumann, or Periodic boundary conditions. Then we have that $f$ is equal to the respective Fourier series (sine, cosine, or full), and

    \[ \lim_{N \to \infty} \max_x\left|f(x) - \sum_{n=1}^N (\cdots)(x)\right| = 0 \]

    so $f(x) = \sum_{n=1}^\infty (\cdots)(x)$
\end{theorem}

This means that if $f$ satisfies the boundary conditions of our PDE, and is $C^1$, then it is of the form we've previously defined and equal to the infinite sum (meaning said sum converges).

\begin{theorem}
    Suppose $f$ is continuous and $f'$ is piecewise continuous, all on some interval $[0,\ell]$. Then for $x \in (0,\ell)$, 

    \[ f(x) = \sum_{n=1}^\infty (\cdots)(x) \]
\end{theorem}

This is a weaker result but is still quite useful, the only downside being that we cannot use it to deal with the endpoints $x = 0, \ell$. 

Here's one last theorem that gives another version of convergence of the sum. 

\begin{theorem}
    Suppose that 

    \[ \int f^2 \ dx < \infty \]

    Then

    \[ \lim_{N \to \infty}\int \left|f(x) - \sum_{n=1}^N (\cdots)(x)\right|^2 \ dx = 0 \]
\end{theorem}

With these in mind, we can now compute some actual Fourier series.

\begin{example}
    Compute the Fourier cosine series of $f(x) = 1$ on $(0, \ell)$.

    For $n \neq 0$, the coefficients are given by

    \begin{align*}
        A_n & = \frac{2}{\ell}\int_0^\ell 1 \cdot \cos\left(\frac{n\pi x}{\ell}\right) \ dx \\
        & = \frac{2}{\ell}\left[\frac{\ell}{n\pi}\sin\left(\frac{n\pi x}{\ell}\right)\right]_0^\ell \\
        & = 0
    \end{align*}

    while for $n = 0$, 

    \[ A_0 = \frac{2}{\ell}\int_o^\ell 1 \ dx = 2 \]

    Thus, we get that

    \[ 1 = \frac{1}{2}A_0 + 0 + \cdots = 1 \]

    This is a silly example, but it is still instructive.
\end{example}

\begin{example}
    Compute the Fourier sine series of $f(x) = 1$ on $(0, \ell)$. 

    The coefficients are given by

    \begin{align*}
        A_n & = \frac{2}{\ell}\int_0^\ell 1 \cdot \sin\left(\frac{n\pi x}{\ell}\right) \ dx \\
        & = \frac{2}{\ell}\left[\frac{-\ell}{n\pi}\cos\left(\frac{n\pi x}{\ell}\right)\right]_0^\ell \\
        & = \frac{-2}{n\pi}(\cos(n\pi) - 1) \\
        & = \frac{-2}{n\pi}[(-1)^n - 1] \\
        & = \begin{cases}
        0 & n \ \text{ is even} \\
        \frac{4}{n\pi} & n \ \text{ is odd}
        \end{cases}
    \end{align*}

    Hence we have that

    \[ 1 \overset{?}{=} \frac{4}{\pi}\left(\sin(x) + \frac{1}{3}\sin(3x) + \frac{1}{5}\sin(5x) + \cdots\right) \]

    Theorem 8.2 tells us that this equality is true when $x \in (0,\ell)$, but not when $x = 0, \ell$. Indeed,

    \[ 1 \neq \frac{4}{\pi}(0 + 0 + \cdots) = 0 \]

    so we get that 1 does not satisfy the Dirichlet boundary conditions in this case. 
\end{example}