\documentclass{ut-thesis}
\makeatletter
\makeatother
% packages
\usepackage[colorlinks]{hyperref} % for links
\usepackage[backend=biber]{biblatex} % for references
\usepackage{graphicx} % for embedding graphics
\usepackage{booktabs} % for pretty tables
\usepackage{lipsum} % for gibberish text

\usepackage{enumitem} % custom labels for lists
\usepackage{tikz}
\usepackage{tikz-cd}

% Custom math commands
\usepackage{amsmath, amssymb}

\newcommand{\N}{\mathbb{N}}
\newcommand{\Z}{\mathbb{Z}}
\newcommand{\Q}{\mathbb{Q}}
\newcommand{\R}{\mathbb{R}}
\newcommand{\C}{\mathbb{C}}
\newcommand{\cl}[1]{\overline{#1}} % closure
\newcommand{\widevec}[1]{\overrightarrow{#1}} % vector hats
\newcommand{\id}{\operatorname{id}} % identity map
\newcommand{\acts}{\curvearrowright}

% Theorem environments
\usepackage{amsthm}

\theoremstyle{plain} % The "plain" style italicizes all body text.
	\newtheorem{thm}{Theorem}
		\numberwithin{thm}{chapter} % Theorem numbers are determined by section.
	\newtheorem{lemma}[thm]{Lemma}
	\newtheorem{prop}[thm]{Proposition}
	\newtheorem{cor}[thm]{Corollary}
	\newtheorem{conj}[thm]{Conjecture}

\theoremstyle{definition} % The "definition" style does not italicize body text..
    \newtheorem{defn}[thm]{Definition}
	\newtheorem{example}[thm]{Example}

% author data
\author{Brandon Papandrea}
\title{Insert Title Here}
\department[]{Department of Mathematical and Computational Sciences}
\gradyear{2026}
% reference database
\addbibresource{main.bib}
\begin{document}
  \frontmatter
    \maketitle
    \begin{abstract}
      \lipsum[1-2]
    \end{abstract}
    \begin{dedication}
      To people.
    \end{dedication}
    \begin{acknowledgements}
      I'll write this later
    \end{acknowledgements}
    \tableofcontents
    \mainmatter
        \chapter{Introduction}

For any set $S$ and group $G$, there is a unique group up to isomorphism satisying the universal property, called the free group generated by $S$. If $|S| = n$, then we call this group the free group of rank $n$, and denote it as $F_n$.

Free groups were first studied as an example of Fuchsian groups, which are discrete subgroups of $PSL_2(\Z)$, but they did not become relevant on their own until the late 1800s. Recall that we may write any group using its group presentation. von Dyck, who was first to express groups using a presentation, showed that free groups have the simplest possible one: The free group $F_n$ has presentation 

\[\langle a_1, a_2, \ldots a_n \rangle\]

meaning it consists of $n$ generators with no relations between them; this is often used as an alternate definition of $F_n$. It is thus the case that every group is the quotient of a free group. The algebraic properties of free groups were investigated by Nielsen, who also gave them their name, whilst a comprehensive overview of free groups was given by Reidemeister in his 1932 book on combinatorial group theory.

The simple nature of free groups is equal parts useful and deceiving. Consdier The Word Problem. 

\begin{center}
    \textbf{The Word Problem}: For a group $G$, let $g$ be an element expressed as a product of $G$'s generators. Find an algorithm that shows, in finitely many steps, whether or not $g$ is the identity element.
\end{center}

While it is known that this problem is undecidable for arbitary groups, it is answerable if $G$ is a free group. Recall that a free group has no relations between the generators, apart from the obvious one that $a_ia_i^{-1} = 1$. Thus, given an arbitrary element of a free group, one can just check if any of these pairs exist, and eliminate them one at a time until none remain; if generators are still present, the element is not the identity, and if no generators are left, the element is the identity. Thus, while The Word Problem is undecidable, it is easily solvable in the case of free groups.  

This does not mean that free groups are a trivial class of groups. In fact, understanding their algebraic properties has been an area of study for over a hundred years. To demonstrate this, we consider two questions.

First, suppose we are given the presentation of an arbitrary group $G$. Can we easily check whether or not $G$ is a free group? This question was answered in the negative in the 1950s, where it was shown that such a question is undecidable. 

The second question is one that is often asked when a property of a structure is first defined: do the substructures of this structure also have this property? In particular, are the subgroups of a free group also free groups? This was answered in the positive in the 1920s, first in a restricted form by Nielsen, then the the full form independently by Dehn and Schreier However, such proofs were not trivial, and required high level topological or algebraic methods to get to a rigorous proof. Understanding the proofs would require understanding very high level results in group theory, far beyond the level of an undegraduate course in abstract algebra. 

Answers to the both of these questions do exist in simpler forms within the field of Combinatorial/Geometric Group Theory, which in part seeks to understand a group's algebraic properties by looking at how it acts on a space. This extra assumption on how a group acts upon a space makes these results much easier to understand at the undergraduate level. 

An answer to the first question can be found thanks to the Ping-Pong Lemma, which describes a sufficient condition for a group to be free, so long as it acts in a certain way on a set $X$:

\begin{thm}[The Ping-Pong Lemma]
    Let $\{g_1, g_2, \ldots, g_n\}$ generate a group $G$, which acts on a set $X$. If 

    \begin{enumerate}[label=(\roman*)]
        \item $X$ contains $n$ subsets $X_1, \ldots, X_n$ such that $X_i \cap X_j = \varnothing$, and
        \item $g_i^k(X_j) \subset X_i$ for all nonzero powers $k$ and $i \neq j$.
    \end{enumerate}

    Then $G$ is isomorphic to $F_n$. 
\end{thm}

More interestingly, an answer to question two is a corollary of a result proven by Serre in 1969, who showed that free groups may be classified by how they act on trees.

\begin{thm}[Serre's Theorem]
    $G$ is a free group if and only if it acts freely on a tree. 
\end{thm}

\begin{cor}[Nielsen-Schreier Theorem]
    Every subgroup of a free group is itself free. 
\end{cor}

In this paper, we present full proofs and explanations of both results. First, we prove the Ping-Pong Lemma both in the case of $n = 2$, and in the general case for arbitrary $n$. We also apply the Ping-Pong Lemma, and another result, to show that free groups of any rank, including one of infinite rank, lies within the matrix group $SL_2(\Z)$. Afterwards, we prove both sides of Serre's Theorem. In proving the forwards direction, we define a special graph for our free group to act on called the Cayley Graph, and then show that free groups acts freely on their corresponding Cayley Graphs. To prove the reverse direction, we show that every group acting on a tree induces a special tiling of the tree, before finding a free generating set using this tiling. 
        \chapter{Free Groups}

\section{The Universal Property}\label{The Universal Property}

For us to work with free groups, we first need to understand what it means for a group to be ``free.'' 

\begin{defn}\label{Free Group Universal Property}
    Let $S$ be a set. Then a group $F_S$ is called the \textbf{free group generated by $S$} if the following holds: for any function $f$ from $S$ to a group $G$, there is a unique homomorphism $\varphi: F_S \to G$ such that the below diagram commutes (where $\iota$ denotes inclusion).

    \[\begin{tikzcd}
        & {F_S} \\
        S & G
        \arrow["\varphi", dashed, from=1-2, to=2-2]
        \arrow["\iota",from=2-1, to=1-2]
        \arrow["f"', from=2-1, to=2-2]
    \end{tikzcd}\]

    We call $S$ the \textbf{generating set} of $F_S$. 
\end{defn}

The condition that $F_S$ must satisfy is called the \textbf{universal property}, and plays an important role in developing free structures in category theory. A priori, it is not obvious that such a group exists for every such set. However, not only is there such a group, but this group is unique up to isomorphism.

\begin{thm}\label{Free Groups Exist}
    For any set $S$, there is a free group $F_S$ generated by $S$.

    $F_S$ is also unique up to isomorphism; if $F^1_S, F^2_S$ are both free groups generated by $S$, with inclusion maps given by $\iota_1: S \to F^1_S, \iota_2: S \to F^2_S$, then there are unique isomorphisms $\hat{\iota_2}: F_S^1 \to F_S^2$ and $\hat{\iota_1}: F_S^2 \to F_S^1$ such that

    \[ \hat{\iota_2} \circ \iota_1 = \iota_2, \quad \hat{\iota_1} \circ \iota_2 = \iota_1, \quad \hat{\iota_1}^{-1} = \hat{\iota_2}\]
\end{thm}

\begin{proof}
    We let $F_S$ be the set of reduced words with letters given by elements of $S$, with the operation given by concatenation and reduction. Now, for any $f: S \to G$, where $G$ is a group, we define $\varphi$ as follows: we send the empty word $\varnothing$ to $1_G$, the identity element of $G$, and for any $s \in S$, 

    \[\varphi(s) = f(s)\]

    where we think of $s$ as a word consisting of a single letter in $F_S$. $\varphi$ can then be uniquely extended to all other words by the fact that it is a homomorphism. 

    We now show uniqueness. Setting $G = F_S^1$ and $f = \iota_1$ in the diagram from Definition \ref{Free Group Universal Property} gives us 

    \[\begin{tikzcd}
        & {F_S^1} \\
        S & {F_S^1}
        \arrow["\id_{F_S^1}", dashed, from=1-2, to=2-2]
        \arrow["\iota_1",from=2-1, to=1-2]
        \arrow["\iota_1"', from=2-1, to=2-2]
    \end{tikzcd}\]

    Similarly, setting $G = F_S^2$ and $f = \iota_2$, we get 

    \[\begin{tikzcd}
        & {F_S^1} \\
        S & {F_S^2}
        \arrow["\hat{\iota_2}", dashed, from=1-2, to=2-2]
        \arrow["\iota_1",from=2-1, to=1-2]
        \arrow["\iota_2"', from=2-1, to=2-2]
    \end{tikzcd}\]

    from which we get that $\hat{\iota_2} \circ \iota_1 = \iota_2$. Swapping $F_S^1$ and $F_S^2$  and their respective maps gives us $\hat{\iota_1}$ such that $\hat{\iota_1} \circ \iota_2 = \iota_1$.

    Notice that $\hat{\iota_1} \circ \hat{\iota_2}: F_S^1 \to F_S^1$ satisfies 

    \[ \hat{\iota_1} \circ \hat{\iota_2} \circ \iota_1 = \hat{\iota_1} \circ \iota_2 = \iota_1 \]

    satisfying the same equation that $\id_{F_S^1}$ does. By uniqueness, we get that $\hat{\iota_1} \circ \hat{\iota_2} = \id_{F_S^1}$. A similar argument shows that $\hat{\iota_2} \circ \hat{\iota_1} = \id_{F_S^2}$. This completes the proof. 
\end{proof}

The presence of a set $S$ seems to complicate our definition, as we must define free groups in terms of this generating set. However, we can greatly reduce the number of sets we must consider by making a key observation. Suppose $S,T$ are sets such that $|S| = |T|$, and let $F_S, F_T$ be the corresponding free groups. Recall that we construct each free group by sending each element of the generating set to a letter in the free group, and define the homomorphisms $\varphi$ by how they map these letters. As $S,T$ are the same size, there is a bijective map $h: S \to T$. More importantly, this map lifts to one on the free groups 

\[ \psi: F_S \to F_T, \quad \psi(\iota_S(s)) = \iota_T(h(s)) \]

As $h$ is a bijection on the generating sets, and each free group is uniquely determined by the letters these sets map to, $\psi$ is an isomorphism of the groups. We have just proven 

\begin{thm}\label{Free Groups From Generating Sets of Same Cardinality are Isomorphic}
    Every free group generated by sets of the same cardinality are isomorphic. 
\end{thm}

Because of this, we only need to consider free groups generated by sets of the form $\{1,2,\ldots, n\}$; the size of this set is called the free group's \textbf{rank}. We will thus denote these groups by $F_n$, called the \textbf{free group of rank $n$}. For the purposes of this paper, we will restrict our study to free groups of countable rank, meaning we consider $F_n$ for all $n \in \N$, as well as $F_\infty$, the free group with generators $x_i$ for all natural $i$.

The construction of free groups makes them naturally important in the study of arbitrary groups. The first to notice this fact was Walther von Dyck, one of the founders of Combinatorial Group Theory, and the first to study groups via generators and relations. It was through this study that he found the usefulness of free groups:

\begin{thm}\label{Groups are Quotients of Free Groups}
    Every group $G$ is the quotient of a free group. 
\end{thm}

\begin{proof}
    Let $G$ have a presentation given by $\langle g_1, g_2, \ldots, g_n | R \rangle$. Let $S = \{1, 2, \ldots, n\}$ and let $F_n$ be the free group generated by $S$. Consider the map 

    \[f: S \to G, \quad f(i) = g_i\]

    Then by the universal property, let $\varphi: F_n \to G$ be the unique homomorphism associated with $f$. It follows that

    \[\ker\varphi = \{g = g_1^{e_1}\cdots g_n^{e_n} : g = 1\} = R\]

    In other words, $\ker G$ is the set of all relations on $G$. Thus, we get that 
    
    \[G = F_n/\ker\varphi\]
\end{proof}

Because of von Dyck's Theorem, we are able to define free groups in a very simple manner. By construction, the relations within a free group $F_n$ are strictly those saying that $x_ix_i^{-1} = 1$ for any $i \in \{1,2,\ldots, n\}$. Thus, free groups have the simplest possible presentation, and can be defined as just a set of generators with no relations:

\begin{defn}\label{Free Groups as Reduced Words}
    The \textbf{free group of rank $n$} is the group $F_n$, the set of all reduced words in symbols $\{x_1, \ldots, x_n\}$, with operation given by concatenation and reduction. Its presentation is 

    \[\langle a_1, a_2, \ldots, a_n \rangle\]

    and the identity is the empty word $\varnothing$. 
\end{defn}

Becuase of this result, we are able to translate the algebraic definition of a group, that which uses the universal property, to one using reduced words, which are much easier to work with. 

\section{The Rank of Subgroups of $F_2$}

Rank is a group theoretic analogue of dimension for vector spaces. 

\begin{defn}
    The \textbf{rank} of a group $G$ is the size of the smallest possible generating set for $G$.

    \[\operatorname{rank}(G) = \min\{|X| : X \subseteq G, \langle X \rangle = G\}\]
\end{defn}

For a vector space $V$, any subspace $W$ will have at most dimension $\dim V$, which makes intuitive sense. However, despite being an analogue of dimension for groups, this property does not hold for arbitrary groups; it hold for special types of groups, such as finite abelian groups. 

The subgroups of free groups can oftentimes have larger rank than the free group itself. In particular, $F_2$, not only contains subgroups of larger rank, but they contain subgroups of every possilbe rank larger than 2, including subgroups of countably infinite rank!

\begin{thm}\label{Free Groups Contain Subgroups of Arbitrarily Large Rank}
    \begin{enumerate}[label=(\alph*)]
        \item For all $n > 2$, $F_n \subset F_2$. 
        \item $F_\infty \subset F_2$, where
        \[F_\infty = \langle a_1, a_2, a_3, \ldots \rangle\]
        is the free group of countably infinite rank. 
    \end{enumerate}
\end{thm}

\begin{proof}
    \begin{enumerate}[label=(\alph*)]
        \item We write $F_2 = \langle a,b \rangle$, and define 
        
        \[w_1 = a\]
        \[w_2 = b^{-1}ab\]
        \[w_3 = b^{-2}ab^{-2}\]
        \[\vdots\]
        \[w_n = b^{-n}ab^n\]

        With the operations of concatenation and reduction, these form the generating set of a group. It thus suffices to show that there are no relations between these generators. First notice that for any $w_i$ and any integer $k$:

        \begin{align*}
            w_i^k & = (b^{-i}ab^i)(b^{-i}ab^i)\cdots(b^{-i}ab^i) \\
            & = b^{-i}a(b^ib^{-i})a(b^ib^{-i})a\cdots a(b^ib^{-i})ab^i \\
            & = b^{-i}a^kb^i
        \end{align*}

        Now, let $w = w_{i_i}^{k_1}\cdots w_{i_l}^{e_l}$ be an arbitrary word. We have that 

        \begin{align*}
            w & = w_{i_i}^{k_1}\cdots w_{i_l}^{e_l} \\
            & = (b^{-i_1}a^{e_1}b^{i_1})(b^{-i_2}a^{e_2}b^{i_2})\cdots(b^{-i_l}a^{e_l}b^{i_l}) \\
            & = b^{-i_1}a^{e_1}b^{i_i - i_2}a^{e_2}b^{i_2 - i_3} \cdots b^{i_{l-1} - i_l}a^{e_l}b^{i_l}
        \end{align*}

        and this is fully reduced. Moreover, it clearly cannot be expressed as the reduced product of any other set other product of the generators. Therefore, we conclude that there are no relations between the $w_i$'s, and hence they generate the free group $F_n$, so $F_n \subset F_2$.

        \item The proof is identical to (a), with the generating set of $F_\infty$ being the set of $w_i$ for all $i \in \N$.
    \end{enumerate}
\end{proof}

One might ask if this result generalizes to free groups of rank higher than 2? In particular, for any $k \in \N$, is it true that $F_n \subset F_k$ for $n > k$? If so, we'd get a chain of inclusions

\[F_2 \supset F_3 \supset F_4 \supset \cdots\]

The answer is no, and follows from a result of Schreier: 

\begin{thm}[Schreier Index Formula]
    If $H$ is a subgroup of $F_n$ with index $e$, then $H$ has rank 
    \[1 + e(n-1)\]
\end{thm}
        \chapter{The Ping-Pong Lemma}

As stated previously, given a group presentation $G = \langle S | R \rangle$, there is no algorithm running in finite time that can determine if this group is free. However, if we provide additional information about how $G$ acts on a set $X$, we can determine if this group is free. This result is known as the Ping-Pong Lemma, and is one of the foundational results in Geometric Group Theory. 

The result is attributed to Felix Klein, who used it in the late 1800s to study certain isometries of hyperbolic space. More recently, it was used by Jacques Tits in 1972 to prove the result now knows as the Tits Alternative. It is sometimes called Schottky's Lemma or Klein's Criterion.

\section{Statements and Proofs}

There are many different statements of the Ping-Pong Lemma for free groups, free products, etc. We focus on a relatively simple version that provides better intuition for why we call it the Ping-Pong Lemma. We start with a special case:

\begin{thm}[The Ping-Pong Lemma for Two Players]\label{Ping Pong for 2 Players}
    Suppose $a,b$ generate a group $G$, and $G \acts X$. If 

    \begin{enumerate}
        \item $X$ has disjoint nonempty subsets $X_a, X_b$, and 
        \item for all powers $k \neq 0$, $a^k(X_b) \subset X_a$ and $b^k(X_a) \subset X_b$
    \end{enumerate}

    Then $G$ is isomorphic to a free group of rank 2.
\end{thm}

\begin{proof}
    As $G$ has only 2 generators, it suffices to show there are no relations in $G$. Let

    \[g = a^{e_1}b^{e_2}\cdots b^{e_{k-1}}a^{e_k}\]

    be a word in $G$ and let $p \in X_b$. Consider $gp$. Notice that $a^{e_k}$ will send $p$ to $X_a$, then $b^{e_{k-1}}$ will send $a^{e_k}p$ back to $X_b$. We go back and forth from $X_a$, to $X_b$, etc.
    
    At the end, $a^{e_1}$ takes $b^{e_2}\cdots b^{e_{k-1}}a^{e_k}p$ to $X_a$. As $X_a \cap X_b = \varnothing$, and $gp \in X_a$ while $p \in X_b$, $gp \neq p$, meaning $g \neq 1$. 

    We conclude by showing that any word in $h \in G$, there is a word $y \in G$ such that $yhy^{-1}$ looks like $g$; in other words, every word in $G$ is conjugate to some word that looks like $g$. Let $h$ be a nontrivial reduced word in $\{a,b,a^{-1},b^{-1}\}$. For the purpose of brevity, we will let $*$ denote a nonzero exponent. We have several cases:

    \begin{enumerate}
        \item If $h = a^*b^*\cdots b^*a^*$, take $y = 1$. 
        \item If $h = b^*a^*\cdots b^*a^i$ or $h = a^ib^*\cdots b^*a^*$, take $y = a^j$ for some $j \neq i$.
        \item If $h = b^*a^*\cdots a^*b^*$, take $y = a$.
    \end{enumerate}

    What this tells is that any word in $G$ is conjugate to a nontrival element of $G$. As all conjugates of 1 are trivial, this means that every word in $G$ must be nontrivial. Thus, there are no relations between elements of $G$, so $G$ is isomorphic to a free group of rank 2.  
\end{proof}

The section of the proof where we bounce between $X_a$ and $X_b$ should give a good idea as to why this is called the Ping-Pong Lemma. $p$ starts in $X_b$, then is sent to $X_a$, then back to $X_b$, then back to $X_a$, etc. It's like watching a ping-pong game, where the ball is $p$, $X_a$ and $X_b$ are the sides of the table, and the ball is ``hit around" by $g$. 

\begin{center}
    \includegraphics[width=\textwidth]{Images/Ping-Pong.png}
\end{center}

We can extend this proof for rank 2 to find free groups of any finite rank:

\begin{thm}[The Ping-Pong Lemma for $n$ Players]\label{Ping Pong for n Players}
    Suppose $\{g_1, \ldots, g_n\}$ generate a group $G$, and $G \acts X$. If 

    \begin{enumerate}
        \item $X$ has pairwise disjoint nonempty subsets $X_1, \ldots, X_n$, and 
        \item for all powers $k \neq 0$, $g_i^k(X_j) \subset X_i$ for all $i \neq j$.
    \end{enumerate}

    Then $G$ is isomorphic to a free group of rank $n$. 
\end{thm}

\begin{proof}
    Let $g = g_{i_1}^{e_1}g_{i_2}^{e_2}\cdots g_{i_k}^{e_k}$ be a nontrivial reduced word in $G$, and let $p \in X_j$ where $j \neq i_k$. It follows that $gp \in X_i$, $i \neq j$, and thus $g \neq 1$. This suffices as all elements of $G$ are conjugate to an element of the above form.  
\end{proof}

After seeing this, one may ask if this forms both a necessary and sufficient condition for a group to be free. That is, given a group $G$ acting on a set $X$, is it free if and only if the properties of $X$ and the group action described in the Ping-Pong Lemma apply? The answer is yes, because for any free group, there is always a set and group action on that set fulfilling these conditions.

\begin{thm}\label{Free Groups Satisfy Ping Pong}
    There is a set $X$ with subsets $X_i$ and a group action of $F_n$ on $X$ fulfilling the conditions of the Ping-Pong Lemma. 
\end{thm}

\begin{proof}
    We let

    \[X = \{1,2,\ldots,n\}\]

    and define $X_i = \{i\}$ for all $i$. This gives us $n$ pairwise disjoint nonempty subsets, fulfilling the first condition of the Ping-Pong Lemma.

    We define our group action of $F_n$ on $X$ with respect to generators. Suppose $F_n$ is generated by $a_1, a_2, \ldots, a_n$. Then for any $x \in X$, we define 

    \[a_ix := i\]

    Clearly, if $k \neq 0$ and $i \neq j$, then 

    \[a_i^kj = i\]

    so $a_i^k(X_j) \subset X_i$, fulfilling the second condition of the Ping-Pong Lemma. 
\end{proof}

\section{An Application: $SL_2(\Z)$ and $GL_2(\Z)$}

The Ping-Pong Lemma is a powerful tool for finding free subgroups of certain groups. In finding them, we can find a large class of free subgroups thanks to the Schreier Index Formula.

Recall that $SL_2(\Z)$ is the group of $2 \times 2$ matrices of determinant 1. Let $G = \langle A, B \rangle$ be a subgroup of $SL_2(\Z)$ generated by 

\[A = \begin{pmatrix}
    1 & 2 \\ 0 & 1
\end{pmatrix}, \quad B = \begin{pmatrix}
    1 & 0 \\ 2 & 1
\end{pmatrix}\]

\begin{lemma}\label{SL2Z Matrix Powers}
    For any power $k \neq 0$,

    \[A^k = \begin{pmatrix}
        1 & 2k \\ 0 & 1
    \end{pmatrix}, \quad B^k = \begin{pmatrix}
        1 & 0 \\ 2k & 1
    \end{pmatrix}\]
\end{lemma}

\begin{proof}
    First assume $k > 0$. We proceed by induction. The base case $k = 1$ is obvious. Now assume that the claim holds on $k$ and consider $k+1$. Then we have that 
    
    \begin{align*}
        A^{k+1} & = (A^k)A \\
                & = \begin{pmatrix}
                    1 & 2k \\ 0 & 1
                \end{pmatrix}\begin{pmatrix}
                    1 & 2 \\ 0 & 1
                \end{pmatrix} \\
                & = \begin{pmatrix}
                    1 & 2 + 2k \\ 0 & 1
                \end{pmatrix} \\
                & = \begin{pmatrix}
                    1 & 2(k+1) \\ 0 & 1
                \end{pmatrix}
    \end{align*}

    Similarly, 

    \begin{align*}
        B^{k+1} & = (B^k)B \\
                & = \begin{pmatrix}
                    1 & 0 \\ 2k & 1
                \end{pmatrix}\begin{pmatrix}
                    1 & 0 \\ 2 & 1
                \end{pmatrix} \\
                & = \begin{pmatrix}
                    1 & 0 \\ 2k + 2 & 1
                \end{pmatrix} \\
                & = \begin{pmatrix}
                    1 & 0 \\ 2(k+1) & 1
                \end{pmatrix}
    \end{align*}

    as desired. For negative exponents, we first note that

    \[\begin{pmatrix}
        1 & 2k \\ 0 & 1 
    \end{pmatrix}\begin{pmatrix}
        1 & -2k \\ 0 & 1
    \end{pmatrix} = \begin{pmatrix}
        1 & 0 \\ 0 & 1
    \end{pmatrix}\]

    \[\begin{pmatrix}
        1 & 0 \\ 2k & 1
    \end{pmatrix}\begin{pmatrix}
        1 & 0 \\ -2k & 1
    \end{pmatrix} = \begin{pmatrix}
        1 & 0 \\ 0 & 1
    \end{pmatrix}\]

    and this completes the proof since $(A^k)^{-1} = A^{-k}$. 
\end{proof}

We claim that $G$ is isomorphic to a free group of rank 2. To do this, we define the action of $G$ on $\Z^2$ by standard matrix multiplication:

\[\begin{pmatrix}
    a & b \\ c & d
\end{pmatrix}\begin{pmatrix}
    x \\ y
\end{pmatrix} = \begin{pmatrix}
    ax + by \\ cx + dy
\end{pmatrix}\]

Moreover, we define our sets as 

\[X_A = \left\{\begin{pmatrix}
x \\ y
\end{pmatrix} \in \R^2 : |x| > |y|\right\}, \quad X_B = \left\{\begin{pmatrix}
x \\ y 
\end{pmatrix} \in \R^2 : |x| < |y|\right\}\]

We claim that $X_A, X_B$ meet the criteria of the Ping-Pong Lemma. First note that these sets are clearly disjoint. Next, let $\vec{x} \in X_a$. Using Lemma \ref{SL2Z Matrix Powers}, we have that for any power $k \neq 0$,

\[A^k\vec{x} = \begin{pmatrix}
    x + 2ky \\ y
\end{pmatrix}\]

and we see that 

\begin{align*}
    |x + 2ky| & \leq |x| + |2k||y| \\
              & < |2k||y| \\
              & < |y|
\end{align*}

so $A^k\vec{x} \in X_B$, meaning $A^k(X_A) \subset X_B$. Similarly, if $\vec{x} \in X_B$, then for any power $k \neq 0$,

\[B^k\vec{x} = \begin{pmatrix}
    x \\ 2kx + y
\end{pmatrix}\]

and we get that 

\begin{align*}
    |2kx + y| & \leq |2k||x| + |y| \\
              & < |2k||x| \\
              & < |x|
\end{align*}

so $B^k\vec{x} \in X_A$, meaning $B^k(X_B) \subset X_A$. The Ping-Pong Lemma then applies, so $G$ is isomorphic to a free group of rank 2. In addition to showing that $F_2$ is a subgroup of $SL_2(\Z)$, by Theorem \ref{Free Groups Contain Subgroups of Arbitrarily Large Rank}, this also tells us that $F_n \subset SL_2(\Z)$ for any $n > 2$. 

We can generalize this result slightly to find free subgroups of $GL_2(\Z)$, the group of invertible $2 \times 2$ integer matrices. Doing this amounts to setting 

\[A = \begin{pmatrix}
    1 & m \\ 0 & 1
\end{pmatrix}, \quad B = \begin{pmatrix}
    1 & 0 \\ m & 1
\end{pmatrix}\]

where $m \geq 2$. Using the exact same methods as before, we may show that $G = \langle A,B \rangle \subset GL_2(\Z)$ is isomorphic to a free group of rank 2. 
        \chapter{A Group Acting Freely on a Tree is Free}

We first prove the forwards direction.

\begin{thm}\label{Serre Forward Direction}
    If $T$ is a tree and $G$ is a group such that $G \acts T$ freely, then $G$ is isomorphic to a free group. 
\end{thm}

To this end, we let $T$ be a tree and we let $G$ be a group such that $G \acts T$ freely. 

\section{G-Tilings}

\begin{defn}
    A \textbf{tile} of a tree $T$ is any subtree of $T$. 

    A \textbf{tiling} of $T$ is a collection of tiles of $T$ such that 

    \begin{enumerate}
        \item The union of all tiles is $T$, and
        \item Two tiles intersect, at most, at a vertex.
    \end{enumerate}

    If $G \acts T$, and if 

    \begin{enumerate}
        \item[3.] There is a tile $T_0$ such that the set of all tiles is $\{gT_0: g \in G\}$
    \end{enumerate}

    then we call the tiling a \textbf{$G$-tiling}.
\end{defn}

It is not obvious a priori that $T$ will always have a $G$-tiling. However, one can show that it always does. 

\begin{thm}\label{Groups Acting on Trees have G-Tilings}
    Let $G$ be a group and $T$ a tree. If $G \acts T$, then $T$ has a $G$-tiling. 
\end{thm}

Before proving this we need two more objects.

\begin{defn}
    For a tree $T$, its \textbf{barycentric subdivision} is the tree $T'$ obtained by putting a new vertex into the middle of every edge in $T$. 
\end{defn}

Without loss of generality, we now assume that every non-leaf vertex in $T$ has at least degree 2; if not, then use the barycentric subdivision $T'$, which has this property. The reason for this will become relevant later in Lemma \ref{T_g's cover T}.

Now, how exactly will we create our $G$-tiling? We can do this using our group action. Let $v_0 \in V(T)$ be a vertex in the tree, and consider its orbit under the group action.

\[ \mathcal{O}(v_0) = \{gv_0 : g \in G\} \]

For every vertex $v \in V(T)$, there is a $gv_0 \in \mathcal{O}(v_0)$ that is closest to $v$ under the path metric. From this, we define a subgraph $T_g$ as follows:

\[ V(T_g) = \{v \in T : d(v,gv_0) \leq d(v,g'v_0) \ \forall g' \neq g\} \]
\[ E(T_g) = \{(u,v) \in E(T) : u,v \in V(T_g)\} \]

In other words, $T_g$ is a subgraph whose vertex set is all those vertices that are closest to $gv$ in $T$ under the path metric, and whose edge set is all those edges whose endpoints are both vertices in $T_g$. 

\begin{lemma}\label{Tg is a Tile}
    Let $G \acts T$. Then $T_g$ is a tile for all $g \in G$.
\end{lemma}

\begin{proof}
    It suffices to prove that $T_g$ is connected, as every connected subgraph of a tree is a subtree. Let $w \in V(T_g)$. There is a unique edge path $w$ to $gv$; suppose it is of length $n$.

    \[
    w_0 = w, w_1, w_2, \ldots, w_{k-1}, w_n = gv
    \]

    Then $d(w_1,gv_0) = n-1$. We claim that $w_i \in V(T_g)$ for all $i$. Starting with $w_1$, suppose by way of contradiction that $w_1 \notin V(T_g)$. Then there is a $g'$ such that 

    \[
    d(w_1,g'v_0) = m < n-1
    \]

    But then going back on edge to $w$, we get that 

    \[
    d(w,g'v_0) \leq m+1 < n
    \]

    which contradicts the fact that $w \in V(T_g)$. Thus, $w_1 \in V(T_g)$. This logic can be repeated for all other vertices in the path, ensuring that all are in $V(T_g)$, as desired. This means that $T_g$ is connected, hence a subtree of $T$, and thus, is a tile. 
\end{proof}

Now that we know the $T_g$'s are in fact tiles, we should also check that they cover all of $T$. 

\begin{lemma}\label{T_g's cover T}
    The union of all tiles $T_g$ for all $g$ is equal to $T$. 
\end{lemma}

\begin{proof}
    It is evident that $\bigcup_{g\in G}V(T_g) = V(T)$, since every vertex must be closest to some $gv_0$. It is also evident that

    \[ \bigcup_{g \in G}E(T_g) \subseteq E(T) \]

    So it suffices to show the reverse inclusion: every edge in $T$ belongs to some $T_g$. Let $(u,w) \in E(T)$. Then by construction of $T$, the distance to $\mathcal{O}(v_0)$ from one of $u,w$ is odd, while the distance from the other is even. Without loss of generality, suppose that

    \[
    d(u,\mathcal{O}(v_0)) \ \text{ is even} \quad \text{ and } \quad d(w,\mathcal{O}(v_0)) \ \text{ is odd}
    \]

    that $d(u,\mathcal{O}(v_0)) < d(w,\mathcal{O}(v_0))$, and that $u \in V(T_g)$. By the Triangle Inequality, we have that 

    \[
    d(w,gv_0) \leq d(w,u) +  d(u,gv_0) = d(u,gv_0) + 1
    \]

    As the distance from $w$ to $\mathcal{O}(v_0)$ is an integer, and cannot be smaller than or equal to $d(u,gv_0)$, the only possibility is that it is equal to $d(w,gv_0)$. Thus, 

    \[ d(w,gv_0) \leq d(w,g'v_0) \ \forall g' \neq g \]

    so $w \in V(T_g)$. We conclude that $(u,w) \in E(T_g)$, as desired. 
\end{proof}

The second condition, that two tiles share at most a vertex, is somewhat obvious. Indeed, if $(u,w)$ is in both $E(T_g)$ and $E(T_{g'})$, then $u,w$ are of equal distance to $gv_0$ and $g'v_0$. But assuming $u$ is of equal distance to $gv_0$ and $g'v_0$, $w$ must be closer to one of $gv_0$ or $g'v_0$, since it is along the unique edge path from $u$ to one of them. 

The final condition can be proven using another lemma: 

\begin{lemma}\label{gT_h = T_gh}
    For $g,h \in G$, $gT_h = T_{gh}$.
\end{lemma}

\begin{proof}
    It suffices to show that $g$ maps vertices in $T_h$ to those in $T_{gh}$, as they're completely determined by vertices. Let $u \in V(T_h)$. Then for all $k \in G$,

    \[
    d(u,hv_0) \leq d(u,kv_0)
    \]

    Left multiplication by $g^{-1}$ is a bijection of $G$, so we may write 

    \[
    d(u,hv_0) \leq d(u,(g^{-1}k)v_0)
    \]

    For all $k \in G$. Now, as $G \acts T$, and $T$ is a metric space using the path metric, the action of $G$ preserves distances. Thus,

    \[
    d(gu, (gh)v) \leq d(gu,kv)
    \]

    so $gu \in T_{gh}$, as desired. 
\end{proof}

\begin{proof}[Proof of Theorem \ref{Groups Acting on Trees have G-Tilings}]
    By Lemmas \ref{Tg is a Tile} and \ref{T_g's cover T}, the collection $\{T_g: g \in G\}$ is a tiling of $T$. Setting $h = 1$ and applying Lemma \ref{gT_h = T_gh}, all tiles are of the form $gT_1$ for some $g \in G$.
\end{proof}

We have thus shown that any group acting on a tree induces a $G$-tiling on the tree.

\section{Proof of Theorem \ref{Serre Forward Direction}}

Now that we know $G$ induces a $G$-tiling on our tree $T$, we can use it to create a free generating set. 

We let $\{T_g\}$ be the $G$-tiling found in Theorem \ref{Groups Acting on Trees have G-Tilings}. As $G$ acts freely on $T$, $\operatorname{Stab}_G(v_0) = \{1\}$, so by the Orbit-Stabilizer Theorem,

\[|\mathcal{O}(v_0)| = 1 \cdot |G| = |G|\]

Thus the elements of $G$ are in bijective correspondence with the elements of the orbit of $v_0$. In other words, every $g \in G$ corresponds to a tile $gT_1$. define 

\[
S = \{g \in G : (gT_1) \cap T_1 \neq \varnothing\}
\]

to be the set of tiles that intersect with $T_1$.

\begin{lemma}\label{S Generates G}
    $S$ generates $G$
\end{lemma}

\begin{proof}
    First we show that if $s \in S$, so too is $s^{-1}$. Indeed let $s \in S$. Then there is a vertex $w$ such that 

    \begin{align*}
        & (sT_1) \cap T_0 = \{w\} \\
        \implies & s^{-1}(sT_1) \cap (s^{-1}T_1) = \{s^{-1}w\} \\
        \implies & T_1 \cap (s^{-1}T_1) = \{s^{-1}w\}
    \end{align*}

    so $s^{-1} \in S$. 

    Now let $g \in G$ and consider $gv_0$. There is a unique path in $T$ from $gv_0$ to $v_0$, passing through tiles 

    \[T_{g_n}, T_{g_{n-1}}, \ldots, T_{g_1}, T_{g_0}\]

    We claim that $g_{i-1}^{-1}g_i\in S$. The above path travels from $T_{g_{i+1}}$ to $T_{g_i}$ without passing through a middle tile, so 

    \[T_{g_{i+1}} \cap T_{g_i} \neq \varnothing\]

    Applying $g_i^{-1}$, we get 

    \[g_i^{-1}T_{g_{i+1}} \cap T_1 \neq \varnothing \implies T_{g_i^{-1}g_{i+1}} \cap T_1 \neq \varnothing\]

    so $g_i^{-1}g_{i+1} \in S$, and we write it as $s_i$. Then, we have that 

    \begin{align*}
        g & = g_n \\
          & = g_{n-1}g_{n-1}^{-1}g_n \\
          & \vdots \\
          & = g_1^{-1}g_1g_2^{-1}g_2\cdots g_{n-1}^{-1}g_n \\
          & = s_1s_2 \cdots s_n
    \end{align*}

    so $g$ may be written as a product of elements of $S$, meaning $S$ generates $G$.
\end{proof}

With all these tools now in place, the last thing we need to do is show that there are no relations between elements of $S$. 

\begin{proof}[Proof of Theorem \ref{Serre Forward Direction}]
    By Lemma \ref{Groups Acting on Trees have G-Tilings} we have a $G$-tiling $\{T_g\}$, which we know generates $G$ by Lemma \ref{S Generates G}. We claim that every $g \in G$ may be written as a unique, reduced product of elements of $S$. Let 

    \[g = s_1s_2 \cdots s_k\]

    with $s_i \neq s_{i+1}^{-1}$. We claim there is a unique path from $gv_0$ to $v_0$ that passes through, in order, the tiles
    
    \[T_g = T_{s_1s_2\cdots s_k}, T_{s_1s_2\cdots s_{k-1}}, \ldots, T_{s_1}, T_1\]

    As $T_{s_1} \cap T_1 = \{w_1\}$ for some $w_1 \in V(T)$, $T_{s_1} \cup T_1$ is a tree, where $v_0 \in V(T_1)$ and $s_1v_0 \in T_{s_1}$. Thus, there is a unique path from $s_1v_0$ to $v_0$ in $T_{s_1} \cup T_1$. Similarly, we have 

    \[T_{s_1s_2} \cap T_{s_1} = s_1^{-1}(T_{s_2} \cap T_1) = \{s_1^{-1}w_2\}\]

    for some $w_2 \in V(T)$, so $T_{s_1s_2} \cup T_{s_1}$ is a tree. Again, this means there is a unique path from $s_1v_0$ to $s_1s_2v_0$ in $T_{s_1s_2} \cup T_1$. If we continue this process inductively, we get the desired unique path. This path completely deterimines the word $s_1s_2\cdots s_k$ representing $g$, and thus $g$ can only be written as this freely reduced word in $S$. This tells us that $S$ has no relations, and thus $G$ is free, as desired. 
\end{proof}

\begin{cor}[Nielsen-Schreier Theorem]
    Ever subgroup of a free group is also a free group. 
\end{cor}

\begin{proof}
    A free action of a group $G$ on a tree trivially restricts to an action of $H \leq G$ on the tree. As $1 \in H$, if the action of $G$ is free, so too must the action of $H$. 
\end{proof}
    \appendix
    \backmatter
    \printbibliography[heading=bibintoc]
    \lipsum[1-6]
\end{document}