\chapter{Free Groups}

\section{The Universal Property}\label{The Universal Property}

For us to work with free groups, we first need to understand what it means for a group to be ``free.'' 

\begin{defn}\label{Free Group Universal Property}
    Let $S$ be a set. Then a group $F_S$ is called the \textbf{free group generated by $S$} if the following holds: for any function $f$ from $S$ to a group $G$, there is a unique homomorphism $\varphi: F_S \to G$ such that the below diagram commutes (where $\iota$ denotes inclusion).

    \[\begin{tikzcd}
        & {F_S} \\
        S & G
        \arrow["\varphi", dashed, from=1-2, to=2-2]
        \arrow["\iota",from=2-1, to=1-2]
        \arrow["f"', from=2-1, to=2-2]
    \end{tikzcd}\]

    We call $S$ the \textbf{generating set} of $F_S$. 
\end{defn}

The condition that $F_S$ must satisfy is called the \textbf{universal property}, and plays an important role in developing free structures in category theory. A priori, it is not obvious that such a group exists for every such set. However, not only is there such a group, but this group is unique up to isomorphism.

\begin{thm}\label{Free Groups Exist}
    For any set $S$, there is a free group $F_S$ generated by $S$.

    $F_S$ is also unique up to isomorphism; if $F^1_S, F^2_S$ are both free groups generated by $S$, with inclusion maps given by $\iota_1: S \to F^1_S, \iota_2: S \to F^2_S$, then there are unique isomorphisms $\hat{\iota_2}: F_S^1 \to F_S^2$ and $\hat{\iota_1}: F_S^2 \to F_S^1$ such that

    \[ \hat{\iota_2} \circ \iota_1 = \iota_2, \quad \hat{\iota_1} \circ \iota_2 = \iota_1, \quad \hat{\iota_1}^{-1} = \hat{\iota_2}\]
\end{thm}

\begin{proof}
    We let $F_S$ be the set of reduced words with letters given by elements of $S$, with the operation given by concatenation and reduction. Now, for any $f: S \to G$, where $G$ is a group, we define $\varphi$ as follows: we send the empty word $\varnothing$ to $1_G$, the identity element of $G$, and for any $s \in S$, 

    \[\varphi(s) = f(s)\]

    where we think of $s$ as a word consisting of a single letter in $F_S$. $\varphi$ can then be uniquely extended to all other words by the fact that it is a homomorphism. 

    We now show uniqueness. Setting $G = F_S^1$ and $f = \iota_1$ in the diagram from Definition \ref{Free Group Universal Property} gives us 

    \[\begin{tikzcd}
        & {F_S^1} \\
        S & {F_S^1}
        \arrow["\id_{F_S^1}", dashed, from=1-2, to=2-2]
        \arrow["\iota_1",from=2-1, to=1-2]
        \arrow["\iota_1"', from=2-1, to=2-2]
    \end{tikzcd}\]

    Similarly, setting $G = F_S^2$ and $f = \iota_2$, we get 

    \[\begin{tikzcd}
        & {F_S^1} \\
        S & {F_S^2}
        \arrow["\hat{\iota_2}", dashed, from=1-2, to=2-2]
        \arrow["\iota_1",from=2-1, to=1-2]
        \arrow["\iota_2"', from=2-1, to=2-2]
    \end{tikzcd}\]

    from which we get that $\hat{\iota_2} \circ \iota_1 = \iota_2$. Swapping $F_S^1$ and $F_S^2$  and their respective maps gives us $\hat{\iota_1}$ such that $\hat{\iota_1} \circ \iota_2 = \iota_1$.

    Notice that $\hat{\iota_1} \circ \hat{\iota_2}: F_S^1 \to F_S^1$ satisfies 

    \[ \hat{\iota_1} \circ \hat{\iota_2} \circ \iota_1 = \hat{\iota_1} \circ \iota_2 = \iota_1 \]

    satisfying the same equation that $\id_{F_S^1}$ does. By uniqueness, we get that $\hat{\iota_1} \circ \hat{\iota_2} = \id_{F_S^1}$. A similar argument shows that $\hat{\iota_2} \circ \hat{\iota_1} = \id_{F_S^2}$. This completes the proof. 
\end{proof}

The presence of a set $S$ seems to complicate our definition, as we must define free groups in terms of this generating set. However, we can greatly reduce the number of sets we must consider by making a key observation. Suppose $S,T$ are sets such that $|S| = |T|$, and let $F_S, F_T$ be the corresponding free groups. Recall that we construct each free group by sending each element of the generating set to a letter in the free group, and define the homomorphisms $\varphi$ by how they map these letters. As $S,T$ are the same size, there is a bijective map $h: S \to T$. More importantly, this map lifts to one on the free groups 

\[ \psi: F_S \to F_T, \quad \psi(\iota_S(s)) = \iota_T(h(s)) \]

As $h$ is a bijection on the generating sets, and each free group is uniquely determined by the letters these sets map to, $\psi$ is an isomorphism of the groups. We have just proven 

\begin{thm}\label{Free Groups From Generating Sets of Same Cardinality are Isomorphic}
    Every free group generated by sets of the same cardinality are isomorphic. 
\end{thm}

Because of this, we only need to consider free groups generated by sets of the form $\{1,2,\ldots, n\}$; the size of this set is called the free group's \textbf{rank}. We will thus denote these groups by $F_n$, called the \textbf{free group of rank $n$}. For the purposes of this paper, we will restrict our study to free groups of countable rank, meaning we consider $F_n$ for all $n \in \N$, as well as $F_\infty$, the free group with generators $x_i$ for all natural $i$.

The construction of free groups makes them naturally important in the study of arbitrary groups. The first to notice this fact was Walther von Dyck, one of the founders of Combinatorial Group Theory, and the first to study groups via generators and relations. It was through this study that he found the usefulness of free groups:

\begin{thm}\label{Groups are Quotients of Free Groups}
    Every group $G$ is the quotient of a free group. 
\end{thm}

\begin{proof}
    Let $G$ have a presentation given by $\langle g_1, g_2, \ldots, g_n | R \rangle$. Let $S = \{1, 2, \ldots, n\}$ and let $F_n$ be the free group generated by $S$. Consider the map 

    \[f: S \to G, \quad f(i) = g_i\]

    Then by the universal property, let $\varphi: F_n \to G$ be the unique homomorphism associated with $f$. It follows that

    \[\ker\varphi = \{g = g_1^{e_1}\cdots g_n^{e_n} : g = 1\} = R\]

    In other words, $\ker G$ is the set of all relations on $G$. Thus, we get that 
    
    \[G = F_n/\ker\varphi\]
\end{proof}

Because of von Dyck's Theorem, we are able to define free groups in a very simple manner. By construction, the relations within a free group $F_n$ are strictly those saying that $x_ix_i^{-1} = 1$ for any $i \in \{1,2,\ldots, n\}$. Thus, free groups have the simplest possible presentation, and can be defined as just a set of generators with no relations:

\begin{defn}\label{Free Groups as Reduced Words}
    The \textbf{free group of rank $n$} is the group $F_n$, the set of all reduced words in symbols $\{x_1, \ldots, x_n\}$, with operation given by concatenation and reduction. Its presentation is 

    \[\langle a_1, a_2, \ldots, a_n \rangle\]

    and the identity is the empty word $\varnothing$. 
\end{defn}

Becuase of this result, we are able to translate the algebraic definition of a group, that which uses the universal property, to one using reduced words, which are much easier to work with. 

\section{The Rank of Subgroups of $F_2$}

Rank is a group theoretic analogue of dimension for vector spaces. 

\begin{defn}
    The \textbf{rank} of a group $G$ is the size of the smallest possible generating set for $G$.

    \[\operatorname{rank}(G) = \min\{|X| : X \subseteq G, \langle X \rangle = G\}\]
\end{defn}

For a vector space $V$, any subspace $W$ will have at most dimension $\dim V$, which makes intuitive sense. However, despite being an analogue of dimension for groups, this property does not hold for arbitrary groups; it hold for special types of groups, such as finite abelian groups. 

The subgroups of free groups can oftentimes have larger rank than the free group itself. In particular, $F_2$, not only contains subgroups of larger rank, but they contain subgroups of every possilbe rank larger than 2, including subgroups of countably infinite rank!

\begin{thm}\label{Free Groups Contain Subgroups of Arbitrarily Large Rank}
    \begin{enumerate}[label=(\alph*)]
        \item For all $n > 2$, $F_n \subset F_2$. 
        \item $F_\infty \subset F_2$, where
        \[F_\infty = \langle a_1, a_2, a_3, \ldots \rangle\]
        is the free group of countably infinite rank. 
    \end{enumerate}
\end{thm}

\begin{proof}
    \begin{enumerate}[label=(\alph*)]
        \item We write $F_2 = \langle a,b \rangle$, and define 
        
        \[w_1 = a\]
        \[w_2 = b^{-1}ab\]
        \[w_3 = b^{-2}ab^{-2}\]
        \[\vdots\]
        \[w_n = b^{-n}ab^n\]

        With the operations of concatenation and reduction, these form the generating set of a group. It thus suffices to show that there are no relations between these generators. First notice that for any $w_i$ and any integer $k$:

        \begin{align*}
            w_i^k & = (b^{-i}ab^i)(b^{-i}ab^i)\cdots(b^{-i}ab^i) \\
            & = b^{-i}a(b^ib^{-i})a(b^ib^{-i})a\cdots a(b^ib^{-i})ab^i \\
            & = b^{-i}a^kb^i
        \end{align*}

        Now, let $w = w_{i_i}^{k_1}\cdots w_{i_l}^{e_l}$ be an arbitrary word. We have that 

        \begin{align*}
            w & = w_{i_i}^{k_1}\cdots w_{i_l}^{e_l} \\
            & = (b^{-i_1}a^{e_1}b^{i_1})(b^{-i_2}a^{e_2}b^{i_2})\cdots(b^{-i_l}a^{e_l}b^{i_l}) \\
            & = b^{-i_1}a^{e_1}b^{i_i - i_2}a^{e_2}b^{i_2 - i_3} \cdots b^{i_{l-1} - i_l}a^{e_l}b^{i_l}
        \end{align*}

        and this is fully reduced. Moreover, it clearly cannot be expressed as the reduced product of any other set other product of the generators. Therefore, we conclude that there are no relations between the $w_i$'s, and hence they generate the free group $F_n$, so $F_n \subset F_2$.

        \item The proof is identical to (a), with the generating set of $F_\infty$ being the set of $w_i$ for all $i \in \N$.
    \end{enumerate}
\end{proof}

One might ask if this result generalizes to free groups of rank higher than 2? In particular, for any $k \in \N$, is it true that $F_n \subset F_k$ for $n > k$? If so, we'd get a chain of inclusions

\[F_2 \supset F_3 \supset F_4 \supset \cdots\]

The answer is no, and follows from a result of Schreier: 

\begin{thm}[Schreier Index Formula]
    If $H$ is a subgroup of $F_n$ with index $e$, then $H$ has rank 
    \[1 + e(n-1)\]
\end{thm}