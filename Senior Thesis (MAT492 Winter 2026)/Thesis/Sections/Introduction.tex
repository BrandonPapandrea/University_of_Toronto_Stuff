\chapter{Introduction}

For any set $S$ and group $G$, there is a unique group up to isomorphism satisying the universal property, called the free group generated by $S$. If $|S| = n$, then we call this group the free group of rank $n$, and denote it as $F_n$.

Free groups were first studied as an example of Fuchsian groups, which are discrete subgroups of $PSL_2(\Z)$, but they did not become relevant on their own until the late 1800s. Recall that we may write any group using its group presentation. von Dyck, who was first to express groups using a presentation, showed that free groups have the simplest possible one: The free group $F_n$ has presentation 

\[\langle a_1, a_2, \ldots a_n \rangle\]

meaning it consists of $n$ generators with no relations between them; this is often used as an alternate definition of $F_n$. It is thus the case that every group is the quotient of a free group. The algebraic properties of free groups were investigated by Nielsen, who also gave them their name, whilst a comprehensive overview of free groups was given by Reidemeister in his 1932 book on combinatorial group theory.

The simple nature of free groups is equal parts useful and deceiving. Consdier The Word Problem. 

\begin{center}
    \textbf{The Word Problem}: For a group $G$, let $g$ be an element expressed as a product of $G$'s generators. Find an algorithm that shows, in finitely many steps, whether or not $g$ is the identity element.
\end{center}

While it is known that this problem is undecidable for arbitary groups, it is answerable if $G$ is a free group. Recall that a free group has no relations between the generators, apart from the obvious one that $a_ia_i^{-1} = 1$. Thus, given an arbitrary element of a free group, one can just check if any of these pairs exist, and eliminate them one at a time until none remain; if generators are still present, the element is not the identity, and if no generators are left, the element is the identity. Thus, while The Word Problem is undecidable, it is easily solvable in the case of free groups.  

This does not mean that free groups are a trivial class of groups. In fact, understanding their algebraic properties has been an area of study for over a hundred years. To demonstrate this, we consider two questions.

First, suppose we are given the presentation of an arbitrary group $G$. Can we easily check whether or not $G$ is a free group? This question was answered in the negative in the 1950s, where it was shown that such a question is undecidable. 

The second question is one that is often asked when a property of a structure is first defined: do the substructures of this structure also have this property? In particular, are the subgroups of a free group also free groups? This was answered in the positive in the 1920s, first in a restricted form by Nielsen, then the the full form independently by Dehn and Schreier However, such proofs were not trivial, and required high level topological or algebraic methods to get to a rigorous proof. Understanding the proofs would require understanding very high level results in group theory, far beyond the level of an undegraduate course in abstract algebra. 

Answers to the both of these questions do exist in simpler forms within the field of Combinatorial/Geometric Group Theory, which in part seeks to understand a group's algebraic properties by looking at how it acts on a space. This extra assumption on how a group acts upon a space makes these results much easier to understand at the undergraduate level. 

An answer to the first question can be found thanks to the Ping-Pong Lemma, which describes a sufficient condition for a group to be free, so long as it acts in a certain way on a set $X$:

\begin{thm}[The Ping-Pong Lemma]
    Let $\{g_1, g_2, \ldots, g_n\}$ generate a group $G$, which acts on a set $X$. If 

    \begin{enumerate}[label=(\roman*)]
        \item $X$ contains $n$ subsets $X_1, \ldots, X_n$ such that $X_i \cap X_j = \varnothing$, and
        \item $g_i^k(X_j) \subset X_i$ for all nonzero powers $k$ and $i \neq j$.
    \end{enumerate}

    Then $G$ is isomorphic to $F_n$. 
\end{thm}

More interestingly, an answer to question two is a corollary of a result proven by Serre in 1969, who showed that free groups may be classified by how they act on trees.

\begin{thm}[Serre's Theorem]
    $G$ is a free group if and only if it acts freely on a tree. 
\end{thm}

\begin{cor}[Nielsen-Schreier Theorem]
    Every subgroup of a free group is itself free. 
\end{cor}

In this paper, we present full proofs and explanations of both results. First, we prove the Ping-Pong Lemma both in the case of $n = 2$, and in the general case for arbitrary $n$. We also apply the Ping-Pong Lemma, and another result, to show that free groups of any rank, including one of infinite rank, lies within the matrix group $SL_2(\Z)$. Afterwards, we prove both sides of Serre's Theorem. In proving the forwards direction, we define a special graph for our free group to act on called the Cayley Graph, and then show that free groups acts freely on their corresponding Cayley Graphs. To prove the reverse direction, we show that every group acting on a tree induces a special tiling of the tree, before finding a free generating set using this tiling. 